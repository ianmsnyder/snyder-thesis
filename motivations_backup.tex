% !TEX root = ../thesis.tex

\chapter{Motivation for top partner searches}


% Keep context in mind here
%Reference previous results, how ATLAS and CMS differ

\section{Hierarchy Problem}%Top Partners} 
\\
%The Higgs Boson, discovered in 2012 at the LHC, was the final missing piece of the Standard Model.  However, the reason for its mass has not been determined.  The Higgs 
The Standard Model, as previously discussed, is an astoundingly accurate theory describing nature.  However it is still an incomplete theory, unable to account for massive neutrinos and unable to describe quantum gravity.  Another problem is that of the Higgs mass.  From equation ??? \color{red}{(to be added in SM section)}\color{black}{}, the electrically neutral part of the Higgs field is a complex scalar with potential: 

\begin{equation}
V=m_{H}^2 |H|^{2} + \lambda |H|^{4} 
\label{eq:higgspot}
\end{equation}

With $\lambda > 0$ and $m_{H}^2 < 0$, the vacuum expectation value (VEV) for $H$ at the minimum of the potential is non-vanishing, which is required be the SM.  This gives a solution of $ \langle H \rangle = \sqrt{-m_{H}^2/2\lambda}} $.  Also $\langle H \rangle \approx 174$ GeV, as determined experimentally by measured properties of the weak interactions, the Higgs mass is 125 GeV, $ \lambda = 0.126$ and $ m_{H}^2 = -(92.9 \mathrm{\ \ GeV})^{2})$.  The problem with these values is that the Higgs mass is corrected by each particle it couples to and the corrections become enormous. \\

As a scalar, the correction to $m_{H}^2$ from a fermion loop goes as: \color{red}{Fix scalar line here} \color{black}{} 

\begin{equation}
\feynmandiagram [inline=(a.base), layered layout, horizontal=b to c]
{ 
	a -- [scalar] b 
	-- [fermion, half left, looseness=1.6, edge label=\(f\)] c 
	-- [fermion, half left, looseness= 1.6] b,
	c -- [scalar] d,
};
= -\frac{\lambda_f^2}{8\pi^2}\Lambda_{\mathrm{UV}}^2
\label{eq:fermloop}
\end{equation}\\



where $\Lambda_{\mathrm{UV}}$ is the ultraviolet cutoff and can be interpreted as the energy scale where new physics comes in to alter the theory.   The mass is therefore quadratically sensitive\footnote{While the Higgs is the only particle that has this quadratic dependence in the SM, the other massive particles get their mass from the Higgs and therefore are indirectly affected.} to the cutoff scale; if the value of $\Lambda_{\mathrm{UV}}$ is on the order of the Planck mass then corrections to the Higgs mass is 30 orders of magnitude above the required value.  This means that the mass without any corrections, the \textit{bare mass}, must have a mass that precisely cancels these divergences.  This is called \textit{fine-tuning}.  \\

This fine-tuning is not considered to be \textit{natural}.  The most reasonable fix to this problem is to introduce new physics, the diagrams of which cancel these loop divergences.  This suggests a new symmetry to systematically cancel each contribution.  The particles that cancel the fermion loops need to be scalars to introduce the diagrams:  \\

\begin{equation}
\begin{tikzpicture}
\begin{feynman}[inline=(a.base)]
\diagram [horizontal=a to b, layered layout] {
	a -- [scalar] b -- [scalar] c
};
\draw [scalar] (b) arc [start angle=270, end angle=-90, radius=0.5cm] node[midway, above] {$\tilde{t}$}; % {\stop};
\end{feynman}
\end{tikzpicture}
= +\frac{\lambda_f^2}{8\pi^2}\Lambda_{\mathrm{UV}}^2
\label{eq:stoploop}
\end{equation} 

This is because the sign difference between the fermion and scalar loops leads to the cancellation of the fermion loops and also persists to higher order loop corrections.  The existence of such scalars arises naturally if there exists a symmetry relating bosons to fermions, called \textit{supersymmetry}.  



\section{Simplified Models}

Simplified models are designed to only involve a few new particles and interactions; these can be limits of more complex scenarios but with new particles integrated out.  They can be described by a few parameters related to collider physics, such as particle mass, cross-section, and branching fraction.  









%%%%%%%%%%%%%%%%%%%%%%%%%%%%%%

\section{Supersymmetry}

\subsection{Supermultiplets}

With this symmetry there is a transformation that turns a bosonic state to a fermionic one and vice versa; an operator, $Q$, that generates such a transformation with:

\begin{alignat}{2}
	Q|Fermion\rangle=&|Boson\rangle,\quad& Q|Boson\rangle=&|Fermion\rangle
\end{alignat}

must be an anti-commuting spinor.  Being a complex object, $Q$ and its Hermitian conjugate $Q^{\dagger}$ carry a spin angular momentum of 1/2, so sypersymmetry must be a spacetime symmetry.  The forms of these symmetries are restricted by the Haag-Lopuszanski-Sohnius extension of the Coleman-Mandula theorem and must satisfy the algebra (with spin indices suppressed) [susyprimer]:

\begin{alignat}{2}
	\{Q, Q^{\dagger}\} &= P^{\mu},\label{eq:QQDaggerOp}\\
	\{Q,Q\} &= \{Q^{\dagger}, Q^{\dagger}\} &&= 0,\label{eq:QQOp}\\
	[P^{\mu},Q] &= [P^{\mu},Q^{\dagger}] &&= 0\label{eq:PQOp}
\end{alignat}

where $P^{\mu}$ is the four-momentum generator of spacetime translations.  The fermion and boson states that are transformed to one another by $Q$ come in pairs called \textit{supermultiplets}, where the boson and fermion states are \textit{superpartners} of each other and are irreducible representations of the algebra.  Because the squared-mass operator, $-P^2$, commutes with the transformation operators, along with all spacetime rotation and translation operators, the superpartners must have the same mass.  The operators also commute with operators of gauge transformations, so superpartners also have the same electric charge, weak isospin, and color degrees of freedom.  \\

Additionally, the number of fermionic and bosonic degrees of freedom in a supermultiplet must be equal.  This means that following combinations of supermultiplets are possible, from most to least simple [susyprimer]: the \textit{chiral/matter/scalar} multiplet consisting of a Weyl fermion (two spin helicity states) and two real scalars (each with one degree of freedom); the \textit{gauge/vector} multiplet with two degrees of freedom consisting of a massless gauge boson (two helicity states) and a Weyl spin-1/2 fermion; and a gravatino supermultiplet, consisting of a spin-2 graviton (two helicity states) and a massless spin-3/2 gravatino (two helicity states).  Any other combination will reduce to these for the so-called Minimal Supersymmetric Supersymmetry Model, or MSSM where there is only one Q.  Other ``extended" supersymmetric theories do not reduce to these supermultiplets.  \\

\subsection{Particle Contents}

Only chiral supermultiplets can contain fermions whose left- and right-handed parts transform differently under the gauge group; since all the SM fermions have this property they must be members of the chiral multiplet and therefore the bosonic partners must be scalar particles, called \textit{sfermions}, named by adding an s (for ``scalar") in front of the standard model particle name, and not spin-1 vector particles.  All supersymmetric particles are noted by a tilde ($\sim$) over its letter symbolizing it. \\

Alternately, there must be two chiral supermultiplet for the Higgs, one to couple to up-type quarks ($H_u^+ , H_u^0$) and one to couple to down-type quarks and charged leptons ($H_d^0, H_d^-$).  Each of these have a spin-1/2 partner, named by adding ``-ino" to the end of the standard model partner, so are named \textit{higgsinos}.  \\

Finally, the gauge bosons  have gauge supermultiplets with spin-1/2 superpartners.  Similarly to the Higgs, the naming convention is to add ``-ino" to the end of the standard model partner, so are named \textit{gauginos}.  The partner for the gluon is the \textit{gluino} and the partners for the \textit{W} and \textit{B} bosons are the \textit{winos} and \textit{bino}.  Similarly to the way in which the \textit{W} and \textit{B} mix to make the $Z^0$ and $\gamma$ the winos and binos mix to make the zino and photino.  \\



\begin{table}[!htb]
	\centering
	\begin{tabular}{c|c|c|c|c}
		\hline
		\multirow{2}{*}{SM Particle}	& \multicolumn{2}{c|}{Field components}			& \multirow{2}{*}{$SU(3)_c$, $SU(2)_L$, $U(1)_Y$}		\\%			& \multirow{2}{*}{Comments} \\
		\cline{2-3}
									& \Gls{sm} 										& \Gls{susy} partners												&												 \\
		\hline\hline
		\multicolumn{5}{c}{Spin-1/2 quarks and spin-0 squarks} \\
		\hline
		$Q$							& $\begin{pmatrix} u_L&d_L\end{pmatrix}$		& $\begin{pmatrix} \tilde{u}_L&\tilde{d}_L\end{pmatrix}$			& $(3, 2, \frac{1}{6})$			\\ %& \multirow{3}{*}{$\times3$ generations}\\
		$\bar{u}$					& $u^{\dagger}_R$								& $\tilde{u}^*_R$													& $(\bar{3}, 1, -\frac{2}{3})$	& \\
		$\bar{d}$					& $d^{\dagger}_R$								& $\tilde{d}^*_R$													& $(\bar{3}, 1, \frac{1}{3})$	& \\
		\hline\hline
		\multicolumn{5}{c}{Spin-1/2 leptons and spin-0 sleptons} \\
		\hline
		$L$							& $\begin{pmatrix} \nu_L&e_L\end{pmatrix}$		& $\begin{pmatrix} \tilde{\nu}_L&\tilde{e}_L\end{pmatrix}$			& $(1, 2,-\frac{1}{2})$		\\	%& \multirow{2}{*}{$\times3$ generations}\\
		$\bar{e}$					& $e^{\dagger}_R$								& $\tilde{e}^*_R$													& $(1, 1, 1)$					\\%& \\
		\hline\hline
		\multicolumn{5}{c}{Spin-0 Higgs and spin-1/2 Higgsinos} \\
		\hline
		$H_u$						& $\begin{pmatrix} H_u^+&H_u^0\end{pmatrix}$	& $\begin{pmatrix} \widetilde{H}_u^+&\widetilde{H}_u^0\end{pmatrix}$& $(1, 2, +\frac{1}{2})$		\\%& \\
		$H_d$						& $\begin{pmatrix} H_d^0&H_d^-\end{pmatrix}$	& $\begin{pmatrix} \widetilde{H}_d^0&\widetilde{H}_d^-\end{pmatrix}$& $(1, 2, -\frac{1}{2})$		\\%& \\
		\hline\hline
		\multicolumn{5}{c}{Spin-1 gauge bosons and spin-1/2 gauginos} \\
		\hline
		Gluons						& $g$											& $\tilde{g}$															& $(8, 1, 0)$					\\%& \\
		\Wboson						& $\begin{pmatrix} W^{\pm}&W^0\end{pmatrix}$	& $\begin{pmatrix} \widetilde{W}^{\pm}&\widetilde{W}^0\end{pmatrix}$				& $(1, 3, 0)$					\\%& \\
		$B$							& $B^0$											& $\widetilde{B}^0$															& $(1, 1, 0)$					\\%& \\
		\hline\hline
	\end{tabular}
	\caption[Summary of \acrshort{mssm} particle contents]{Summary of {\mssm} particle contents, organized as pairs of known {\sm} particles and their proposed {\susy} partners.}
	\label{tab:smsusypartners}
\end{table}


%Lagrangian
\subsection{Interactions in the MSSM}

The MSSM has many chiral supermultiplets with gauge and non-gauge interactions.  The non-gauge couplings are very restricted by the requirement that the action be invariant under SUSY transformations.  Applying this restriction gives us the interacting Lagrangian:

\begin{equation}
	\Lagr_{\mathrm{int}} = ( -\frac{1}{2} W^{ij}\psi_{i}\psi_{j} + W^{i}F_{i}) + c.c.% x^{ij}F_{i}F_{j}) +c.c. -U
\end{equation}

where $W^{ij}$ and  $W^{i}$ are polynomials in scalar fields with degrees 1, 2, 0, and 4 and $\psi_{i}$ and $\psi_{j}$ are Weyl spinors.  $W$ is holomorphic in the complex field $\phi_{k}$ and can be written as:

\begin{equation}
	W^{ij} = M^{ij} + y^{ijk}\phi_{k}
\end{equation}

where $M^{ij}$ is a symmetric mass matrix for fermion fields, $y^{ijk}$ is a Yukawa coupling of a scalar field $\phi_{k}$.  We can write $M^{ij}$ as:

\begin{equation}
	W^{ij} = \frac{\delta}{\delta\phi_{i}\delta\phi_{j}}W
\end{equation}

where

\begin{equation}
	W = \frac{1}{2}M^{ij}\phi_{i}\phi_{j} + \frac{1}{6}y^{ijk}\phi_{i}\phi_{j}\phi_{k}
\end{equation}

and is called the superpotential.  This is a holomorphic function of the scalar field $\phi_{i}$ treated as complex variables.  \\

This gives the Lagrangian: 

\begin{equation}
\begin{split}
\Lagr =& \partial^\mu \phi^{*i} \partial_\mu \phi_i - V(\phi,\phi^*)
+ i \psi^{\dagger i} \bar{\sigma}^\mu \partial_\mu \psi_i
- \frac{1}{2} M^{ij} \psi_i\psi_j - \frac{1}{2} M_{ij}^{*} \psi^{\dagger i}\psi^{\dagger j} \\
&- \frac{1}{2} y^{ijk} \phi_i \psi_j \psi_k - \frac{1}{2} y_{ijk}^{*} \phi^{*i}
\psi^{\dagger j} \psi^{\dagger k}
\label{eq:lagrchiral}
\end{split}
\end{equation}

where the scalar potential $V(\phi,\phi^*)$ is:

\begin{align}
	V(\phi,\phi^*) =& W^k W_k^*
	\nonumber
	\\ 
	=&M^*_{ik} M^{kj} \phi^{*i} \phi_{j}
	+{1\over 2} M^{in} y_{jkn}^* \phi_i \phi^{*j} \phi^{*k}
	+{1\over 2} M_{in}^{*} y^{jkn} \phi^{*i} \phi_j \phi_k
	+{1\over 4} y^{ijn} y_{kln}^{*} \phi_i \phi_j \phi^{*k} \phi^{*l}
	\label{eq:ordpot}
\end{align}

% R parity
\subsection{R-Parity}

While the superpotential is minimal in that it is sufficient to produce a viable model, there are additional terms that are consistent with the theory but are omitted because they violate lepton or baryon number.  It is not desirable to simply take baryon and lepton conservation as a postulate because this is a consequence of renormalization in the SM.  However a new symmetry, called ``matter parity" can be introduced, defined as:

\begin{equation}
P_{M} = (-1)^{3(B-L)}
\end{equation}

where B and L are the baryon and lepton numbers respectively, for each particle in the theory.  Quark and lepton supermultiplets have $P_M=-1$ while the Higgs supermultiplets have $P_M=+1$.  Gauge bosons and gauginos, which do not have baryon or lepton number, are assigned $P_M=+1$.  A candidate term in the Lagrangian is only allowed if the product of all $P_M$ terms is +1.  This is a more exact and fundamental theory than baryon and lepton number conservation.  A new symmetry, ``R-parity," can be introduced to also account for conservation of spin as:

\begin{equation}
P_{R} = (-1)^{3(B-L) + 2s}
\end{equation}

where $s$ is the spin of the particle.  All the particles in the SM have positive, or even, R-parity, while all SUSY particles have negative, or odd, R-parity.  If R-parity is exactly conserved then there is no mixing between SM and SUSY particles.  Additionally, the lightest supersymmetric particle must be stable since it cannot decay to SM particles.  This makes it an attractive candidate for dark matter.  Additionally, SUSY particles can only be produced in pairs and SUSY particles heavier than the LSP must decay to an odd number of lighter SUSY particles. 


\subsection{Supersymmetry Breaking}

Particles in mutiplets will have the same mass in an unbroken supersymmetry; because this is not the case, as sparticles would have been easily discovered, supersymmetry is a broken symmetry in the vacuum state.  Specifically, it must be broken \textit{spontaneously}, meaning that the underlying model has a Langrangian density that is invariant under supersymmetry but a vacuum state that is not.  This way supersymmetry is hidden at low energies.  Since the relationship between the dimensionless couplings of the SM and supersymmetry has to be maintained in order for supersymmetry to be a solution to the hierarchy problem, the idea of ``soft" supersymmetry breaking comes in as: 


\begin{equation}
	\Lagr = \Lagr_{\mathrm{SUSY}} + \Lagr_{\mathrm{soft}}
\end{equation}

where $\Lagr_{\mathrm{SUSY}}$ contains all couplings and Yukawa interactions and preserves supersymmetry invariance and $\Lagr_{\mathrm{soft}}$ contains only mass terms and coupling parameters with positive mass dimension to naturally maintain a hierarchy between the electroweak scale and Planck scale.  The soft terms introduce new Higgs mass corrections as:


\begin{equation}
	\Delta m_H^2 = m_{\mathrm{soft}}^2 \left[ \frac{\lambda}{16\pi^2}\ln(\Lambda_{\mathrm{UV}}/m_{\mathrm{soft}})+\dots \right]
\end{equation}

where $\lambda$ is a dimensionless coupling term and $m_{\mathrm{soft}}$ is the mass scale associated with the soft terms.  In order to be a viable solution to the hierarchy problem the soft mass term, and thus the lightest supersymmetric particles, should be on the order of the TeV scale and not too much larger.  \\


The most general Lagrangian is then:

\begin{align}
\Lagr_{\rm soft}^{\rm MSSM} =& -\frac{1}{2}\left ( M_3 \widetilde{g} \widetilde{g}
+ M_2 \widetilde{W} \widetilde{W} + M_1 \widetilde{B}\widetilde{B}
+\mathrm{c.c.} \right )
\nonumber
\\
&
-\left ( \widetilde{\bar{u}} \,{\bf a_u}\, \widetilde{Q} H_u
- \widetilde{\bar{d}} \,{\bf a_d}\, \widetilde{Q} H_d
- \widetilde{\bar{e}} \,{\bf a_e}\, \widetilde{L} H_d
+ \mathrm{c.c.} \right ) 
\nonumber
\\
&
-\widetilde{Q}^\dagger \, {\bf m^2_{Q}}\, \widetilde{Q}
-\widetilde{L}^\dagger \,{\bf m^2_{L}}\,\widetilde{L}
-\widetilde{\bar{u}} \,{\bf m^2_{{\bar{u}}}}\, {\widetilde{\bar{u}}}^\dagger
-\widetilde{\bar{d}} \,{\bf m^2_{{\bar{d}}}} \, {\widetilde{\bar{d}}}^\dagger
-\widetilde{\bar{e}} \,{\bf m^2_{{\bar{e}}}}\, {\widetilde{\bar{e}}}^\dagger
\nonumber \\
&
- \, m_{H_u}^2 H_u^* H_u - m_{H_d}^2 H_d^* H_d
- \left ( m_3^2 H_u H_d + \mathrm{c.c.} \right )
\label{MSSMsoft}
\end{align}


where $M_1$, $M_2$, and $M_3$ are the gluino, wino, and bino mass terms; $a_u$, $a_d$, and $a_e$ is a complex 3x3 matrix in family space that correspond with the Yukawa couplings of the superpotential; \textbf{$m^{2}_{Q}$}, \textbf{$m^{2}_{\bar{u}}$}, \textbf{$m^{2}_{\bar{d}}$}, \textbf{$m^{2}_{L}$}, and \textbf{$m^{2}_{\bar{e}}$} is a 3x3 matrix in family space that can have complex entries but must be hermitian so the Lagrangian can be real; and SUSY-breaking contributions to the Higgs potential, $m^{2}_{H_{u}}$ and $m^{2}_{H_{d}}$. \\

This Lagrangian differs from the SUSY-preserving Lagrangian in that it introduces new parameters not present in the SM - in fact 105 masses, phases and mixing angles that do not have a counterpart in the ordinary SM, so SUSY breaking introduces an arbitrariness in the Lagrangian.  \\


%breaking scale

\subsection{Supersymmetry Breaking Scale}

Spontaneous SUSY breaking requires an extension of the MSSM to a hidden sector.  The hidden sector have small or no coupling to the visible sector, which are the chiral multiplets of the MSSM, but do share some interactions that are responsible for mediating the SUSY breaking.  \\

There are two primary competing proposals for what the mediating interactions may be.  The first is new physics that enter at the Planck scale, including gravity, called the Planck-scale-mediated supersymmetry breaking (PMSB).  In this scenario SUSY is broken in the hidden sector by a VEV ($\langle F \rangle$) and the soft terms are roughly $m_{soft} \sim \langle F \rangle/M_{P}$ because $m_{soft}$ must disappear as $\langle F \rangle \rightarrow 0$ where SUSY is unbroken and the limit where $M_{P} \rightarrow \infty$ where gravity is irrelevant.  If $m_{soft}$ is a few hundred GeV then $\langle F \rangle \sim 10^{10}$ or $10^{11}$ GeV. \\

The second proposal is that the flavor-bind mediating interactions for SUSY breaking are the electroweak and QCD gauge interactions in the SM, called gauge-mediated supersymmetry breaking (GMSB).  In this scenario the MSSM soft terms come from loop diagrams involving some messenger particles that are new chiral supermultiplets that couple to the SUSY-breaking VEV ($\langle F \rangle$).   Then for the MSSM soft terms,  $m_{soft} \sim \frac{\alpha_{a}}{4\pi}\frac{\langle F \rangle}{M_{\mathrm{mess}}}$ where $M_{\mathrm{mess}}$ is the mass of the messenger particles, and if $M_{\mathrm{mess}}$ and $\sqrt{\langle F \rangle}$ are comparable then the SUSY scale can be as low as $\langle F \rangle \sim 10^4$ GeV, which is clearly lower than the gravity-mediated case and gives $m_{\mathrm{soft}}$ the right order of magnitude. \\

There is also a possibility that the partitioning of the MSSM and SUSY breaking sectors are geographical; in this proposal there are extra spatial dimensions that are of the Kaluza-Klein type or warped type so that there is a physical distance that separates the visible and hidden sectors.  This can fit with string theory which suggests six extra spatial dimensions. \\

%% soft lagran



% Mass spectrum


% gauge coupling unification (add figure 6.8 in primer)

\subsection{Gauge Coupling Unification}

Ideally in the SM the gauge couplings unify at some energy scale; however, running the gauge couplings in the SM according to the renormalization group equations to high energy scales show that the couplings cross over but do not unify.  However, a feature of the MSSM is that the three gauge coupling constants are unified at an energy scale of $10^{15}$ or $10_^{16}$ GeV.  This is referred to as the grand unified theory (GUT) scale.  The unification is not exact but very close, as seen in Figure \ref{fig:unification} and adjustments can be made approaching the GUT scale.  

\begin{figure}[tbh]
	\centering
	\includegraphics[width=.7\textwidth,trim={0 0 0 0},clip]{gaugeunification}
	\caption{Renormalization group eveolution of inverse gauge couplings in the Standard Model (dashed) and MSSM (solid) with varying SUSY particle mass thresholds and couplings.  The SM couplings do not unify at any point while the MSSM couplings nearly unify at an energy scale of $10_^{16}$ GeV.  [susyprimer] \color{red}{Replace with higher resolution image.} }
	\label{fig:unification}
\end{figure}

% searching for susy/mass spectrum:
\subsection{Mass Spectrum}

In the MSSM there are two complex Higgs doublets rather than just one as in the SM.  Each get their own VEV as $\langle H_{u} \rangle = v_{u}$ and $\langle H_{d} \rangle = v_{d}$.  This are related to the mass of the {\Zboson} boson and electroweak gauge couplings as:

\begin{equation}
v_u^2 + v_d^2 = v^2 = \frac{2m_{\Zboson}^2}{(g^2+g^{'2})} \approx (246 \mathrm{GeV})^2
\end{equation}

with the ratio between the two \gls{vev}:

\begin{equation}
\tan \beta \equiv \frac{v_u}{v_d}
\end{equation}
%\frac{\pi}{2}$.

where both $v_u$ and $v_d$ are real and positive, and the value of $\tan \beta$ is not fixed by experiments but depends on Lagrangian parameters but must be $0 < \tan \beta < \frac{\pi}{2}$.  \\

The Higgs scalar fields in the MSSM consist of eight real scalar degrees of freedom.  After electroweak symmetry breaking three of these are the Nambu-Goldstone bosons, $G^{0}$ and $G^{\pm}$, that become longitudinal modes of the $Z^0$ and $W^{\pm}$ bosons.  The other five degrees Higgs scalar mass eigenstates include two CP-even neutral scalars, $h^0$ and $H^0$, one CP-odd neutral scalar $A^0$, one charge +1 $H^+$, and one which is its conjugate $H^-$. The discovered $\sim 125$ GeV Higgs is usually assumed to be the lighter $h^0$. \\

%Electrowinos  

The higgsinos and electroweak gauginos mix with each other because of electroweak symmetry breaking, similarly to the $W$ and $B$ bosons mixing in the SM.  The neutral higgsinos ($\widetilde{H}^0_u,\widetilde{H}^0_d$)  mix with the neutral gauginos ($\widetilde{W}^0$, $\widetilde{B}$) to create four neutral mass eigenstates called neutralinos, $\widetilde{\chi}^0_1-\widetilde{\chi}^0_4$, while the charged higgsinos and winos mix to form four charginos, $\widetilde{\chi}^\pm_1-\widetilde{\chi}^\pm_4$.  The exact mixing varies from model to model, so the mass and charge are the focus of experimental searches.  The lightest neutralino is assumed to be the lightest supersymmetric partner (LSP), unless there is a lighter gravitino or unless R-parity is not conserved, and is therefore the only particle in the MSSM that makes a good dark matter candidate.  \\

% gluino

The gluino is a color octet fermion and so cannot mix with any other particle in the MSSM. Its mass parameter, $M_{3}$ is related to the bino and wino mass parameters, $M_{1}$ and $M_{2}$ as $M_{3}:M_{2}:M_{1} \approx 6:2:1$ (following MSUGRA or GMSB boundary conditions) so the gluino is expected to be much heavier than the neutralinos and charginos.  \\

% sfermions

The squarks in the first and second families are nearly degenerate and are much heavier than the sleptons due to large radiative corrections from loops with the gluino, and can't be much lighter than the mass of the gluino.  Additionally, left-handed sfermions are heavier than right-handed sfermions due to effects from the renormalization group.  The third family sfermions are most likely the lightest sfermions due to contributions from their renormalization group equations. \\

Potentially dangerous flavor-changing and CP-violating effects in the MSSM can be avoided if SUSY breaking is universal.  In the case where the squark and slepton squared-mass mixing angles are flavor-blind, and each proportional to the 3x3 identity matrix in family space, the squark and slepton mixing angles are trivial and SUSY contributions to flavor changing currents are very small.  Additionally, assuming the scalar couplings are proportional to the corresponding Yukawa coupling matrix ensures that only squarks and sleptons of the third family can have large scalar couplings.  Furthermore, large CP-violating effects can be avoided by assuming soft parameters do not introduce new complex phases.  These conditions make up a weak version of the soft supersymmetry-breaking universality hypothesis and has fewer parameters than the general case.  There are several alternatives to the universality hypothesis but add complexity or require very heavy masses. \\

\begin{figure}[tbh]
	\centering
	\includegraphics[width=.5\textwidth,trim={0 0 0 0},clip]{massspectrum}
	\caption{A possible mass spectrum for the MSSM.  This is one particular model for GMSB; different models vary greatly, including in mass scales, which is not shown.  [susyprimer] \color{red}{Replace with higher resolution image.  Add other possible spectra?} }
	\label{fig:massspect}
\end{figure}
% squark decays

The decay of a squark to a quark and gluino will dominate if kinematically allowed because it has QCD strength.  Otherwise squarks can decay to a quark and neutralino or chargino, and the decay to a quark and LSP is kinematically favored and can dominate for right-handed squarks if the LSP is mostly bino.  Left-handed squarks may prefer decaying to heavier neutralinos and charginos because the coupling to winos is stronger than binos.  \\

\subsection{Search for Direct Pair-Produced Stops}

The large Yukawa couplings tend to drive down the masses of the third generation squarks in the renormalization group equations faster than the first two generations, making them lighter and easier to search for.  Additionally, the large Yukawa couplings make the masses of the third generation squarks have the largest impact on the mass of the Higgs; if naturalness is to be maintained then their masses should be small.  \\

This means that to avoid fine-tuning the stop mass should be small.  As it has the highest impact on naturalness, and is also the lightest squark, it is a good particle to search for. \color{red}{Some more motivation here?}\color{black}{}

\subsection{Simplified Models} %https://arxiv.org/pdf/0810.3921.pdf

It's not feasible to scan the complete space of the MSSM because of the large parameter space available.  Also searching for a more specific model such as GMSB is also not feasible since it does not allow for mass reordering.  Therefore simplified models are used with only a few parameters to vary and several benchmark mass points.  This gives a good coarse-level description of SUSY physics and deviations from the models can be used to characterize underlying physics.  Generally a simplified model will have just one pair-produced species of particles and only a few steps in the decay chain.  Doing this makes developing search strategies and interpreting results easier.  \\

Results from simplified models can then be reinterpreted for more complex models.  For example, Phenomenological MSSM (pMSSM) uses theoretical assumptions and experimental results to motivate reducing the number of free parameters from 105 to 19.  Each set of values for the 19 values is a model point and models can be excluded by sampling the parameter grid and applying results of analysis.  \\






%As a scalar particle, the Higgs mass correction goes as:
 %Because the top quark is so much more massive than other fundamental particles (nearly 40 times heavier than the next heaviest fermion) and because of the large top Yukawa coupling,  its mass essentially sets the mass correction for the higgs. 

%\begin{equation}
%\feynmandiagram [inline=(a.base), layered layout, horizontal=b to c]
%{ 
%	a -- [scalar] b 
%	-- [fermion, half left, looseness=1.6, edge label=\(t\)] c 
%	-- [fermion, half left, looseness=1.6] b,
%	c -- [scalar] d,
%};
%= -\frac{6y_t^2}{16\pi^2}\Lambda_{\mathrm{UV}}^2
%\label{eq:toploop}
%\end{equation}

