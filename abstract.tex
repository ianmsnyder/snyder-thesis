\abstract{This dissertation presents a search for pair production of a scalar partner to the top
  quark in proton-proton collisions at the ATLAS detector at the Large Hadron Collider (LHC) located in Geneva, Switzerland.  The LHC is a hadron collider which collides two beams of protons with a center of mass energy of 13 TeV.  The ATLAS detector is a general purpose detector and is one of four detectors at the LHC. The data was recorded during Run 2 with a total of 36.1 \ifb\ at a center-of-mass energy of ${\rts=13\tev}$.
In supersymmetry, the scalar partner to the top quark is the stop, which decays to a top quark and neutralino or to a bottom quark and chargino.  The experimental signature considered is four or more jets plus missing transverse
  momentum.  The data yielded no significant excess over the Standard Model background
  expectation, and exclusion limits are reported in terms of the stop and neutralino masses.  Assuming a branching fraction of $100\%$ to a top quark and neutralino, stop masses in the range 450-1000 GeV are excluded for neutralino masses below
  $160\GeV$. In the case where the stop mass is close to the top mass plus the neutralino mass, masses between 235-590 GeV are excluded. The results are also interpreted in terms of the Phenomenological Minimal Supersymmetric Standard Model. \\
  
  Additionally, work on an upgrade to the ATLAS trigger system, the Global Feature Extractor, is presented.  This upgrade will have the unique ability to scan over the entire calorimeter to trigger on global variables and large radius jets.  A scheme was developed to calibrate the gFEX that also reduces pileup noise.}
  
  