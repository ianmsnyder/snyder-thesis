\chapter{Introduction}
\label{ch:intro}

% Gentle introduction to the standard model here - more details in an appendix
% Use flaws in standard model as motivation

% Intro paragraph

Ever since the ancient Greeks philosophized on the smallest bit of materials, people have wondered what are the basic building blocks of matter.  With the development of science, the search for answers changed from philosophizing to experimentation and in the last several centuries atoms were discovered.  Of course, the atom is not the most fundamental building block of matter; its nucleus is made up of protons and neutrons, with electrons orbiting it.  Protons and neutrons are in turn made up of quarks, and whether these are fundamental and point-like is still being investigated.  \\

Additional quarks and leptons that do not exist in ordinary matter have also been discovered, along with other fundamental particles.  These particles and the forces that determine their interactions make up the Standard Model of Particle Physics (SM) and are shown in Table \ref{tab:SMparticles}.  These particles are heavier than those found in ordinary matter and are highly unstable.  In order to create these particles experimentally physicists need to reach higher and higher energies.  To achieve these energies, colliders accelerate particles such as protons to a very high momentum and then impact them with other particles.  Experimental particle physicists develop and use these colliders, which are among the largest experiments in history.  \\

Particle physics is not just useful for the very small; while particle physicists were probing the building blocks of matter, astronomers and cosmologists were looking to the skies to understand the origin and fate of our universe.  The particles and interactions in the SM also govern the life cycle of stars and the creation of heavy elements.  Dark matter, which is known to exist but not predicted by the SM, determines the large-scale structure of the universe.  Thus particle physics attempts to describe both the very small and very large. \\

This chapter will introduce the Standard Model and discuss SM processes relevant to the topic of this thesis, as well as present physics that the SM fails to explain, and Chapter \ref{ch:motivations} discusses a possible solution to a few of the most glaring issues.  Chapter \ref{ch:experiment} presents the experimental apparatus used in this search, and Chapter \ref{ch:gFEX} discusses one of the upgrades to the detector, the Global Feature Extractor.  Chapter \ref{ch:analysisOverview} gives an overview of the stop search and Chapter \ref{ch:combinedPerformance} describes how events are reconstructed.  Chapter \ref{ch:analysis} describes the analysis in detail and Chapter \ref{ch:conclusion} concludes the dissertation and presents the outlooks for the future of the analysis.  \\


%Experimental particle physics is the study of the most fundamental particles in nature and their interactions.   

%Particle physics strives to understand the very large but studying the very small.  The goal to completely understand the universe, including its history and future, and the very nature of reality is at its focus.  The largest experiments in the history of science have been built in order to measure properties of the microscopic, and the results have been astounding.  The most accurate theory in physics describes the most fundamental of particles, such as quarks, which make up protons and neutrons, which along with electrons are the building blocks of all common matter.

\section{The Standard Model}
% Audience: Boyana

The Standard Model of Particle Physics has been developed and tested by many experiments and stands as our most accurate theory describing the microscopic world.  In the SM all matter is made up of fermions (spin-$\frac{1}{2} $ particles).  Fermions are composed of quarks and leptons.  \\% Quarks make up hadrons, which include mesons and baryons. \\ Particles made up of quarks are called hadrons.

Ordinary matter is made up of the first generation; protons ($uud$), neutrons ($udd$), and electrons make up atoms.  The remaining quarks make up more exotic and short-lived hadrons we see in collider experiments and in cosmic radiation.  The remaining leptons, with the exception of neutrinos, can be thought of as heavy versions of the electron and are unstable.  A summary of all particles in the SM can be found in Table \ref{tab:SMparticles}. \\

\begin{table}[!htb]
	\centering
	\caption{Particles of the Standard Model\cite{pdg}.}
	\begin{tabular}{c||c|c|c|c|}
		\hline 
			&	Particle	&	Spin		&	Charge	&	Mass \\
		\hline \hline
		Quarks	&	&	&	&	\\
		\hline
		$u$ family	&	$u$	&	$\frac{1}{2}$	&	$\frac{2}{3}$	&	$2.2\substack{+0.6 \\ -0.4}$ MeV \\
				&	$c$	&		&		&	$1.28 \pm 0.03$ GeV \\
				&	$t$	&		&		&	$173.1 \pm 0.6$ GeV \\

		$d$ family	&	$d$	&	$\frac{1}{2}$	&	$-\frac{1}{3}$	&	$4.7\substack{+0.5 \\ -0.4}$ MeV \\
				&	$s$	&		&		&	$96\substack{+8 \\ -4}$ MeV \\
				&	$b$	&		&		&	$4.18\substack{+0.04 \\ -0.03}$ GeV \\
		\hline		
		Leptons 	&	&	&	&	\\
		\hline
		$e$ family	&	$e$	&	$\frac{1}{2}$	&	-1	&	$0.5109989461 \pm 0.000000003$ MeV \\
				&	$v_{e}$	&		&	0	&	$<2$ eV \\
		$\mu$ family	&	$\mu$	&	$\frac{1}{2}$	&	-1	&	$105.6583745 \pm 0.0000024$ MeV \\
				&	$v_{\mu}$	&		&	0	&	$<2$ eV \\	
		$\tau$ family	&	$\tau$	&	$\frac{1}{2}$	&	-1	&	$1776.86 \pm 0.12$ GeV \\
				&	$v_{\tau}$	&		&	0	&	$<2$ eV \\	
		\hline
		Bosons 	&	&	&	&	\\
		\hline
		Vector 	&	$\gamma$	&	1	&	0	&	$<10^{-18}$ eV\\
				&	$g$			&	1	&	0	&	0 \\
				&	$W$		&	1	&	$\pm$1	&	$80.385 \pm 0.015$ GeV \\
				&	$Z$		&	1	&	0		&	$91.1876 \pm 0.0021$ GeV \\
		Scalar	&	Higgs	&	0	&	0		&	$125.09 \pm 0.21 \pm 0.11$ GeV \\
		\hline \hline			
	\end{tabular}
	\label{tab:SMparticles}
\end{table}		


%Next sections:
% Quarks & Leptons

\subsection{Force Carriers}

Vector bosons (integer spin) act as force carriers in the SM.  Force carriers are quanta of energy of a particular field and transfer forces between particles. \\ %Photons act as the force carrier for electrodynamics and act between electrically charged particles, such as between electrons in Quantum Electrodynamics, or QED.  The $W$ and $Z$ bosons are massive bosons that are the force carriers for the weak force and can interact with all the particles in the SM.  Gluons carry ``color'' charge and are the force carriers for the strong force between gluons, a theory called Quantum Chromodynamics, or QCD.  The Higgs boson acts to give mass to the massive particles in the SM. \\


% In experimental chapter: 

	\subsubsection{Photons}
	
	Photons ($\gamma$) are massless, chargeless, and are the force-carriers in Quantum Electrodynamics (QED).  QED governs the interactions of charged matter and describes how photons and matter interact.  The coupling constant, which indicates coupling strength, is of order 10\textsuperscript{-2}.\\
	
	In addition to being important in production and decays of particles, QED interactions are used in detectors.  For example, calorimeters take advantage of the following processes:
	
	\begin{itemize}
		\item Photoelectric effect: photons are absorbed by a material and charge carriers, usually electrons, are emitted by the material
		\item Compton scattering: a particle absorbs a photon, recoils and emits a photon with less energy
		\item Pair production: a photon of sufficient energy decays into an electron and positron
		\item Bremsstrahlung: a charged particle loses energy by emitting a photon when deflected by an electric or magnetic field
	\end{itemize}
	
	These interactions cause a cascade, or shower, of particles whose energy can be measured.  This is further discussed in Section \ref{sec:calorimeters}. \\
	
	\subsubsection{Gluons}
	
	Gluons are also massless and chargeless and are the force-carriers in Quantum Chromodynamics (QCD), which describes strong interactions between quarks in the SM.  Gluons carry color, the analog to electric charge in the strong force (although photons do not carry charge while gluons carry color).  The coupling constant is for the strong force is of order 1 but is limited to short distances.  \\ %The coupling constant is a number that determines the strength of the force in an interaction.  
	
	Gluons also play a main role in particle production in hadron colliders.  Since both beams of particles in the collision at the LHC are protons, compared to protons and antiprotons at the Tevatron, there are very few antiquarks to annihilate with quarks to produce particles.  Instead, most production is from gluon fusion, in which two gluons fuse into a single high energy gluon which in turn can decay into other particles. \\% that couple to the strong force. \\
	
	Unlike the other forces, the strong force increases as two quarks are pulled apart, and a quark-antiquark pair will be created from input energy before quarks can be free.  Quarks therefore exist in bound states, hadrons, with either a quark-antiquark pair, called mesons, or sets of three quarks (antiquarks) called baryons (antibaryons).  \\
	
	In a collision a highly energetic quark or gluon can fragment into a collimated spray of hadrons, called jets.  Production of two back-to-back jets, called a dijet event, are very common at the LHC, as are multijet events.  By measuring the energy and direction of a jet, one can measure properties of the original parton.  Jets are discussed more in Section \ref{sec:jets}.  \\
	
	It is possible for a particle to emit radiation prior to annihilating with the oncoming beam of particles; this is called initial-state radiation (ISR).  When a final state particle emits radiation it is referred to as final-state radiation (FSR).  ISR can change the momentum of the particles produced as the interacting particles recoil against the ISR particles.   \\
	
	
	\subsubsection{\boldmath$W$ and \boldmath$Z$ Bosons}
	
	The $W$ and $Z$ bosons are the force carriers for the weak force.  Unlike the other force carriers, these are very massive and the $W$ bosons are also charged.  Because the the vector bosons have a large mass, they are short-lived with a lifetime under 10\textsuperscript{-24}s and a coupling constant on the order of 10\textsuperscript{-6} at low energies, although at high energy colliders this coupling can be similar to that of the strong force.  The weak force is the only force that affects every fermion in the SM and weak interactions are the only interactions that allow for a particle to change its flavor.  \\ %Additionally, the weak force violates parity symmetry and charge-parity symmetry.  
	
	The vector bosons are primarily produced at the LHC in Drell-Yan processes, where the vector boson is a product of quark-antiquark annihilation.  Processes that can produce a $Z$ boson can also occur with heavy flavor jets, such as $b$-jets as shown in Figure \ref{fig:zplusjets}.  $Z$ bosons can also be emitted from an off-shell quark, as shown in Figure \ref{fig:ttz} where a $Z$ is emitted from an off-shell top quark.  $Z$ bosons decay to leptons or quarks about $\sim$80\% of the time and to a neutrino-antineutrino pair $\sim$20\% of the time\cite{pdg}.  \\%Additionally, off-shell quarks produced in an event can emit a $Z$ bosons, which decays as previously stated.  \\
	
\begin{figure}[h!]
  \centering
	\includegraphics[width=0.3\textwidth]{./zPlusJets}
\caption[$Z$ with jets production mechanism]{\label{fig:zplusjets}{ One possible mechanism for creation of $Z$ with jets. }} 
\end{figure}	
	
	Similarly $W$ bosons can be produced in association with heavy flavor quarks.  $W$ bosons decay leptonically, to a lepton and lepton neutrino, $\sim$33\% of the time and decay hadronically, to a quark-antiquark pair, the remainder of the time.  It is possible for a pair of $W$ bosons to be produced as well; these are called diboson events and are shown in Figure \ref{fig:diboson}. \\
	
\begin{figure}[h!]
  \centering
	\includegraphics[width=0.7\textwidth]{./diboson_background}
\caption[Diboson production]{\label{fig:diboson}{ Examples of bosons produced in pairs in an event. }} 
\end{figure}	
	


\subsection{Quarks and Leptons}

The quarks and leptons are comprised of three generations as shown in Table \ref{tab:generations}.  The first generation of fermions include up ($u$) quarks, down ($d$) quarks, electrons ($e^{-}$) and electron neutrinos ($\nu_{e}$).  The second and third families, with exception of the neutrinos, are heavier and will decay to stable particles.  There also exists for each matter particle an antimatter particle having that same masses and quantum numbers as the corresponding matter particle except with opposite charge.  \\

\begin{table}[!htb]
	\centering
	\caption{Generations of the Standard Model.}
	\begin{tabular}{c||c|c|c|}
		\hline 
			&	First	&	Second		&	Third \\
		\hline \hline
		up-type quarks	&	$u$ 	 &	$c$ &	$t$	\\
		down-type	 quarks &	$d$ &	$s$ &	$b$	\\
		charged leptons 	& $e$	&	$\mu$	&	$\tau$ \\
		neutral leptons	&	 $\nu_e$	& $\nu_{\mu}$	&	$\nu_{\tau}$	\\
		\hline \hline			
	\end{tabular}
	\label{tab:generations}
\end{table}		


%\footnote{In the SM neutrinos are massless, but this will be discusses in a later section.}

% SM force carriers (maybe before hadronization)
	% subsection on each force carrier (photons, QED) (gluons, QCD), 
		%% Hadronization and jets (before it hits detector) (what a jet is, think broadly) (Move into QCD section? need to introduce gluon)
		
	
	% Higgs (W, Z, H)
	
	\subsubsection{The Top Quark}
	
The top quark was discovered in 1995 at the Tevatron\cite{topdiscovery}.  Its distinguishing feature is its extremely heavy mass at 173.1 GeV, compared to the next heaviest quark, the bottom quark, with a mass of 4.5 GeV or the proton with a mass of about 1 GeV.  The lightest quark is the up quark and has a mass of 2.4 MeV which is 0.0001\% the mass of the top quark..  Due to its heavy mass, the top quark is extremely unstable and have a lifetime of 5 $\times$ 10\textsuperscript{-25} seconds. This is shorter than the time to hadronize and so there are no top hadrons.  Because they couple strongly to bottom quarks, they nearly always decay immediately to a bottom quark and a $W$ boson\footnote{The theoretical rate of top to bottom is 0.999 and the experimentally measured rate is 1.021$\pm$0.032\cite{pdg}.}.  Bottom quarks in turn have a longer lifetime because of the small decay rates to up and charm quarks, and so we can detect bottom quarks in detectors by finding a secondary vertex which is displaced from the primary vertex.  See Section \ref{sec:btag} for more details. \\

\begin{figure}[h!]
  \centering
	\includegraphics[width=.5\textwidth,trim={0 0 0 0},clip]{./topDecays}
\caption[Branching fractions of the decay channels of the top quark]{\label{fig:dmMake}{Branching fractions of the decay channels of the top quark (figure from \cite{topquarkdecaysPieChart}).}} 
\end{figure}


Top quarks can be produced either in pairs, as in Figure \ref{fig:ttbarfeynman} or as a single top with another quark, such as a bottom quark.  After production top quarks decay to a bottom quark and a $W$ boson.  $W$ bosons decay to either a lepton and neutrino or a quark-antiquark pair the conserves charge (such as a $u\bar{d}$).  In the case of the quark-antiquark pair, each of these quarks will hadronize and the final result of the top decay is three jets including a bottom jet.  \\

\begin{figure}[h!]
  \centering
	\includegraphics[width=.4\textwidth]{./ttbar_background}
\caption[Top-antitop pair production]{\label{fig:ttbarfeynman}{ Top-antitop pair production with one $W$ boson decaying leptonically and one decaying hadronically. }} 
\end{figure}

%single top, ttbar diagrams
Top quarks, if off their mass shell, can also emit a $Z$ boson, as shown in Figure \ref{fig:ttz}. \\

\begin{figure}[h!]
  \centering
	\includegraphics[width=0.9\textwidth]{./ttZ}
\caption[$t\bar{t}+Z$ production]{\label{fig:ttz}{ Two possible production modes for a top-antitop pair and a $Z$ boson. }} 
\end{figure}
	
	

\subsection{Electroweak Physics and Higgs Boson} %(Higgs Mechanism) 	

	Above about 10\textsuperscript{15} GeV the weak and electromagnetic forces are unified to a single force.  At this scale the massless gauge bosons are the $W_{1}$, $W_{2}$, $W_{3}$, and the $B$ bosons.  After electroweak symmetry breaking the $W_{1}$ and $W_{2}$ mix to form the $W^{\pm}$ bosons and the $B$ and $W_{3}$ mix to form the $Z$ boson and the photon.  Therefore the electromagnetic and weak forces can be thought of as two aspects of one unified force.  \\
	
	The process by which the vector bosons gain mass is called the \textit{Higgs mechanism}.  The Higgs mechanism occurs whenever a field has a nonzero vacuum expectation value (vev).  In the SM this only occurs with the field of the Higgs boson since the Higgs field has a nonzero value everywhere.  This breaks the symmetry of the electroweak interaction and gives mass to the gauge bosons.\\%, making Yukawa coupling terms, which describes the coupling between the Higgs field and massless quark and lepton fields, into mass terms.  \\%This is due to the interaction of the gauge fields with the nonzero vacuum expectation value of the Higgs field.  \\
		
		Fermions and vector bosons acquire mass by interacting with the Higgs boson, and the coupling to the Higgs is proportional to the mass of the particle.  This coupling is referred to as Yukawa coupling and the top quark, being the most massive particle in the SM, also has the largest Yukawa coupling with a value of about 1. \\
		
		The Higgs boson was discovered in 2012 at ATLAS\cite{HiggsAtlas} (and CMS\cite{cmshiggsdiscovery}) with a mass of 126.0 $\pm$ 0.4(stat) $\pm$ 0.4(sys) GeV\cite{HiggsAtlas}. This was the results of four decades of searches at the Tevatron, LEP, and finally the LHC, and is the first and only fundamental scalar that has discovered.  Figure \ref{fig:higgsdecays} shows the branching fractions of the Higgs boson.  The Higgs was discovered by a combination of searches in the $H \ra ZZ^{(*)} \ra 4l,\ H \ra \gamma\gamma, \ H \ra WW^{(*)} \ra e \nu e \nu, \ H \ra b\bar{b},\ \mathrm{and}\ H \ra \tau^{+}\tau^{-}$ channels. \\
		
\begin{figure}[h!]
  \centering
	\includegraphics[width=0.9\textwidth]{./figures/Higgs_BR_LM.eps}
\caption[Higgs boson decay modes]{\label{fig:higgsdecays}{ Decay modes for the Higgs boson as a function of its mass (figure from the ATLAS Collaboration). }} 
\end{figure}	
	
		

% Important SM processes
	% Top quarks + decays - statement about top-> bottom w/reference, other SM processes for backgrounds
	

%Additionally, jets can create a lot of particles, including neutrinos.  

	
		% W+jets, Z+jets, - vector boson diagrams + decay, ttbar diagram
		% all backgrounds - processes that end up being backgrounds
		% Multijet background
		% ISR (production, intro to compressed region)
		
		
		
		% include statement that processes are important to analysis
	%include diagrams to refer back to - all feynman diagrams

% Think about flow, move sections around if necessary - how to build up


% Put any extras in appendix


\section{Beyond the Standard Model}

The Standard Model is an astoundingly accurate theory describing nature.  However it is still an incomplete theory; for example it is unable to account for massive neutrinos. \\% and unable to describe quantum gravity.

%\subsubsection{Neutrino Mass}

Neutrinos are unique in that they are composed of combinations of mass states.  Because of this fact neutrinos will oscillate from one flavor to another\cite{SuperKneutrinos}, for example electron neutrino to muon neutrino.  This also means that there are mass differences between the neutrino families, and thus that neutrinos have mass, and the sum of the masses are currently constrained to less than 0.17 eV.\cite{pdg}.  This is not predicted by the SM, which predicts massless neutrinos. \\

Major flaws of the SM include the failure to account for dark matter, lack of gauge unification, and the hierarchy problem.  \\

\subsection{Dark Matter}

The SM also only accurately describes the 4.8\% of the energy density of the universe as baryonic matter makes up that small fraction; dark energy makes up 69.4\% and dark matter makes up 25.8\%\cite{planckpaper}, neither of which are explained by the SM.  The evidence for the existence of dark energy is in the increasingly rapid expansion of the universe.  \\

Dark matter was proposed in the 20\textsuperscript{th} century, beginning in 1922 when Jacobus Kapteyn suggested it after observing stellar velocities\cite{dm1} and again by Fritz Zwicky in 1933 who applied the virial theorem to galaxy rotation and found that most matter in galaxies must be made up of dark matter\cite{dm2}.  It was observed that the stellar velocity was inconsistent with the amount of matter as measured by the brightness of the galaxies and that much more matter was needed to hold the galaxies together.  Therefore, there must be some unseen matter.  Evidence from galaxy rotation curves came in 1939\cite{dm3} and again with more accurate measurements in 1980\cite{dm4}.  It was found that the rotation velocity becomes approximately constant as the radius increases, which implies a halo of dark matter with the amount of matter proportional to the radius from the center of the galaxy.  Dark matter also has a large effect on the large-scale structure of the universe; by clumping together in the early universe it also gravitationally attracted ordinary matter and is therefore responsible for galaxy formation.  \\

One of the largest pieces of evidence comes from the Bullet Cluster\cite{bulletcluster}, two colliding galaxy clusters.  The stars in the clusters passed through the collisions and slightly slowed while gasses, which make up most of the matter of the galaxies, interacted electromagnetically and was slowed much more.  Measurements from gravitational lensing showed that most of the matter in the clusters were separated from the baryonic matter, where dark matter had passed through without being slowed.  \\

While the issue of dark energy is mostly relegated to the realm of cosmology, the issue of dark matter is a problem of particle physics with several approaches, either producing dark matter in colliders, observing possible dark matter decays in cosmic ray detectors, and detecting interactions of dark matter with baryonic matter - sometimes referred to in the field as ``make it, break it, shake it" approaches to detecting dark matter, as shown in Figure \ref{fig:dmMake}.  ``Making it" means to collide SM particles in an accelerator to produce DM particles; ``breaking it" means to observe DM particles annihilating to produce SM particles; and ``shake it" refers to direct detection where a DM particle scatters off a SM particle.  Since we should be able to create dark matter particles in colliders, searching for dark matter particles and having a viable model to explain it is a high priority.  \\


\begin{figure}[h!]
  \centering
	\includegraphics[width=.6\textwidth,trim={10cm 4.5cm 10cm 13cm},clip]{./SM_DM_annihilation.pdf}
\caption[Make it, break it, shake it]{\label{fig:dmMake}{Three options for detecting dark matter: collider production, where SM particles annihilate to dark matter; direct detection, where dark matter particles scatter off SM particles; and indirect detection, where dark matter particles annihilate to SM particles.  }} 
\end{figure}

\subsection{Gauge Unification}
\label{gaugeUni}

Since observing that the electromagnetic and weak forces are unified at high energies, a goal of theoretical physics is for all three gauge forces to be unified at some energy.  Unfortunately the in the SM this fails, as shown in Figure \ref{fig:gaugeUni}, and so there is much work to develop a grand unified theory (GUT) that fixes this.  \\

\begin{figure}[h!]
  \centering
	\includegraphics[width=.5\textwidth]{./gaugeUnification.pdf}
\caption[Coupling evolution]{\label{fig:gaugeUni}{ Evolution of the inverse of the three couplings with increasing energy (figure from \cite{Kazakov}).  Any two of the three forces unify at sufficient energy but a modification to the SM is needed so that the three couplings unify at the same energy.  }} 
\end{figure}


\subsection{Hierarchy Problem}
\label{sec:HierarchyProblem}

The Higgs boson couples to every particle with mass and large quantum corrections should make its mass enormous compared to its measured mass of 125.09 GeV.  As a scalar, the correction to $m_{H}^2$ from a fermion loop goes as: %\color{red}{Fix scalar line here} \color{black}{} 

%\begin{align}
\begin{equation}
\begin{gathered}
\begin{tikzpicture}
\begin{feynman}[inline=(a.base)]
\vertex (pf);
\vertex [right=3cm of pf] (qf1);

%\feynmandiagram [inline=(a.base), layered layout, horizontal=b to c, tree layout]
\diagram[inline=(a.base), horizontal=b to c, layered layout]
{ 
	a [particle = $h^0$] -- [scalar] b 
	-- [fermion, half left, looseness=1.6, edge label=\(f\)] c 
	-- [fermion, half left, looseness=1.6] b,
	c -- [scalar] {(qf1)},
};
\end{feynman}
\end{tikzpicture}
\end{gathered}= -\frac{\lambda_f^2}{8\pi^2}\Lambda_{\mathrm{UV}}^2
\label{eq:fermloop}
\end{equation}
%\end{align}\\

where $\Lambda_{\mathrm{UV}}$ is the ultraviolet cutoff and can be interpreted as the energy scale where new physics comes in to alter the theory.   The mass is therefore quadratically sensitive\footnote{While the Higgs is the only particle that has this quadratic dependence in the SM, the other massive particles get their mass from the Higgs and therefore are indirectly affected.} to the cutoff scale; if the value of $\Lambda_{\mathrm{UV}}$ is on the order of the Planck mass (10\textsuperscript{18}-10\textsuperscript{19} GeV) then corrections to the Higgs mass is 30 orders of magnitude above the required value.  This means that the mass without any corrections, the \textit{bare mass}, must have a mass that precisely cancels these divergences.  This is called \textit{fine-tuning} and the problem is referred to as the hierarchy problem.  \\

Too much fine-tuning is not considered to be \textit{natural}.  To maintain $\sim$1\% fine-tuning new physics must come in at the electroweak scale, so this new physics should be accessible at the LHC and is therefore right around the corner.  \\

%In order for the universe to not be extremely fine-tuned new physics must come in at the electroweak scale.  \\

These flaws motivate Beyond the SM (BSM) theories, which propose new particles and forces, and experiments to test them.  One theory that offers a possible solution to these issues with the SM is discussed in Chapter \ref{ch:motivations}.




% Flaws of SM, Neutrino masses, dark matter - lead into solution, link to next chapter - move hierarchy problem to this chapter?
	% add plot of gauge unification - show how not unified here
	
%chapter 2 - intro to SUSY and how solves these 3 problems

% Story, not textbook
% spend time on details, hierarchy problem, dark matter
% define all terms - don't use jargon without defining it










%
%Neutrinos come in three flavors that couple to each of the leptons, $\nu_{e}$, $\nu_{\mu}$, and $\nu_{\tau}$. These neutrinos do not have a definite mass but are linear combinations of other states, $\nu_{1}$, $\nu_{2}$, and $\nu_{3}$ that have masses $m_{1}$, $m_{2}$, and $m_{3}$ respectively.  In a simple 2-neutrino example, the linear combination goes as:
%
%$\nu_{\alpha} = \nu_{i}\mathrm{cos}\theta_{ij} + \nu_{j}\mathrm{sin}\theta_{ij}$
%$\nu_{\beta} = -\nu_{i}\mathrm{cos}\theta_{ij} + \nu_{j}\mathrm{sin}\theta_{ij}$
%
%where \theta_{ij} is the mixing angle 
%One failure of the 