\chapter{Conclusions}
\label{ch:conclusion}


Results from a search for stop production based
on integrated luminosity of \lumi\ data of $\rts = 13 \tev ~pp$ 
collisions recorded by the ATLAS experiment at the LHC in 2015 and
2016 are presented. Top squarks are searched for in
final states with high-\pT\ jets and large missing transverse
momentum. In this dissertation, the top squark is assumed to decay via $\stop
\rightarrow t^{(*)} \ninoone$ with large or small mass differences between the top squark and the neutralino $\Delta
m(\stop,\ninoone)$ and via $\stop\to b \chinoonepm$, where $m_{\chinoonepm}=\mLSP+1\gev$. 
Gluino-mediated \stop\ production is
studied, in which gluinos decay via $\gluino\rightarrow t\stop$, with a
small $\Delta m(\stop,\ninoone)$. \\

No significant excess above the expected SM background prediction is observed. Exclusion limits at 95\% confidence level on the combination of top-squark and LSP mass are derived resulting in the exclusion of top-squark masses in the range 450-950 GeV for $\ninoone$ masses below $160\GeV$. For the case where $\mstop\sim m_t+\mLSP$, top-squark masses between 235-590 GeV are excluded. In addition, model-independent limits and $p$-values for each signal region are reported. Limits that take into account an additional decay of $\stop\to b \chinoonepm$ are also set with an exclusion of top-squark masses between 450 and 850 GeV for $\mLSP<240\gev$ and BR($\stop\to t \LSP$)=50\% for $m_{\chinoonepm}=\mLSP+1\gev$. Limits are also derived in several pMSSM models, where one model assumes a wino-like NLSP, one assumes a higgsino-like LSP, and the other model is constrained by the dark-matter relic density.  Finally, exclusion contours are reported for gluino production where $\mstop=\mLSP+5\gev$ resulting in gluino masses being constraint to be above 1800 GeV for \stop\ masses below 800 GeV. \\

By the end of Run 2 the LHC will have delivered $\sim$150 \ifb of data.  This is nearly five times the amount of data used in this analysis, which will increase the sensitivity to higher mass stops.  There is also ongoing R\&D to improve the analysis.  Although supersymmetry has not yet been detected, there is still a lot of phase space that will be opened with more data.  Discovering supersymmetry would start a new era of particle physics, and if it is not discovered, explanations for the failings of the Standard Model will continue to be investigated.  


