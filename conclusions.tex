\chapter{Conclusions}
\label{ch:conclusion}


Results from a search for top-squark production based
on integrated luminosity of \lumi\ data of $\rts = 13 \tev ~pp$ 
collisions recorded by the ATLAS experiment at the LHC in 2015 and
2016 are presented. Top squarks are searched for in
final states with high-\pT\ jets and large missing transverse
momentum. In this paper, the top squark is assumed to decay via $\stop
\rightarrow t^{(*)} \ninoone$ with large or small mass differences between the top squark and the neutralino $\Delta
m(\stop,\ninoone)$ and via $\stop\to b \chinoonepm$, where $m_{\chinoonepm}=\mLSP+1\gev$. 
Gluino-mediated \stop\ production is
studied, in which gluinos decay via $\gluino\rightarrow t\stop$, with a
small $\Delta m(\stop,\ninoone)$. \\

No significant excess above the expected SM background prediction is observed. Exclusion limits at 95\% confidence level on the combination of top-squark and LSP mass are derived resulting in the exclusion of top-squark masses in the range
\stopLimLowLSP\ GeV for $\ninoone$ masses below $160\GeV$. For the case where $\mstop\sim m_t+\mLSP$, top-squark masses between \stopLimDiag\ GeV are excluded. In addition, model-independent limits and $p$-values for each signal region are reported. Limits that take into account an additional decay of $\stop\to b \chinoonepm$ are also set with an exclusion of top-squark masses between 450 and 850 GeV for $\mLSP<240\gev$ and BR($\stop\to t \LSP$)=50\% for $m_{\chinoonepm}=\mLSP+1\gev$. Limits are also derived in two pMSSM models, where one model assumes a wino-like NLSP and the other model is constrained by the dark-matter relic density. In addition to limits in pMSSM slices, limits are set in terms of one pMSSM-inspired simplified model where $m_{\chinoonepm}=\mLSP+5\gev$ and $m_{\ninotwo}=\mLSP+10\gev$. Finally, exlusion contours are reported for gluino production where $\mstop=\mLSP+5\gev$ resulting in gluino masses being constraint to be above 1800 GeV for \stop\ masses below 800 GeV. \\
