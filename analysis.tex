\chapter{Analysis}
\label{ch:analysis}



This chapter describes the analysis in depth, beginning with the signal regions and continuing with background estimation and finally presenting the results, interpretations, and outlook.  The lightest supersymmetric parter to the top quark being produced in direct pair production is the main focus of the analysis, while other scenarios are also discussed.  

%\section{Object Selection}

%The object definitions used follow the ATLAS recommendations and SUSY group standards.


\section{Signal Region Definitions}
\label{sec:srDefs}

Leading order stop pair-production at the LHC is dominated by gluon fusion, followed by $q\bar{q}$ scattering.  The stops are pair produced, conserving $R-$parity, and are produced directly, so there are no intermediate particles in the simplified model.  \\

As discussed in Chapter \ref{ch:analysisOverview} there are five distinct signal regions developed to optimize discovery significance for different topologies, SRA-E.  All of the searches share the following preselection requirements:

\begin{itemize}
	\item For data samples, events must be in the Good Runs List (GRL), which are runs that pass certain data quality requirements.  This gives a total luminosity of 36.46 fb\textsuperscript{-1}.
	\item For data samples, cleaning from noise bursts and incomplete events.
	\item The event must pass the lowest unprescaled \met\ trigger as well as have \met\ $>$ 250 GeV.
	\item The event must have a reconstructed primary vertex.
	\item The event must not contain any ``bad jets" from the jet definition described in section \ref{section:jetcleaning} with \pt\ $>$ 20 GeV. %``BadLooser"
	\item The event must not contain any cosmic muons as discussed in section \ref{section:jetcleaning}
	\item The event must not contain any bad muons as described in section \ref{section:muons}.
	\item The event must contain no baseline electron candidates with \pt\ $>$ 7 GeV and no baseline muons with \pt\ $>$ 6 GeV.
	\item The event must contain at least four jets.
	\item The \dphi\ between the leading two (three) jets and the \met, \dphijettwomet (\dphijetthreemet), must be greater than 0.4 for ISR based regions (non-ISR based regions). 
	\item The \mettrk\ must be greater than 30 GeV.
	\item The \dphi, between the calo \met\ and the \mettrk, \dphimettrk, must be smaller than $\pi/3$.
\item At least one b-tagged jet at the 77\% working point is required.
\end{itemize}

The analysis relies heavily on reconstruction top quark candidates, which is done using jet ``reclustering."  Jet reclustering is performed using the \antikt\ algorithm with a larger distance parameter (i.e. $R=1.2$) over the calibrated \antikt\ $R=0.4$ jet collection. The highest (second-highest) \pT\ reclustered jet is designated as the first (second) top candidate. Various optimization studies have found that this method using $R=1.2$ (top candidate) and $R=0.8$ ($W$ candidate) results in the best signal sensitivity. The mass distribution is shown in Figure \ref{fig:preselection}.  The masses are indicated by \mantikttwelvezero, \mantikttwelveone, \mantikteightzero, \mantikteightone. \\



A suite of discriminating variables based on \antikt\ $R=0.4$ jets will be considered in the analysis optimization: 
\begin{itemize}
	\item $\dphijettwomet$: The difference in $\phi$ between the jet and \met\ for the two leading jets in the event. This variable rejects events with fake \met\ from QCD, hadronic \ttbar, and detector effects.
	\item $\HT$: The scalar sum of the \pt\ of all signal \antikt\ $R=0.4$ jets ($\pt>20\gev$, $|\eta|<2.8$, after overlap removal).
	\item $\mtjetimet$: The transverse mass (\mt) between the $i$th jet and the \met\ in the event. The massless approximation is used for this and all following \mt\ variables:\newline $\mtjetimet = \sqrt{2\ptjeti\met\left(1-\cos{\Delta\phi\left(\mathrm{jet}^i,\met\right)}\right)}$, where $\ptjeti$ is the transverse momentum of the $i$th jet.
	\item $\mtbmin$: Transverse mass between closest $b$-jet to $\met$ and $\met$. This variable provides the most powerful discrimination between signal and semileptonic \ttbar\ background.
	\item $\mtbmax$: Transverse mass between furthest $b$-jet to $\met$ and $\met$. This variable provides very good discrimination between signal and semileptonic \ttbar\ background.
	\item $\drbb$: The angular separation between the two jets with the highest $b$ weights. This variable is useful in discriminating against the $Z(\nu\overline\nu)+b\overline{b}+\rm{jets}$ background. %MV2c10
\end{itemize}

A common preselection used for all five sets of signal regions is defined in Table~\ref{tab:SRcommon}.  % while the details of each signal region is given in section~\ref{sec:SignalRegionAB},~\ref{sec:SignalRegionC},~\ref{sec:SignalRegionD},~\ref{sec:SignalRegionE}.

Distributions of (a)~\mantikttwelvezero\ and (b)~\mtbmin\ are shown in Figure \ref{fig:preselection}.  

\begin{figure}[t]
  \begin{center}
    \subfloat[]{\includegraphics[width=0.5\textwidth]{figures/preselection/AntiKt12M0_preCutSRPlot_withRatio}}
    \subfloat[]{\includegraphics[width=0.5\textwidth]{figures/preselection/MtBMin_preCutSRPlot_withRatio}}
    \caption[Distributions of the discriminating variables \mantikttwelvezero\ and \mtbmin\ after common preselection.]{Distributions of the discriminating variables (a)~\mantikttwelvezero\ and (b)~\mtbmin\ after the common preselection and an additional $\mtbmin>50\gev$ requirement. The stacked histograms show the SM prediction before being normalized using scale factors derived from the simultaneous fit to all dominant backgrounds. The ``Data/SM" plots show the ratio of data events to the total SM prediction. The hatched uncertainty band around the SM prediction and in the ratio plots illustrates the combination of statistical and detector-related systematic uncertainties. The rightmost bin includes overflow events.}
    \label{fig:preselection}
  \end{center}
\end{figure}
\clearpage


\begin{table}[htbp]
  \caption[Signal region selection criteria.]{Selection criteria common to all signal regions.  Different triggers were used for different data periods in 2016.}
  \begin{center}
    \begin{tabular}{l|c} \hline\hline
      Trigger & Data 2015: \verb+HLT_xe70_mht_L1XE50+ \\ %\hline
              & Data 2016: \verb+HLT_xe90_mht_L1XE50+, \\  
              &  \verb+HLT_xe100_mht_L1XE50+,  \\ 
              &  \verb+HLT_xe110_mht_L1XE50+  \\ %\hline
              & \\ [-2.5ex] \hline
      $\met$ & $> 250\GeV$ \\ %\hline
              & \\ [-2.5ex] \hline
      $N_{\rm{lep}}$ & 0 \\ \hline
      \antikt\ $R=0.4$ jets & $\ge 4,~\pt>80,80,40,40 \gev$ \\ \hline
      $b$-tagged jets & $\ge1$ \\ \hline
      % $\dphijettwomet$ & $> \pi/5$ \\ 
      $\dphijettwomet$ or $\dphijetthreemet$ & $> 0.4$ \\ 
              & \\ [-2.5ex] \hline
      $\mettrk$  & $> 30 \gev$ \\ \hline 
      % & \\ [-2.5ex]
      $\dphimettrk$ & $<\pi/3$ \\ \hline
      % & \\ [-2.5ex] \hline
      % $\tau$ veto & yes \\ \hline
      % $\mtbmetmindphi$ & $> 175 \gev$ \\ \hline \hline
    \end{tabular}
  \end{center}
  \label{tab:SRcommon}
\end{table}

\subsection{SRA and SRB}

SRA and SRB are optimized for high stop masses.  In addition to the preselection, SRA and SRB have common requirements of $\mtbmin>200\gev$, which reduces $t\bar{t}$ background,  two b-tagged jets, and a $\tau$-veto. \\

SRA is optimized to be sensitive to decays of heavy stops into a top quark and a light \LSP. The main discriminating variables are the reclustered top masses, with $R=1.2$ and $R=0.8$, \mtbmin, \drbjetbjet, and \met. Events are divided into three categories based on the reconstructed top candidate mass ($R=1.2$ reclustered jet mass). The TT category includes events with two well-reconstructed top candidates, the TW category contains events with a well-reconstructed leading \pt\ top candidate and a well-reconstructed subleading $W$ candidate (from the subleading $R=1.2$ reclustered mass), and the T0 category represents events with only a leading top candidate. \\

The categorization showed an improvement in discovery significance, assuming the shape of the $\mstop=800\gev,\mLSP=1\gev$ benchmark, from 2$\sigma$ to 3$\sigma$, yielding comparable results to a boosted decision tree (BDT). For the benchmark point with $\mstop=1000\gev,\mLSP=1\gev$, after the SRA-B preselection, $\sim$91\% (TT=38\%, TW=22\%, and T0=31\%) of events fall into one of these three categories.  \\ %In addition to the requirements listed in Table~\ref{tab:SRcommon}, SRA and SRB have common requirements of $\mtbmin>200\gev$, two b-tagged jets, and a $\tau$-veto (in addition to the preselection cuts).   \\ %NOTE: include BDT studies?

Additionally, requirements on the stransverse mass (\mttwo)~\cite{Lester:1999tx,Barr:2003rg}
are made which are especially powerful in the T0 category where a $\chi^2$ method is applied to reconstruct top quarks with lower momenta where reclustering was suboptimal. 
The \mttwo\ variable is constructed from the direction and magnitude of the \ptmiss\ vector in the transverse plane as well as the direction of two top-quark candidates reconstructed using a $\chi^2$ method. 
The minimization in this method is done in terms of a $\chi^2$-like penalty function, $\chi^2 = (m_{\mathrm{cand}}-m_{\mathrm{true}})^2/m_{\mathrm{true}}$, where $m_{\mathrm{cand}}$ is the candidate mass and $m_{\mathrm{true}}$ is set to 80.4 GeV for \Wboson\ candidates and 173.2 GeV for top candidates. 
Initially, single or pairs of $R=0.4$ jets form \Wboson\ candidates which are then combined with additional $b$-tagged jets in the event to construct top candidates. The top candidates selected by the $\chi^2$ method are only used for the momenta in \mttwo\ while the mass hypotheses for the top quarks and the invisible particles are set to 173.2 GeV and 0 GeV, respectively. \\

A similar strategy was taken for the optimization of SRB which is aimed at being sensitive to $\mstop=600\gev,\mLSP=300\gev$. In addition to the \met\ and reclustered masses, \mtbmax, \mtbmin, and \drbjetbjet\ were used in the optimization.  The fraction of events in each category is after the SRA-B preselection: TT=14\%, TW=20\%, T0=35\%. All three categories are used in the optimization resulting in the signal regions defined in Table~\ref{tab:SignalRegionAB}. \\

\begin{table}[htb]
  \caption[Selection criteria for SRA and SRB.]{Selection criteria for SRA and SRB, in addition to the common preselection requirements as shown in Table \ref{tab:SRcommon}. The signal regions are separated into topological categories based on reconstructed top-candidate masses.}
  \begin{center}
  \def\arraystretch{1.4}
   \begin{tabular}{c||l|c|c|c} \hline\hline
      {\bf Signal Region}      &                    & {\bf TT}    & {\bf TW}     & {\bf T0}     \\ \hline \hline
                               & \mantikttwelvezero & \multicolumn{3}{c}{$>120\gev$}            \\ \cline{2-5}
                               & \mantikttwelveone  & $>120\gev$  & $[60,120]\gev$ & $<60\gev$    \\ \cline{2-5}
                               & \mtbmin            & \multicolumn{3}{c}{$>200\gev$}            \\ \cline{2-5}
                               & \nBJet    & \multicolumn{3}{c}{$\ge2$}                \\ \cline{2-5}
                               & $\tau$-veto        & \multicolumn{3}{c}{yes}                   \\ \cline{2-5} 
                               & \dphijetthreemet   & \multicolumn{3}{c}{$>0.4$}                \\ \cline{2-5}\hline \hline
      \multirow{3}{*}{{\bf A}} & \mantikteightzero  & \multicolumn{3}{c}{$>60\gev$}             \\ \cline{2-5}
                               & \drbjetbjet        & $>1$        & \multicolumn{2}{c}{-}       \\ \cline{2-5}
                               & \mttwo             & $>400$ GeV  & $>400$ GeV   & $>500$ GeV   \\ \cline{2-5}
                               & \met               & $>400 \gev$ & $> 500 \gev$ & $> 550 \gev$ \\ \hline \hline
      \multirow{2}{*}{{\bf B}} & \mtbmax            & \multicolumn{3}{c}{$>200\gev$}            \\ \cline{2-5}
                               & \drbjetbjet        & \multicolumn{3}{c}{$>1.2$}                \\ \cline{2-5}              
\hline\hline
    \end{tabular}
\end{center}
\label{tab:SignalRegionAB}
\end{table}


%Distributions of discriminating variables for SRA and SRB are shown in Figures \ref{fig:SRAMetAntiKt8M} and \ref{fig:SRBDRBBMtBMax} respectively, including the benchmark points.  Note that cutting on the variables reduces background more than it reduces the signal, thus improving sensitivity.  \\
%
%\begin{figure}[htb]
%	\begin{center}
%		\includegraphics[width=0.445\textwidth]{figures/Met_SRA_TT.eps} 
%		\includegraphics[width=0.445\textwidth]{figures/AntiKt8M_0__SRA_TT.eps} \\
%		\includegraphics[width=0.445\textwidth]{figures/Met_SRA_TW.eps} 
%		\includegraphics[width=0.445\textwidth]{figures/AntiKt8M_0__SRA_TW.eps} \\
%		\includegraphics[width=0.445\textwidth]{figures/Met_SRA_T0.eps} 
%		\includegraphics[width=0.445\textwidth]{figures/AntiKt8M_0__SRA_T0.eps}
%		\caption[Distributions of the \met\ and the \mantikteightzero\ for SRA]{\label{fig:SRAMetAntiKt8M} Distributions of the \met\ and the \mantikteightzero\ for SRA-TT, SRA-TW, and SRA-T0 after all requirements (except for the \met\ and \mantikteightzero, respectively) of Table~\ref{tab:SignalRegionAB} are made. The stacked histogram represent the total expected background estimated from MC while the hashed area represents the uncertainty due to MC statistics. Signal is shown in dashed and dotted lines for the $\mstop=600\gev,\mLSP=300\gev$ and $\mstop=1000\gev,\mLSP=1\gev$ benchmarks, respectively. }
%	\end{center}
%\end{figure}
%\clearpage
%
%\begin{figure}[!htb]
%	\begin{center}
%		\includegraphics[width=0.445\textwidth]{figures/DRBB_SRB_TT.eps}
%		\includegraphics[width=0.445\textwidth]{figures/MtBMax_SRB_TT.eps}\\
%		\includegraphics[width=0.445\textwidth]{figures/DRBB_SRB_TW.eps}
%		\includegraphics[width=0.445\textwidth]{figures/MtBMax_SRB_TW.eps}\\
%		\includegraphics[width=0.445\textwidth]{figures/DRBB_SRB_T0.eps}
%		\includegraphics[width=0.445\textwidth]{figures/MtBMax_SRB_T0.eps}\\
%		\caption[Distributions of the \drbb\ and the \mtbmax\ for SRB]{\label{fig:SRBDRBBMtBMax} Distributions of the \drbb\ for SRB-TT, SRB-TW, SRB-T0 and after all requirements except for the \drbb\ of Table~\ref{tab:SignalRegionAB} are made. The \mtbmax\ distributions are also shown with all but the \mtbmax\ requirements made.  The stacked histogram represent the total expected background estimated from MC while the hashed area represents the uncertainty due to MC statistics. Signal is shown in dashed and dotted lines for the $\mstop=600~\gev,\mLSP=300~\gev$ and $\mstop=1000~\gev,\mLSP=1~\gev$ benchmarks, respectively. }
%	\end{center}
%\end{figure}
%\clearpage


 
%To interpret the results the p-values are reported in each category along with the combined p-value.  

%SRC
\subsection{SRC}
The signature of stop decays when $\Delta m(\stop,\LSP)\sim m_{t}$ is significantly softer with low \met. This decay topology is very similar to non-resonant \ttbar\ production making signal and background separation challenging. However, several kinematic properties can be exploited to separate stop decays from \ttbar\ when an ISR jet is present in the final state. \\

An additional set of discriminating variables is defined for signal regions using ISR to gain sensitivity the compressed ($\mstop-\mLSP\sim m_t$) signal grid region. These variables are all defined in the transverse center-of-mass (CM) of the sparticle plus ISR frame. Visible objects are grouped into being either a part of the ISR or the sparticle system. This is performed using a recursive jigsaw reconstruction technique\cite{RJR_ISR}, which looks for a ``thrust axis" where the \pt-projection of all jets and \met\ in the center of mass frame in the event are maximized.  This axis then divides the space into the sparticle or ISR system.  This association with the ISR or sparticle system is indicated by an ISR or S superscript, respectively. The ``V" subscript denotes the visible part of system.  For example, \mV\ denotes the transverse mass of only the visible (jets+leptons) part of the sparticle system without the \met.  The variables considered are:

\begin{itemize}
\item  \nBJetS: number of b-tagged jets associated with the sparticle hemisphere.
\item \nJetS: number of jets associated with the sparticle hemisphere.
\item  \pTSBZero: \pt\ of the leading b-jet in the sparticle hemisphere.
\item  \pTSFour: \pt\ of the fourth jet ordered in \pt\ in the sparticle hemisphere.
\item  \dPhiISRMET: angular separation in $\phi$ of the ISR and the \met in the CM frame.
\item  \pTISR: \pt\ of the ISR system, evaluated in the CM frame.
\item  \mS: transverse mass between the whole sparticle system and \met.
\item  \mV/\mS: ratio of the transverse mass of the only the visible part of the sparticle system without \met and the whole sparticle system including \met.
\item  \rISR: Ratio between invisible system (\met in CM frame) and \pTISR
\end{itemize}

After the preselection, defined in Table~\ref{tab:SRcommon},
additional requirements are made resulting in five signal regions,
SRC-1 through SRC-5, for which the exact requirements are listed in
Table~\ref{tab:SignalRegionC}. 

\begin{table}[htpb]
  \caption[Selection criteria for SRC]{Selection criteria for SRC, in addition to the common preselection requirements as shown in Table \ref{tab:SRcommon}. The signal regions are separated into windows based on ranges of $\RISR$.}
  \begin{center}
    \def\arraystretch{1.4}
    \begin{tabular}{c||c|c|c|c|c} \hline\hline
      {\bf Variable} & SRC1 & SRC2 & SRC3 & SRC4 & SRC5 \\ \hline \hline
       \nBJet & \multicolumn{5}{c}{$\ge1$} \\ \hline
      \nBJetS & \multicolumn{5}{c}{$\ge1$} \\ \hline
      \nJetS & \multicolumn{5}{c}{$\ge5$}  \\ \hline
      \pTSBZero & \multicolumn{5}{c}{$>40\gev$}  \\ \hline
      \mS & \multicolumn{5}{c}{$>300\gev$}  \\ \hline
      \dPhiISRMET & \multicolumn{5}{c}{$>3.0$}  \\ \hline
      \pTISR & \multicolumn{5}{c}{$>400$ GeV}   \\ \hline
      \pTSFour & \multicolumn{5}{c}{$>50$ GeV}   \\ \hline
      \rISR & 0.30--0.40 & 0.40--0.50 & 0.50--0.60 & 0.60--0.70 & 0.70--0.80\\  \hline \hline
    \end{tabular}
  \end{center}
  \label{tab:SignalRegionC}
\end{table}





%SRD

\subsection{SRD}

The selections for SRD are optimized for the decay of both pair-produced top squarks into a $b$ quark and a \chinoonepm. In this case no top-quark candidates are reconstructed, so the sum of the transverse momenta of the two jets with the highest $b-$tagging weight, as well as that of the second, fourth, and fifth highest, are used for additional background rejection.  The models considered for the optimization have the chargino mass fixed to two times the neutralino mass, $m(\chinoonepm) = 2 \cdot m(\ninoone)$. \\

The best selections for the signal samples with m(\stop) = 400 GeV, m(\ninoone) = 50 GeV (SRD-low), %m(\stop) = 600 GeV, m(\ninoone) = 100 GeV (SRD-med), 
m(\stop) = 700 GeV, m(\ninoone) = 100 GeV (SRD-high) are reported in Table~\ref{tab:SRDsel}. The two regions are not combined, individual p-values are quoted for discovery while the region with the best expected sensitivity is chosen during the exclusion fit.

\begin{table}[!htb]
  \caption[Selection criteria for SRD]{Selection criteria for SRD, in addition to the common preselection requirements as shown in Table \ref{tab:SRcommon}.}
  \begin{center}
  \def\arraystretch{1.4}
  \begin{tabular}{c||c|c}
    \hline\hline
    {\bf Variable}       & {\bf SRD-low} & {\bf SRD-high} \\
    \hline \hline
    \dphijetthreemet     & \multicolumn{2}{c}{$>0.4$}     \\ \hline
    \nBJet      & \multicolumn{2}{c}{$\geq$2}    \\\hline
    \drbjetbjet     & \multicolumn{2}{c}{$>$ 0.8}    \\ \hline
    \ptbzero+\ptbone & $>300$ GeV    & $>400$ GeV     \\ \hline
    $\tau$-veto          & \multicolumn{2}{c}{yes}        \\ \hline
    \ptone\              & \multicolumn{2}{c}{$>150\GeV$} \\ \hline
    \ptthree\            & $>100\GeV$    & $>80\GeV$      \\ \hline
    \ptfour\             & \multicolumn{2}{c}{$>60\GeV$}  \\ \hline
    \mtbmin\             & $>250\GeV$    & $>350\GeV$     \\ \hline
    \mtbmax\             & $>300\GeV$    & $>450\GeV$     \\ 
    \hline\hline
  \end{tabular}
  \end{center}
  \label{tab:SRDsel}
\end{table}


\subsection{SRE}

SRE is designed for a model for which the tops are highly boosted. Such signatures can either come from direct stop pair production with a very high stop mass, or in the gluino-mediated compressed-stop scenario with large $\mgluino - \mstop$ . The benchmark for this signal region is a model where $(\mgluino, \mstop, \mLSP) = (1700, 400, 395) \gev$. Due to the large boost, the top daughters are more collimated compared to typical topology expected in \SRA.  Compared to direct stop pair production with $\mstop=800 \gev$ and $\mLSP=1 \gev$, the $\Delta R$ separation between the \Wboson\ and the bottom quark tends to be smaller. This is shown in Figure \ref{fig:SRBoost_dRWb}.  Therefore, \antikt\ $R=0.8$ reclustered jet collection will be considered as the top candidates instead of $R=1.2$ masses in other signal region. Table~\ref{tab:SRE} shows the selection criteria for SRE. 

\begin{table}[!htb]
  \caption[Selection criteria for SRE]{Selection criteria for SRE in addition to the common preselection requirements as shown in Table \ref{tab:SRcommon}.}
  \begin{center}
  \def\arraystretch{1.4}
  \begin{tabular}{c||c}
    \hline\hline
    {\bf Variable}    & {\bf SRE}          \\
    \hline \hline
    \dphijetthreemet  & $>0.4$             \\ \hline
    \nBJet   & $\geq$2            \\     \hline
    \mantikteightzero & $>120$ \gev        \\     \hline
    \mantikteightone  & $>80$ \gev         \\     \hline
    \mtbmin\          & $>200$ \gev        \\     \hline
    \met\             & $> 550 \gev$       \\     \hline
    \HT               & $>800 \gev$        \\     \hline
    \htsig            & $> 18 \sqrt{\GeV}$ \\     
\hline\hline
  \end{tabular}
  \end{center}
  \label{tab:SignalRegionE}
\end{table}

\begin{figure}[h]
	\centering
	%\begin{subfigure}[b]{0.45\textwidth}
		%\includegraphics[width=\textwidth]{figures/dRWbT800L1}
		%\caption{$m(\stop,\ninoone)=(800,1)~\GeV$}
		%\label{fig:SRBoost_dRWb_T800L1}
	%\end{subfigure}
	%~ %add desired spacing between images, e. g. ~, \quad, \qquad, \hfill etc. 
	%(or a blank line to force the subfigure onto a new line)
	%\begin{subfigure}[b]{0.45\textwidth}
		%\includegraphics[width=\textwidth]{figures/dRWbGtc5_1400_400}
		%\caption{$m(\gluino,\stop,\ninoone)=(1400,400,395)~\GeV$}
		%\label{fig:SRBoost_dRWb_Gtc_1400_400}
	%\end{subfigure}
	\includegraphics[width=\textwidth]{figures/drTruthSRE.pdf}
	\caption[$\Delta R$ between the \Wboson\ and the $b$-quark vs. the top \pt]{The true $\Delta R$ between the \Wboson\ and the $b$-quark vs.\ the truth top \pt\ for (a) SRA and (b) SRE. The common preselection criteria are applied with the exception of the $b$-jet requirement.}
	\label{fig:SRBoost_dRWb}
\end{figure}
\clearpage

\section{Background Estimation}
\label{sec:backgroundEstimation}


%SRAB





% followed by $t\bar{t}$.  In  \ttbar+V. The next three dominant backgrounds for each of the categories have about equal contributions and are, \ttbar, and W+jets. Where statistically possible separate control regions are used for the TT, TW, and T0 categories for the dominant backgrounds. \\ %For the Z+jets background normalization a set of two-lepton, two b-tagged jet control regions (described in Sect.~\ref{sec:ZllCR}) are used while for the \ttbar\ and W+jets normalization two orthogonal sets of one-lepton control regions (described in Sect.~\ref{sec:TopCRdef} and Sect.~\ref{sec:WCR}) are used. The \ttbar+V normalization is derived from a one-lepton $\ttbar+\gamma$ control region.



%Z
\subsection{\boldmath$Z+$jets}

The $Z \rightarrow \nu\nu$+jets background becomes more relevant as the \MET\ requirement is tightened. A possible way to estimate the $Z \rightarrow
\nu \nu$ background is by using a $Z \rightarrow \ell \ell$+jets control
sample. The latter channel has the advantage of an easier selection of
pure samples in terms of non-$Z$ background, but it is characterized by a
lower branching fraction than the background that it is trying to
estimate due to the  axial-vector couplings of the charged leptons to the $Z$.  This becomes particularly problematic when estimating the
background for events with large $Z$ \pT\ where the number of expected
events becomes very small. In the current analysis the number of
$Z$+jets events is reduced by the \MET\ selection, the high jet
multiplicity and the requirement for 2 $b-$tagged jets.


   
\subsubsection{Control Region}
%CR
A control region is designed for TT and TW in SRA and SRB and for T0 in SRA and SRB, as well as for SRD and SRE.  A CR for SRC was not developed since there is negligible background from $Z+$jets.  A summary of the CR selections can be found in Table~\ref{tab:selectionCRZs} and the lepton triggers used in the analysis are shown in Table \ref{tb:lepTriggers}.  \\

%INT note:
%\begin{table}[htpb]
%  \caption{Selection for the $Z$ CR with 2 b-jets.}
%  \begin{center}
%    \begin{tabular}{c|c|c|c|c}
%      \hline \hline
%      Selection                 & CRZAB-TT-TW & CRZAB-T0 & CRZD & CRZE       \\
%      \hline \hline
%      Leptons selection & \multicolumn{4}{c}{exactly 2 opposite charge electrons or muons} \\
%      %Leptons \pt & \multicolumn{4}{c}{28 GeV} \\
%      \hline
%      Jet multiplicity & \multicolumn{2}{c|}{$ \ge4 $} & $\ge5$ & $\ge4$ \\
%      \hline      
%      Jet \pT\ & \multicolumn{4}{c}{(80,80,40,40) GeV} \\
%      \hline
%      Lepton invariant mass & \multicolumn{4}{c}{$86 < M(\ell\ell) < 96$ GeV} \\
%      \hline
%      $E_{T}^{miss}$  & \multicolumn{4}{c}{$ < 50$ GeV} \\
%      \hline
%      \metprime &\multicolumn{4}{c}{$ > 100$ GeV} \\
%      \hline
%      b-jets & \multicolumn{4}{c}{$ >=2 $}\\
%      \hline
%      \mantikttwelvezero & \multicolumn{2}{c|}{$>120$ GeV} & \multicolumn{2}{c}{-} \\
%      \hline
%      \mantikttwelveone & $>60$ GeV& $<60$ GeV & \multicolumn{2}{c}{-} \\
%      \hline
%      \mtbminprime &  \multicolumn{2}{c|}{-} & 200 & 200 \\
%      \hline 
%      \mtbmaxprime &  \multicolumn{2}{c|}{-} & 200 & - \\
%      \hline 
%      \HT &  \multicolumn{3}{c|}{-} & 500  \\
%      \hline\hline
%    \end{tabular}
%  \end{center}
%  \label{tb:selectionZllCR}
%\end{table}

\begin{table}[htpb]
  \caption{Selection criteria for the $\Zjets$ control regions used to estimate the $\Zjets$ background contributions in the signal regions.} 
  \begin{center}
    \def\arraystretch{1.4}
    \begin{tabular}{c||c|c|c|c}
      \hline \hline
      Selection           & CRZAB-TT-TW                    & CRZAB-T0     & CRZD & CRZE           \\ \hline \hline
     Trigger              & \multicolumn{4}{c}{electron or muon}                                   \\ 
     \hline
     $N_{\ell}$           & \multicolumn{4}{c}{2, opposite charge, same flavour}           \\ 
     \hline
     $\pT^{\ell}$         & \multicolumn{4}{c}{$>28 \gev$}                                        \\ 
     \hline
     $m_{\ell\ell}$       & \multicolumn{4}{c}{[86,96] \gev}                                      \\ 
     \hline
     $N_{\mathrm{jet}}$   & \multicolumn{4}{c}{$\ge 4$}                                           \\
      \hline     
      \ptzero, \ptone, \pttwo, \ptthree            & \multicolumn{4}{c}{$80,80,40,40\gev$}                              \\
      \hline
      $\met$              & \multicolumn{4}{c}{$<50 \gev$}                                        \\ 
      \hline
      $\metprime$         & \multicolumn{4}{c}{$ > 100$ \gev}                                     \\
      \hline
     \nBJet     & \multicolumn{4}{c}{$\ge 2 $}                                          \\
      \hline
       \mantikttwelvezero & \multicolumn{2}{c|}{$>120\gev$} & \multicolumn{2}{c}{-}                \\ 
      \hline
       \mantikttwelveone  & $>60\gev$                      & $<60\gev$    & \multicolumn{2}{c}{-} \\ 
       \hline
       \mtbminprime       & \multicolumn{2}{c|}{-}          & \multicolumn{2}{c}{$>200\,$\gev}     \\
      \hline  
      \mtbmaxprime        & \multicolumn{2}{c|}{-}          & $>200\,$\gev & -                     \\
      \hline
      \HT                 & \multicolumn{3}{c|}{-}          & $>500\,$\gev                         \\
       \hline\hline
    \end{tabular}
  \end{center}
  \label{tab:selectionCRZs}
\end{table}

%Figure \ref{fig:CRZ} shows the distribution of some kinematic variables for the CRs.  A normalization factor is derived as the ratio between data and MC, correcting for the contamination for non-$Z$ backgrounds. %This is found
%%to be range from 1 to 1.4. \\
%
%\begin{figure}[htbp]
%  \centering
%    \includegraphics[width=0.48\textwidth]{figures/MetLep_CRZAB_TT_TW_log.eps}
%    \includegraphics[width=0.48\textwidth]{figures/MT2Chi2Lep_CRZAB_TT_TW_log.eps}
%   \includegraphics[width=0.48\textwidth]{figures/DRBB_CRZAB_T0.eps}
%   \includegraphics[width=0.48\textwidth]{figures/JetPt_4__CRZD_log.eps}
%   \includegraphics[width=0.48\textwidth]{figures/Ht_CRZE_log.eps}
%   \includegraphics[width=0.48\textwidth]{figures/AntiKt8M_0__CRZE.eps}
%    \caption{\label{fig:CRZ} Postfit distributions for the CRZs of the \metprime, \mttwo\ for SRA-TT and -TW, \drbjetbjet\ for SRA-T0, fourth leading jet \pt\ for SRD, and \htsigprime\ and \mantikteightzero for SRE. The ratio between data and MC is given in the bottom panel. The hashed area in both the top and lower panel represent the uncertainty due to MC statistics and detector systematics.}
%\end{figure}
%\clearpage
%VRs



\begin{table}[htpb]
  \caption{Lepton triggers}
  \small
  \begin{center}
    \begin{tabular}{c|c} \hline\hline
      Channel & Trigger \\  \hline
              & {\bf Data 2015} \\ \hline
      Electron & \verb+HLT_e24_lhmedium_L1EM20VH OR HLT_e60_lhmedium OR HLT_e120_lhloose+         \\ 
      Muon & \verb+HLT_mu20_iloose_L1MU15 OR HLT_mu50+ \\
      \hline
              & {\bf Data 2016} \\ \hline
      Electron & \verb+HLT_e26_lhtight_nod0_ivarloose OR HLT_e60_lhmedium_nod0+\\ 
               & \verb+OR HLT_e140_lhloose_nod0+         \\ 
      Muon & \verb+HLT_mu26_ivarmedium OR HLT_mu50+ \\
      \hline \hline
    \end{tabular}
  \end{center}
  \label{tb:lepTriggers}
\end{table}

\subsubsection{Validation Region}

Zero-lepton validation regions for \Zboson +jets dedicated to the various SRs have
been designed, except for SRC as the contribution of \Zboson +jets background to that particular SR is negligible. The various selections are summarized in Table~\ref{tab:selectionVRZs}. To avoid overlap with the signal region the \drbjetbjet\ and/or the \mantikttwelvezero/\mantikteightzero\ requirement is reversed. These reversals also help in reducing \ttbar\ and signal contamination.  The signal contamination can be seen in Appendix \ref{sec:signalContamination}.  \\  %Signal contamination is below 25\% for all validations regions with the signal causing the highest contamination having $\mstop=500\gev$ with $\mLSP=200\gev$. For VRZAB the signal contamination is 25\% for signal with $\mstop=500\gev$ with $\mLSP=200\gev$ (an already excluded signal point) and 15\% for all other signal points. Similarly, for VRZD the highest signal contamination comes from signal with $\mstop=600\gev$ with $\mLSP=200\gev$, $\mstop=500\gev$ with $\mLSP=200\gev$, and $\mstop=550\gev$ with $\mLSP=250\gev$ at 22\%, 21\%, and 17\%, respectively. The remainder of the signal points have signal contamination less than 15\%. As with VRZAB, the high signal contamination models are all points that have already been excluded. Finally, the signal models with the highest signal contamination for VRZE have $\mstop=500\gev$ with $\mLSP=200\gev$, and $\mstop=550\gev$ with $\mLSP=250\gev$ with a contamination of 19\% and 17\%, respectively. All remaining signal points have less than 15\% contamination. \\

%\begin{table}[htpb]
%  \caption{Selection for the $Z$ VRs for SRA/SRB, SRD, and SRE. The same preselection as in the SRs and shown in Table~\ref{tab:SRcommon} are applied.}
%  \begin{center}
%    \begin{tabular}{c|c|c|c}
%      \hline \hline
%      Selection                    & VRZAB                    & VRZD           & VRZE          \\
%      \hline \hline
%      Jet $\rm{p_T^0}, \rm{p_T^1}$ & $80, 80 $ GeV            & $150, 80 $ GeV & $80, 80 $ GeV \\
%\hline
%Lepton \pt                         & \multicolumn{3}{c}{$ >28\gev$}                            \\
%\hline
%Jet multiplicity                   & $\ge4$                   & $\ge5$         & $\ge4$        \\
%      \hline
%      b-jets                       & \multicolumn{3}{c}{$ >=2 $}                               \\
%      \hline
%      $\tau$ veto                  & \multicolumn{2}{c|}{yes} & no                             \\
%      \hline
%      \mtbmin                      & \multicolumn{3}{c}{$>200$ GeV}                            \\
%      \hline 
%      \mantikttwelvezero           & $ <120 $ GeV             & \multicolumn{2}{c}{-}          \\
%      \hline 
%      \drbb                        & $<1$                     & $<0.8$         & $<1$          \\
%      \hline 
%      \mtbmax                      & -                        & $>200$ GeV     & -             \\
%      \hline 
%      \HT                          & \multicolumn{2}{c|}{-}   & $>500$ GeV                     \\
%      \hline
%      \htsig                       & \multicolumn{2}{c|}{-}   & $>14$ $\sqrt{\gev}$            \\
%      \hline 
%      \mantikteightzero            & \multicolumn{2}{c|}{-}   & $ <120 $ GeV                   \\ 
%      \hline\hline
%    \end{tabular}
%  \end{center}
%  \label{tb:ZVRsel}
%\end{table}

\begin{table}[htpb]
  \caption[Selection criteria for the \Zboson\ validation regions]{Selection criteria for the \Zboson\ validation regions used
    to validate the \Zboson\ background estimates in the signal regions.} 
  \begin{center}
    \def\arraystretch{1.4}
    \begin{tabular}{c|c|c|c}
      \hline \hline
      Selection           & VRZAB                    & VRZD           & VRZE          \\
      \hline \hline
      Jet \ptzero, \ptone & $>80, >80 $ GeV            & $>150, >80 $ GeV & $>80, >80 $ GeV \\
      \hline
      $N_{\mathrm{jet}}$     & $\ge4$                   & $\ge5$         & $\ge4$        \\
      \hline
      \nBJet              & \multicolumn{3}{c}{$ \ge2 $}                              \\
      \hline
      $\tau$-veto         & \multicolumn{2}{c|}{yes} & no                             \\
      \hline
      \mtbmin             & \multicolumn{3}{c}{$>200$ GeV}                            \\
      \hline 
      \mantikttwelvezero  & $ <120 $ GeV             & \multicolumn{2}{c}{-}          \\
      \hline 
      \drbb               & $<1.0$                   & $<0.8$         & $<1.0$        \\
      \hline 
      \mtbmax             & -                        & $>200$ GeV     & -             \\
      \hline 
      \HT                 & \multicolumn{2}{c|}{-}   & $>500$ GeV                     \\
      \hline
      \htsig              & \multicolumn{2}{c|}{-}   & $>14$ $\sqrt{\gev}$            \\
      \hline 
      \mantikteightzero   & \multicolumn{2}{c|}{-}   & $ <120 $ GeV                   \\ 
      \hline\hline
    \end{tabular}
  \end{center}
  \label{tab:selectionVRZs}
\end{table}


%Figure \ref{fig:VRZ} shows the distribution of some kinematic variables for the VRs.
%
%\begin{figure}[htbp]
%  \centering
%    \includegraphics[width=0.48\textwidth]{figures/Met_VRZAB_withRatio.eps}
%     \includegraphics[width=0.48\textwidth]{figures/JetPt0_VRZAB_withRatio.eps}
%     \includegraphics[width=0.48\textwidth]{figures/NJets_VRZAB_withRatio.eps}
%   \includegraphics[width=0.48\textwidth]{figures/JetPt4_VRZD_withRatio.eps}
%    \includegraphics[width=0.48\textwidth]{figures/DRBB_VRZD_withRatio.eps}
%    \includegraphics[width=0.48\textwidth]{figures/MtBMin_VRZE_withRatio.eps}
%    \caption{\label{fig:VRZ} Postfit distributions for the VRZs of the \met, leading jet \pt\ for SRAB, and number of jets for SRAB, fourth leading jet \pt\ and \drbb\ for SRD, and \mtbmin for SRE. The ratio between data and MC is given in the bottom panel. The hashed area in both the top and lower panel represent the uncertainty due to MC statistics and detector systematics.}
%    \end{figure}
%\clearpage

%ttbar
\subsection{\boldmath$t\bar{t}$, \boldmath$W+$jets, and single-top}
\label{sec:1leptonCR}

%In this section the regions defined to estimate the \Wjets\ (CRW), \ttbar\ (CRTX), and single-top (CRST) backgrounds are introduced.\\
The \ttbar\ (CRTX), \Wjets\ (CRW), and single-top (CRST)  backgrounds contribute to the signal region selections because one lepton from the decay of a $W$ boson is out of acceptance, is mis-identified as a jet, or is an hadronically decaying $\tau$-lepton. The control regions to estimate the normalization to these backgrounds are thus defined by exploiting a one-lepton (electron or muon) selection, making them orthogonal to the SRs. For consistency with the signal regions, the same \met\ triggers are used as in the SR (Table~\ref{tab:SRcommon}). In these regions the lepton is counted as a jet for the \pt\ requirements and the jet reclustering but not for the QCD cleaning selections. The top control region is further divided to match the various signal regions. A specially designed top control region is used for SRC using similar ISR and recursive jigsaw methods. \\


The three sets of CRs (with multiple CRTs) are mutually exclusive. The requirements on the number of b-jets and on \mantikttwelvezero\ ensures that CRW is orthogonal with CRT and CRST. The selection on $\Delta R(b_{0,1},\ell)_{\mathrm{min}}$, defined as the minimum $\Delta R$ between the two jets with the highest b-tag weight and the selected lepton, ensures the orthogonality of CRT and CRST. In CRST the requirement on the $\Delta R$ of the two leading-weight b-jets is necessary to reject a large part of the remaining \ttbar\ background.\\% and reach a single top purity of $\sim$50\%.\\

%\begin{table}[htpb]
%  \caption{Summary of the selection for the 1-lepton, single top, $W$+jets and common to all top control regions. The signal lepton is treated as a jet for the jet counting and \pt\ ordering as well as for the top reco.}
%  \begin{center}
%    \begin{tabular}{c|c|c|c}
%      \hline \hline
%                                    & CRTX                        & CRST        & CRW                \\ \hline
%      Number of leptons             & \multicolumn{3}{c}{1}                                            \\ \hline
%      Number of jets (incl. lepton) & \multicolumn{3}{c}{$\geq 4$}                                     \\ \hline
%      $\pt$ of jets (incl. lepton)  & \multicolumn{3}{c}{(80,80,40,40) GeV}                            \\ \hline
%      \mindphijettwomet             & \multicolumn{3}{c}{$> 0.4$}                                      \\ \hline
%      $\met$                        & \multicolumn{3}{c}{$>250$ GeV}                                   \\ \hline
%      %\mtlepmet                    & $>30$,$<120$ GeV              & \multicolumn{2}{c}{$>30$,$<100$} \\ \hline
%      \mtlepmet                     & varies                        & \multicolumn{2}{c}{$>30$,$<100$} \\ \hline
%      Number of $b$-jets            & varies                        & $\ge2$ &$=1$                            \\ \hline
%      %Number of $b$-jets           & \multicolumn{2}{c|}{$\geq 2$} & $=1$                             \\ \hline
%      \mantikttwelvezero            & varies                        & $>120$ GeV  & $<60\,$GeV         \\ \hline
%      %\mtbmin                      & $>100\,$GeV                   & $>200\,$GeV & -                  \\ \hline
%      \mtbmin                       & varies                        & $>200\,$GeV & -                  \\ \hline
%      %\mindrblep                   & $<1.5$                        & $>1.5$      & $>2.0$             \\ \hline
%      \mindrblep                    & varies                        & \multicolumn{2}{c}{$>2.0$}             \\ \hline
%      \drbjetbjet                   & -                             & $>1.5$      & -                  \\ \hline \hline
%    \end{tabular}
%  \end{center}
%  \label{tab:1LCR_BaseDefs}
%\end{table}

%While the signal contamination in CRW and CRT is negligible (less than 10\%), in the single top control region contamination of almost 25\% is observed for signal models of direct stop pair-production with the stop decaying, with $BR=1$, to a b-quark and a chargino ($m(\chinoonepm)=2\cdot m(\ninoone)$).\\


\subsubsection{\boldmath$t\bar{t}$ Control Region}

Table \ref{selectionCRTs} show the definitions of the various top control regions. SRA and SRB each have a set of three orthogonal control regions defined by the top candidate categories. Additionally, control regions are designed for SRC, SRD and SRE.  \\ %~\ref{tab:crTopABDef} and ~\ref{tab:crTopCDEDef}, after the requirements shown in Table~\ref{tab:1LCR_BaseDefs}

The control region for SRC is designed using the same sensitive variables as the SRC definition to mimic the signal regions as close as possible while maintaining a high purity of the dominant background semi-leptonic \ttbar.  A cut of $\mtlepmet < 80\gev$ is added to remove signal contamination and a $\mindrblep<2.0$ cut is added to increase purity and ensure orthogonality to CRW. Requirements for CRTC are shown in Table \ref{tab:ttbar1LepCRISR_def}. \\


%INT note:
%\begin{table}[htb]
%  \caption{Control region definitions, in addition to the requirements presented in Table~\ref{tab:1LCR_BaseDefs} for CRTA and CRTB.}
%  \begin{center}
%    \def\arraystretch{1.4}%
%    \begin{tabular}{c||l|c|c|c} \hline\hline
%      {\bf CRT}              &                    & {\bf TT}     & {\bf TW}     & {\bf T0}     \\ \hline \hline
%                               & \mantikttwelvezero & \multicolumn{3}{c}{$>120\gev$}             \\ \cline{2-5}
%                               & \mantikttwelveone  & $>120\gev$   & $60-120\gev$ & $<60\gev$    \\ \cline{2-5}
%                               & \mtlepmet          & \multicolumn{3}{c}{$>30$,$<100$ GeV}           \\ \cline{2-5}
%                               & Number of $b$-jets & \multicolumn{3}{c}{$\ge2$}                 \\ \cline{2-5}
%                               & \mtbmin            & \multicolumn{3}{c}{$>100\,$GeV}            \\  \cline{2-5}
%                               & \mindrblep         & \multicolumn{3}{c}{$<1.5$}                 \\  \cline{2-5}
%\hline\hline
%      \multirow{3}{*}{{\bf A}} & \mantikteightzero  & \multicolumn{3}{c}{$>60\gev$}              \\ \cline{2-5}
%                               & \drbjetbjet        & $>1$         & \multicolumn{2}{c}{-}       \\ \cline{2-5}
%                               & \met               & $> 250 \gev$ & $> 300 \gev$ & $> 350 \gev$ \\ \hline \hline
%      \multirow{2}{*}{{\bf B}} & \mtbmax            & \multicolumn{3}{c}{$>200\gev$}             \\ \cline{2-5}
%                               & \drbjetbjet        & \multicolumn{3}{c}{$>1.2$}                 \\ \cline{2-5}              
%      \hline\hline
%    \end{tabular}
%  \end{center}
%  \label{tab:CRTABDef}
%\end{table}%
%
%
%\begin{table}[htpb]
%  \caption{Summary of the selection for the 1-lepton top control region for CRTC, CRTD and CRTE, in addition to the requirements presented in Table~\ref{tab:1LCR_BaseDefs}. }
%  \begin{center}
%    \begin{tabular}{l|c|c|c}
%      \hline \hline
%                           & CRTC    & CRTD     & CRTE                  \\ \hline
%      \mtlepmet            & $<80$ GeV & \multicolumn{2}{c}{$>30$,$<100$ GeV} \\ \hline
%      Number of jets   & $\ge4$    & $\ge5$ &  $\ge4$          \\ \hline
%      Number of $b$-jets   & $\ge1$    & \multicolumn{2}{c}{$\ge2$}           \\ \hline
%      \mtbmin              & -         & \multicolumn{2}{c}{$>100\,$GeV}      \\  \hline
%      \mindrblep           & $<2.0$    & \multicolumn{2}{c}{$<1.5$}           \\  \hline
%      \nJetS                 & $\ge5$    & \multicolumn{2}{c}{-}                \\ \hline
%     \nBJetS                 & $\ge1$    & \multicolumn{2}{c}{-}                \\ \hline
%      \pTSFour             & $>40\gev$ & \multicolumn{2}{c}{-}                \\ \hline
%      \pTISR               & $\ge 400$ GeV & \multicolumn{2}{c}{-}                \\ \hline \hline
%      \drbjetbjet          & -         & $>0.8$     & -                       \\ \hline
%      \mtbmax              & -         & $>100\gev$ & -                       \\ \hline
%      jet \ptone           & -         & $>150$ GeV & -                       \\ \hline
%      jet \ptthree         & -         & $>80$ GeV  & -                       \\ \hline
%      %b-jet \ptzero        & -         & $>150$ GeV & -                       \\ \hline
%      b-jet \ptzero+\ptone & -         & $>300$ GeV & -                       \\ \hline\hline
%      \mantikteightzero    & \multicolumn{2}{c|}{-}          & $>120\,$GeV             \\ \hline 
%      \mantikteightone     & \multicolumn{2}{c|}{-}          & $>80\,$GeV              \\ \hline
%      \HT                  & \multicolumn{2}{c|}{-}          & $>500\,$GeV             \\ 
%      \hline \hline
%    \end{tabular}
%  \end{center}
%  \label{tab:CRTCDEDef}
%\end{table}

\begin{landscape}
\begin{table}[htpb]
  \caption{Selection criteria for the \ttbar\ control regions used to estimate the \ttbar\ background contributions in the signal regions.} 
  \begin{center}
    \def\arraystretch{1.4}
\scriptsize
    \begin{tabular}{c||c|c|c|c|c|c|c|c|c}
      \hline \hline
      Selection  & CRTA-TT & CRTA-TW & CRTA-T0 & CRTB-TT & CRTB-TW & CRTB-T0 & CRTC & CRTD & CRTE \\ \hline \hline
      Trigger    & \multicolumn{9}{c}{\met}                                                                         \\ 
      \hline
      $N_{\ell}$ & \multicolumn{9}{c}{1}                                                                            \\ 
     \hline
     $\pT^{\ell}$ & \multicolumn{9}{c}{$>20 \gev$}   \\ 
     \hline
     $N_{\mathrm{jet}}$ & \multicolumn{9}{c}{$\ge 4$ (including
       electron or muon)} \\
      \hline     
      \ptzero, \ptone, \pttwo, \ptthree & \multicolumn{9}{c}{$80,80,40,40\gev$} \\
      \hline
      \nBJet & \multicolumn{9}{c}{$ \ge 2 $}\\
\hline   
      $\dphijettwomet$ & \multicolumn{9}{c}{$>0.4$} \\ 
      \hline
      $\dphijetthreemet$ & \multicolumn{6}{c|}{$>0.4$} & - & \multicolumn{2}{c}{$>0.4$}\\ 
      \hline
      $\mT(\ell,\met)$   &     \multicolumn{6}{c|}{$[30,100]\gev$} & $<100\gev$ & \multicolumn{2}{c}{$[30,100]\gev$}  \\ 
  \hline
       \mtbmin              & \multicolumn{6}{c|}{$>100\,$\gev} & - &  \multicolumn{2}{c}{$>100\,$\gev} \\ 
       \hline
       $\drblmin$ & \multicolumn{6}{c|}{$<1.5$} & $<2.0$ &  \multicolumn{2}{c}{$<1.5$}\\
      \hline
       \mantikttwelvezero      & \multicolumn{6}{c|}{$>120 \gev$} & \multicolumn{3}{c}{-} \\ 
      \hline
       \mantikttwelveone      & $>120 \gev$  & $[60, 120] \gev$  & $<60 \gev$  & $>120 \gev$  & $[60, 120] \gev$  & $<60 \gev$ & \multicolumn{3}{c}{-} \\ 
\hline
 \mantikteightzero      & \multicolumn{3}{c|}{$>60 \gev$} &  \multicolumn{5}{c|}{-} & $>120 \gev$ \\ 
      \hline
       \mantikteightone  &    \multicolumn{8}{c|}{-}  & $>80 \gev$  \\ 
\hline
 $\met$ & $ >250$ \gev & $ >300$ \gev & $ >350$ \gev & \multicolumn{6}{c}{$>250 \gev$} \\  
       \hline
       $\drbjetbjet$                       & $>1.0$ & \multicolumn{2}{c|}{-}& \multicolumn{3}{c|}{$>1.2$} & - & $>0.8$ & - \\
       \hline
       $\mtbmax$                        & \multicolumn{3}{c|}{-}& \multicolumn{3}{c|}{$>200\gev$} & - & $>100\gev$ & - \\
       \hline
       $\ptone$                        & \multicolumn{7}{c|}{-} & $>150\gev$ & - \\
       \hline
       $\ptthree$                        & \multicolumn{7}{c|}{-} & $>80\gev$ & - \\
       \hline
       $\ptbzero+\ptbone$                        & \multicolumn{7}{c|}{-} & $>300\gev$ & - \\
       \hline
       $\NjV$         & \multicolumn{6}{c|}{-}& $ \ge 5$ & \multicolumn{2}{c}{-} \\
      \hline
       $\NbV$         & \multicolumn{6}{c|}{-}& $ \ge 1$ & \multicolumn{2}{c}{-} \\
      \hline
       $\PTISR$         & \multicolumn{6}{c|}{-}& $>400\gev$ & \multicolumn{2}{c}{-} \\
\hline
       $\pTSFour$         & \multicolumn{6}{c|}{-}& $>40\gev$ & \multicolumn{2}{c}{-} \\
\hline
       $\HT$         & \multicolumn{8}{c|}{-}& $>500\gev$ \\

       \hline\hline
    \end{tabular}
  \end{center}
  \label{tab:selectionCRTs}
\end{table}
\end{landscape}

Figure \ref{fig:CRT} shows the distributions of some of the discriminating variables in the top control regions. \\

%\begin{figure}[htbp]
%  \begin{center}
%    \includegraphics[width=0.45\textwidth]{figures/Met_CRTopATT_log.eps}%/Met_CRTATT_log} 
%    \includegraphics[width=0.45\textwidth]{figures/MT2Chi2_CRTopATT_log.eps}%MT2Chi2_CRTATT_log} 
%  \includegraphics[width=0.45\textwidth]{figures/MtBMin_CRTopATW_log.eps}%MtBMin_CRTATW_log} 
%  \includegraphics[width=0.45\textwidth]{figures/MtBMax_CRTBTT_log} 
%   % \includegraphics[width=0.45\textwidth]{figures/ttbar/postfit/MtBMax_CRTBT0_log} 
%    \includegraphics[width=0.45\textwidth]{figures/JetPt4_CRTD_log} 
%    \includegraphics[width=0.45\textwidth]{figures/HtSig_CRTE} 
% \end{center}
%  \caption{CRT postfit distributions for 36.07 \ifb\ of data. The \met\ and \mttwo\ for CRTATT, \mtbmin\ for CRTATW, \mtbmax\ for CRTBTT, fourth leading jet \pt\ for CRTD, and \htsig\ for CRTE postfit distributions are shown. The ratio between data and MC is shown in the bottom panel. The hashed area in both the top and lower panel represent the uncertainty due to MC statistics and detector systematic uncertainties.}
%  \label{fig:CRT}
%\end{figure}
\begin{figure}[htbp]
  \begin{center}
    \includegraphics[width=0.45\textwidth]{figures/ttbar/postfit/Met_CRTopATT_log} 
    \includegraphics[width=0.45\textwidth]{figures/ttbar/postfit/MtBMin_CRTopATW_log} 
     \includegraphics[width=0.45\textwidth]{figures/ttbar/postfit/MT2Chi2_CRTopAT0_log}    
      \includegraphics[width=0.45\textwidth]{figures/ttbar/postfit/MtBMax_CRTopBTT_log} 
    \includegraphics[width=0.45\textwidth]{figures/ttbar/postfit/Ht_CRTopE_log} 
    \includegraphics[width=0.45\textwidth]{figures/ttbar/postfit/JetPt_4__CRTopD_log} 
  \end{center}
  \caption{Postfit distributions for \intlumi\ \ifb\ of data for various discriminating variables in CRTA, CRTB, CRTD, and CRTE.  The ratio between data and MC is shown in the bottom panel. The hashed area in both the top and lower panel represent the uncertainty due to MC statistics and detector systematic uncertainties.}
  \label{fig:CRT}
\end{figure}
\clearpage



\begin{table}[htpb]
  \caption{One-lepton \ttbar+ISR control region definitions. The same \met\ triggers as mentions in Table~\ref{tab:SRcommon} are used. }
  \begin{center}
    \def\arraystretch{1.4}%
    \begin{tabular}{c|c} \hline\hline
      {\bf Variable}     & 1L 1b \ttbar+ISR CR \\ \hline \hline
      Number of leptons  & 1                   \\
      Number of $b$-jets & $\ge1$              \\
      \mtlepmet          & $<80\gev$           \\ 
      \mindrblep         & $<2.0$              \\ 
      \NjV               & $\ge5$              \\
      \NbV               & $\ge1$              \\
      \pTSFour           & $>40\gev$           \\
      \PTISR             & $\ge 400$           \\ \hline \hline
    \end{tabular}
  \end{center}
  \label{tab:ttbar1LepCRISR_def}
\end{table}%

%The distributions of the SRC variables in the \ttbar+jets\ one-lepton control regions are shown in
%Fig.~\ref{fig:ttbar1LepCRISR_1b}. \\

%\begin{figure}[htbp]
%  \centering
%   \includegraphics[width=0.45\textwidth]{figures/CA_dphiISRI_CRTC_withRatio.eps}
%   \includegraphics[width=0.45\textwidth]{figures/CA_PTISR_CRTC_withRatio.eps}
%  \caption{\label{fig:ttbar1LepCRISR_1b}{Distribution of ISR signal region
%      \dphiISR\ and \pTISR\ variables selections in the one-lepton 1 b-jet \ttbar\ control
%      region. Normalizations are pre-fit and taken from Monte Carlo expectations.}}
%\end{figure}
%\clearpage

%\begin{table}[htpb]
%  \caption{Background composition of $\ttbar$ control regions for SRA normalized to 36.07 \ifb. The signal lepton is treated as a jet. }
%  \begin{center}
%    \input{CRTATTYields.tex}
%    \input{CRTATWYields.tex}
%    \input{CRTAT0Yields.tex}
%  \end{center}
%  \label{tab:CRTAYields}
%\end{table}

%
%\begin{table}[htpb]
%  \caption{Background composition of $\ttbar$ control regions for SRB normalized to 36.07 \ifb. The signal lepton is treated as a jet. }
%  \begin{center}
%    \input{CRTBTTYields.tex}
%    \input{CRTBTWYields.tex}
%    \input{CRTBT0Yields.tex}
%  \end{center}
%  \label{tab:CRTBYields}
%\end{table}


%%\begin{table}[htpb]
%%  \caption{Background composition of $\ttbar$ control regions for SRD and SRE normalized to 36.07 \ifb. The signal lepton is treated as a jet. }
%%  \begin{center}
%%    \input{CRTCYields.tex}
%%    \input{CRTDYields.tex}
%%    \input{CRTEYields.tex}
%%  \end{center}
%%  \label{tab:CRTCDEYields}
%%\end{table}
%
\subsubsection{\boldmath$t\bar{t}$ Validation Region}


The selections for the $\ttbar$ validation regions in the 0-lepton, two b-jets channel for the $t\bar{t}$ background are summarized in Tables~\ref{tab:vrTopABDef} and ~\ref{tab:vrTopCDEDef}. The same preselection as discussed in Table~\ref{tab:SRcommon} are used.

\begin{table}[htb]
  \caption[VRTopA and VRTopB region definitions.]{Validation region definitions, in addition to the requirements presented in Table~\ref{tab:SRcommon} for VRTopA and VRTopB.}
  \begin{center}
    \def\arraystretch{1.4}%
    \begin{tabular}{c||l|c|c|c} \hline\hline
      {\bf VRTop}                &                    & {\bf TT}     & {\bf TW}     & {\bf T0}     \\ \hline \hline
                                 & \mantikttwelvezero & \multicolumn{3}{c}{$>120\gev$}             \\ \cline{2-5}
                                 & \mantikttwelveone  & $>120\gev$   & $60-120\gev$ & $<60\gev$    \\ \cline{2-5}
                                 & \mtbmin            & $>100,<200$ GeV & $>140,<200$ GeV & $>160,<200$ GeV       \\ \cline{2-5}
                                 & Number of $b$-jets & \multicolumn{3}{c}{ $\geq 2$  }            \\ \cline{2-5}
      \hline\hline
      \multirow{3}{*}{{\bf A}}   & \mantikteightzero  & \multicolumn{3}{c}{$>60\gev$}              \\ \cline{2-5}
      & \drbjetbjet        & $>1$         & \multicolumn{2}{c}{-}       \\ \cline{2-5}
                                 & \met               & $> 300 \gev$ & $> 400 \gev$ & $> 450 \gev$ \\ \hline \hline
      \multirow{2}{*}{{\bf B}}   & \drbjetbjet           & \multicolumn{3}{c}{$>1.2$}             \\ \cline{2-5}
       & \mtbmax           & \multicolumn{3}{c}{$>200$ GeV}             \\ \cline{2-5}
      \hline\hline
    \end{tabular}
  \end{center}
  \label{tab:vrTopABDef}
\end{table}%


\begin{table}[htpb]
  \caption[VRTopC, VRTopD and VRTopE Selection criteria.]{Summary of the selection for the 0-lepton top validation region for VRTopC, VRTopD and VRTopE, in addition to the requirements presented in Table~\ref{tab:SRcommon}. }
  \begin{center}
    \begin{tabular}{l|c|c|c}
      \hline \hline
                           & VRTopC                & VRTopD     & VRTopE                   \\ \hline
      \mtbmin              & -                     & \multicolumn{2}{c}{ $>100,<200$ GeV  } \\ \hline
      Number of jets   & $\ge4$                & $\ge 5$ & $\ge4$       \\ \hline \hline
      Number of $b$-jets   & $\ge1$                & \multicolumn{2}{c}{ $\geq 2$  }       \\ \hline \hline
      \nJetS                 & $\ge4$                & \multicolumn{2}{c}{-}                 \\ \hline
      \nBJetS                 & $\ge1$                & \multicolumn{2}{c}{-}                 \\ \hline
      \pTSBZero                 & $\ge 40$ GeV          & \multicolumn{2}{c}{-}                 \\ \hline
      \pTSFour             & $>40\gev$ & \multicolumn{2}{c}{-}                \\ \hline
      \pTISR               & $\ge 400$ GeV         & \multicolumn{2}{c}{-}                 \\ \hline
      \mS                  & $>100\gev$            & \multicolumn{2}{c}{-}                 \\ \hline
      $\mV/\mS$            & $<0.6$                & \multicolumn{2}{c}{-}                 \\ \hline
      \dPhiISRMET            & $<3.00$               & \multicolumn{2}{c}{-}                 \\ \hline\hline
      %\RISR                & $\ge0.45$             & \multicolumn{2}{c}{-}                 \\ \hline \hline
      \drbjetbjet          & -                     & $>0.8$     & -                        \\ \hline
      \mtbmax              & -                     & $>300\gev$ & -                        \\ \hline
      jet \ptone           & -                     & $>150$ GeV & -                        \\ \hline
      jet \ptthree         & -                     & $>80$ GeV & -                        \\ \hline
      %b-jet \ptzero        & -                     & $>150$ GeV & -                        \\ \hline
      b-jet \ptzero+\ptone & -                     & $>300$ GeV & -                        \\ \hline
      Jet multiplicity          & $\ge4$                    & $\ge5$     & $\ge4$                        \\ \hline
      
      $\tau$-veto          & -                     & yes        & -                        \\ \hline \hline
      \mantikteightzero    & \multicolumn{2}{c|}{-} & $>120\,$GeV                           \\ \hline 
      \mantikteightone     & \multicolumn{2}{c|}{-} & $>80\,$GeV                            \\ \hline
      \hline
    \end{tabular}
  \end{center}
  \label{tab:vrTopCDEDef}
\end{table}


%\begin{figure}[htbp]
%  \begin{center}
%    \includegraphics[width=0.45\textwidth]{figures/Met_VRTopATT_log.eps} 
%    \includegraphics[width=0.45\textwidth]{figures/MT2Chi2_VRTopATT_log.eps} 
%  \includegraphics[width=0.45\textwidth]{figures/MtBMin_VRTopATW_log.eps} 
%  \includegraphics[width=0.45\textwidth]{figures/MtBMax_VRTopBTT_log.eps} 
%   % \includegraphics[width=0.45\textwidth]{figures/ttbar/postfit/MtBMax_VRTopBT0_log} 
%    \includegraphics[width=0.45\textwidth]{figures/JetPt_4__VRTopD_log.eps} 
%    \includegraphics[width=0.45\textwidth]{figures/HtSig_VRTopE.eps} 
% \end{center}
%  \caption{VRTop postfit distributions for 36.07 \ifb\ of data. The \met\ and \mttwo\ for VRTopATT, \mtbmin\ for VRTopATW, \mtbmax\ for VRTopBTT, fourth leading jet \pt\ for VRTopD, and \htsig\ for VRTopE postfit distributions are shown. The ratio between data and MC is shown in the bottom panel. The hashed area in both the top and lower panel represent the uncertainty due to MC statistics and detector systematic uncertainties.}
%  \label{fig:VRTop}
%\end{figure}
%\clearpage

%The yields in the $\ttbar$ validation regions are summarised in Table~\ref{tab:VRTopA_yields},~\ref{tab:VRTopB_yields}, and~\ref{tab:VRTopCDE_yields}. The $\ttbar$ purity in this region is $\sim$40\%.\\
%A selection of postfit distributions for the data-MC comparison in the \ttbar\
%VR is shown in Fig.~\ref{fig:VRTopATT}, \ref{fig:VRTopATW},
%\ref{fig:VRTopAT0}, \ref{fig:VRTopBTT}, \ref{fig:VRTopBTW}, \ref{fig:VRTopBT0}, \ref{fig:VRTopC}, \ref{fig:VRTopD}, \ref{fig:VRTopE}.

%\begin{table}[!htb]
%  \centering
%  \input{VRTopATTYields.tex}
%  \input{VRTopATWYields.tex}
%  \input{VRTopAT0Yields.tex}
%  \caption{Yields in the $\ttbar$ validation regions for SRA in 36.07 \ifb\ of data. The Data/MC SF is the simple ratio of Data/Total MC without any correlations. }
%  % The highest signal contamination comes from ($\mstop$, $\mLSP$) = (250, 77) and amounts to 8\% in VRttbar1 and 10\% in VRttbar2.}
%  \label{tab:VRTopA_yields}
%\end{table}
%
%\begin{table}[!htb]
%  \centering
%  \input{VRTopBTTYields.tex}
%  \input{VRTopBTWYields.tex}
%  \input{VRTopBT0Yields.tex}
%  \caption{Yields in the $\ttbar$ validation regions for SRB in 36.07 \ifb\ of data. The Data/MC SF is the simple ratio of Data/Total MC without any correlations. }
%  % The highest signal contamination comes from ($\mstop$, $\mLSP$) = (250, 77) and amounts to 8\% in VRttbar1 and 10\% in VRttbar2.}
%  \label{tab:VRTopB_yields}
%\end{table}
%
%
%\begin{table}[!htb]
%  \centering
%  \input{VRTopCYields.tex}
%  \input{VRTopDYields.tex}
%  \input{VRTopEYields.tex}
%  \caption{Yields in the $\ttbar$ validation regions for SRC, SRD, and SRE in 36.07 \ifb\ of data. The Data/MC SF is the simple ratio of Data/Total MC without any correlations. }
%  % The highest signal contamination comes from ($\mstop$, $\mLSP$) = (250, 77) and amounts to 8\% in VRttbar1 and 10\% in VRttbar2.}
%  \label{tab:VRTopCDE_yields}
%\end{table}







%SRD
% dominant backgrounds (not too jargony), CRs, VRs, 

\subsubsection{\boldmath$W+$jets CR}

The selection for CRW is shown in table \ref{tab:1LCR_BaseDefs}.  Data/MC comparisons in the \Wjets\ control region are shown in
Fig.~\ref{fig:CRW}.  The MC is normalized to 36.07 \ifb,
and no normalization factors are applied to any of the SM
components. Several variables for which there is an extrapolation from
CRW to the various SRs are shown. \\ %For SRA the extrapolation is in
%\met, \drbjetbjet, \mantikttwelvezero,  \mantikttwelveone, and
%\mantikteightzero\ while for SRB the extrapolation is in \mtbmin\,
%\mtbmax, and \drbjetbjet. The leading jet \pt s and the leading
%b-tagged jet \pt\ are shown because for these variables there is an extrapolation from CRW to SRD while the \met, \mtbmin, \HT, \htsig, \mantikteightzero, and \mantikteightone\ are shown due to the extrapolation to SRE. The signal contamination is less than 10\% for all signal points with the largest contamination coming from the pure bChino decay of stops with mass 500 GeV and LSP mass of 50 GeV where the chargino mass is assumed to be 100 GeV. No particular trends are observed in the data-MC ratios in any of the distribution. Within statistical uncertainty the data are compatible with the MC SM expectation. \\
%\begin{figure}[!htb]
%  \centering
% \includegraphics[width=0.45\textwidth]{figures/Met_CRW_log.eps}
% \includegraphics[width=0.45\textwidth]{figures/HtSig_CRW.eps}
% \includegraphics[width=0.45\textwidth]{figures/MtBMin_CRW.eps}
% \includegraphics[width=0.45\textwidth]{figures/AntiKt12M_0__CRW.eps}
%  \caption{Postfit data/MC comparisons in the CRW for \met , \htsig , \mtbmin , and \mantikttwelveone .  The background contributions from MC are normalized to 36.07 \ifb, and summed together, the data points are shown in black. The hatched band in the ratio shows the MC statistical and detector uncertainties.}
%\label{fig:CRW}
%\end{figure}
%\clearpage

%\begin{table}[!htb]
%  \centering
%  \input{CRWYields.tex}
%  \caption{Yields in the CRW in 36.07 \ifb\ of data.  }
%  \label{tab:CRW_yields}
%\end{table}
%
%\begin{figure}[!htb]
%  \centering
%  \includegraphics[width=0.45\textwidth]{figures/wJets/postfit/JetPt_1__CRW_log.eps}
%  \includegraphics[width=0.45\textwidth]{figures/wJets/postfit/JetPt_3__CRW_log.eps}
%  \includegraphics[width=0.45\textwidth]{figures/wJets/postfit/JetPt_4__CRW_log.eps}
%  \includegraphics[width=0.45\textwidth]{figures/wJets/postfit/NJets_CRW_log.eps}
%  %\includegraphics[width=0.45\textwidth]{figures/wJets/postfit/JetPt_JetLeadTagIndex_JetPt_JetSubleadTagIndex__CRW_log.eps}
%  \includegraphics[width=0.45\textwidth]{figures/wJets/postfit/Ht_CRW_log.eps}
%  \includegraphics[width=0.45\textwidth]{figures/wJets/postfit/MinDRBLep_CRW.eps}
%  \includegraphics[width=0.45\textwidth]{figures/wJets/postfit/DRBB_CRW.eps}
%  \caption{Postfit data/MC comparisons in the CRW. From left to right and top to bottom, the variables shown are the leading four jet \pt, the leading b-tagged jet \pt, \HT, \htsig, \mindrblep, and \drbjetbjet. The background contributions from MC are normalized to 36.07 \ifb, and summed together, the data points are shown in black. The hatched band in the ratio shows the MC statistical and detector uncertainties.}
%  \label{fig:CRWpts}
%\end{figure}
%
%\begin{figure}[!htb]
%  \centering
%  \includegraphics[width=0.45\textwidth]{figures/wJets/postfit/Met_CRW_log.eps}
%  \includegraphics[width=0.45\textwidth]{figures/wJets/postfit/MT2Chi2_CRW_log.eps}
%  \includegraphics[width=0.45\textwidth]{figures/wJets/postfit/HtSig_CRW.eps}
%  \includegraphics[width=0.45\textwidth]{figures/wJets/postfit/MtBMin_CRW.eps}
%  \includegraphics[width=0.45\textwidth]{figures/wJets/postfit/MtBMax_CRW.eps}
%  \caption{Postfit data/MC comparisons in the CRW. From left to right and top to bottom, the variables shown are \met, \mtbmin, \mtbmax, \mantikttwelvezero, \mantikttwelveone, and \mantikteightzero. The background contributions from MC are normalized to 36.07 \ifb, and summed together, the data points are shown in black. The hatched band in the ratio shows the MC statistical and detector uncertainties.}
%  \label{fig:CRW}
%\end{figure}
%
%\begin{figure}[!htb]
%  \centering
%  \includegraphics[width=0.45\textwidth]{figures/wJets/postfit/AntiKt12M_0__CRW.eps}
%  \includegraphics[width=0.45\textwidth]{figures/wJets/postfit/AntiKt12M_1__CRW.eps}
%  \includegraphics[width=0.45\textwidth]{figures/wJets/postfit/AntiKt8M_0__CRW.eps}
%  \includegraphics[width=0.45\textwidth]{figures/wJets/postfit/AntiKt8M_1__CRW.eps}
%  \caption{Postfit data/MC comparisons in the CRW. From left to right and top to bottom, the variables shown are \met, \mtbmin, \mtbmax, \mantikttwelvezero, \mantikttwelveone, and \mantikteightzero. The background contributions from MC are normalized to 36.07 \ifb, and summed together, the data points are shown in black. The hatched band in the ratio shows the MC statistical and detecotor uncertainties.}
%  \label{fig:CRWMasses}
%\end{figure}


\subsubsection{\boldmath$W+$jets Validation Region}
\label{section:VRW}

The selections for a possible \Wjets\ validation region in the 1-lepton, two b-jets channel (VRW) are summarised in Tab.~\ref{tab:VRW}.

\begin{table}[htpb]
  \caption[Summary of the selection for the 1-lepton $W$+jets validation region.]{Summary of the selection for the 1-lepton $W$+jets validation region. The signal lepton is treated as a jet. The same \met\ triggers as mentioned in Table~\ref{tab:SRcommon} are used.}
  \begin{center}
    \begin{tabular}{c|c}
      \hline \hline
                                          & VRW              \\ \hline
      Number of leptons                   & 1                \\ \hline
      Number of jets (incl. lepton)       & $\geq 4$         \\ \hline
      $\pt$ of jets (incl. lepton) in GeV & (80,80,40,40)    \\ \hline
      Number of $b$-jets                  & $\geq 2$         \\ \hline
      \mindphijettwomet                   & $>0.4$           \\ \hline
      $\met$                              & $>250$ GeV       \\ \hline
      \mtlepmet                           & $>30$,$<100$ GeV \\ \hline
      \mantikttwelvezero                  & $<70\,$GeV       \\ \hline
      \mtbmin                             & 150 GeV          \\ \hline
      \mindrblep                          & $>1.8$           \\ \hline \hline
    \end{tabular}
  \end{center}
  \label{tab:VRW}
\end{table}

%The yields in the VRW region are summarized in Tab.~\ref{tab:VRW_yields}. The \Wjets\ purity in this region is 30\% while the signal contamination is less than 15\% for all signal benchmark points except for ($\mstop$, $m_{\chinoonepm}$, $\mLSP$) = (550, 100, 50) GeV which has been excluded in previous searches. 

%\begin{table}[!htb]
%  \centering
%  \input{VRWYields.tex}
%  \caption{Yields in the VRW in 36.07 \ifb\ of data. The uncertainty on the SF, which is compatible within statistical uncertainties with the SF from the CR, should come down significantly. }
%  \label{tab:VRW_yields}
%\end{table}

A selection of distributions where data is compared to MC (no normalization factors applied) is shown in Fig.~\ref{fig:VRW}. 

%\begin{figure}[!htb]
%  \centering
%  \includegraphics[width=0.45\textwidth]{figures/wJets/postfit/JetPt_1__VRW_log.eps}
%  \includegraphics[width=0.45\textwidth]{figures/wJets/postfit/JetPt_3__VRW_log.eps}
%  \includegraphics[width=0.45\textwidth]{figures/wJets/postfit/JetPt_4__VRW_log.eps}
%  \includegraphics[width=0.45\textwidth]{figures/wJets/postfit/NJets_VRW_log.eps}
%  \includegraphics[width=0.45\textwidth]{figures/wJets/postfit/JetPt_JetLeadTagIndex_JetPt_JetSubleadTagIndex__VRW_log.eps}
%  \includegraphics[width=0.45\textwidth]{figures/wJets/postfit/Ht_VRW_log.eps}
%  \includegraphics[width=0.45\textwidth]{figures/wJets/postfit/MinDRBLep_VRW.eps}
%  \includegraphics[width=0.45\textwidth]{figures/wJets/postfit/DRBB_VRW.eps}
%  \caption{Postfit data/MC comparisons in the VRW. From left to right and top to bottom, the variables shown are the leading four jet \pt, the leading b-tagged jet \pt, \HT, \htsig, \mindrblep, and \drbjetbjet. The background contributions from MC are normalized to 36.07 \ifb, and summed together, the data points are shown in black. The hatched band in the ratio shows the MC statistical and detector uncertainties.}
%  \label{fig:VRWpts}
%\end{figure}

%\begin{figure}[!htb]
%  \centering
%  \includegraphics[width=0.45\textwidth]{figures/Met_VRW_withRatio.eps}
%  \includegraphics[width=0.45\textwidth]{figures/MT2Chi2_VRW_withRatio_log.eps}
%  \includegraphics[width=0.45\textwidth]{figures/HtSig_VRW_withRatio.eps} %figures/postfit/Ht...
%  \includegraphics[width=0.45\textwidth]{figures/MtBMin_VRW_withRatio.eps}
%  \includegraphics[width=0.45\textwidth]{figures/MtBMax_VRW_withRatio.eps}
%  \caption{Postfit data/MC comparisons in the VRW. From left to right and top to bottom, the variables shown are \met, \mtbmin, \mtbmax, \mantikttwelvezero, \mantikttwelveone, and \mantikteightzero. The background contributions from MC are normalized to 36.07 \ifb, and summed together, the data points are shown in black. The hatched band in the ratio shows the MC statistical and detector uncertainties.}
%  \label{fig:VRW}
%\end{figure}
%\clearpage


%\begin{figure}[!htb]
%  \centering
%  \includegraphics[width=0.45\textwidth]{figures/wJets/postfit/AntiKt12M_0__VRW.eps}
%  \includegraphics[width=0.45\textwidth]{figures/wJets/postfit/AntiKt12M_1__VRW.eps}
%  \includegraphics[width=0.45\textwidth]{figures/wJets/postfit/AntiKt8M_0__VRW.eps}
%  \includegraphics[width=0.45\textwidth]{figures/wJets/postfit/AntiKt8M_1__VRW.eps}
%  \caption{Postfit data/MC comparisons in the VRW. From left to right and top to bottom, the variables shown are \met, \mtbmin, \mtbmax, \mantikttwelvezero, \mantikttwelveone, and \mantikteightzero. The background contributions from MC are normalized to 36.07 \ifb, and summed together, the data points are shown in black. The hatched band in the ratio shows the MC statistical and detecotor uncertainties.}
%  \label{fig:VRWMasses}
%\end{figure}


%The relative \Wjets\ composition split in contributions from the CVetoBVeto, CFilterBVeto and BFilter samples in the \Wjets\ control region is 19.6\%, 36.6\%, and 43.7\% respectively. The same fractions in the WVR are 2.1\%, 7.7\%, and 90.2\%. As an example in SRB-T0 the \Wjets\ relative composition is 0.1\% CVetoBVeto, 11.8\% CFilterBVeto, 88.1\% BFilter - the numbers of CVetoBVeto and CFilterBVeto events in the SR are affected by very large uncertainty due to MC statistics. This shows that even though the composition in the CR does not reflect completely the SR composition, the VR is designed to have very similar type of events as the SR and being hence a very good test of the goodness of the Fit in the CR.
%
%The $W+c$ contribution in the CR amounts to 9.73\%, in the VR to 5.15\%, and in SRC-high to 1.86\% (number based on 20.1, Sherpa 2.1 samples).
%
%\begin{figure}[htpb]
%  \begin{center}
%    \subfloat[]{\includegraphics[width=0.44\textwidth]{figures/Znunu/MT2Chi2Lep_CRZAB_T0_log.eps}}%
%    \subfloat[]{\includegraphics[width=0.44\textwidth]{figures/Znunu/MetLep_CRZE_log.eps}}\\
%    \subfloat[]{\includegraphics[width=0.44\textwidth]{figures/ttbar/CA_RISR_CRTC.eps}}
%    \subfloat[]{\includegraphics[width=0.44\textwidth]{figures/Wjets/MtBMax_CRW_log.eps}}\\
%     \subfloat[]{\includegraphics[width=0.44\textwidth]{figures/singleTop/JetPt_1__CRST_log.eps}}
%   \subfloat[]{\includegraphics[width=0.44\textwidth]{figures/ttGamma/SigPhotonPt_0__CRTTGamma_log.eps}}%
%
%    %\subfloat[]{\includegraphics[width=0.455\textwidth]{figures/Znunu/MT2Chi2Lep_CRZAB_TT_TW_log.eps}}\\
%    %\subfloat[]{\includegraphics[width=0.455\textwidth]{figures/Znunu/AntiKt12M_1__CRZAB_TT_TW.eps}}%
%    %\subfloat[]{\includegraphics[width=0.455\textwidth]{figures/Znunu/MtBMaxLep_CRZAB_TT_TW_log.eps}}\\
%    %\subfloat[]{\includegraphics[width=0.455\textwidth]{figures/Znunu/MT2Chi2Lep_CRZAB_T0_log.eps}}
%    %\subfloat[]{\includegraphics[width=0.455\textwidth]{figures/Znunu/MetLep_CRZAB_T0_log.eps}}
%    \caption{(a)~\mttwoprime\ distribution in CRZAB-T0, (b)~\metprime\ in CRZE
%      , (c)~the \rISR\ distribution in CRTC, 
%      (d)~the \mtbmax\ distribution in CRW, (e)~the transverse momentum of the second-leading-\pT\ jet in CRST,  and
%      (f)~the photon \pT\ distribution in CRTTGamma. The stacked histograms show the SM expectation, normalized using scale factors derived from the simultaneous fit to all backgrounds. The ``Data/SM" plots show the ratio of data events to the total SM expectation. The hatched uncertainty band around the SM expectation and in the ratio plot illustrates the combination of MC statistical and detector-related systematic uncertainties. The rightmost bin includes all overflows.}
%    \label{fig:CRs}
%  \end{center}
%\end{figure}

\subsection{\boldmath$t\bar{t}+Z$ by \boldmath$t\bar{t}+\gamma$}

The $\ttbar+Z$ background, where the $Z$ boson decays into neutrinos, is an irreducible background and is increased with respect to the Run 1 analysis. Designing a CR to estimate the $\ttbar+Z$ background by using the charged leptonic $Z$ boson decays would be favorable. However, such CR is difficult to design due to low statistics and the small branching fraction to leptons. In particular a 2-lepton CR suffers from a large contamination of $\ttbar$ and $Z$ + jets processes. For this reason, another data driven approach is followed by building a one-lepton CR for $\ttbar\gamma$ which is a similar process. A zero-lepton region was considered as a validation region but it was found to have a too low $\ttbar\gamma$ contribution, with $\gamma$+jets being the main contaminant.\\

The CR is designed to minimize the differences between the two processes and keep the theoretical uncertainties from the extrapolation of the $\gamma$ to the \Zboson\ low. 

The $\ttbar+\gamma$ CR requires exactly one photon, exactly one signal lepton (electron or muon, and at least four jets of which at least two are required to be b-tagged. Moreover, due to the difference in mass between the \Zboson\ and the $\gamma$, to mimic the $Z \rightarrow \nu\nu$ decay, the highest \pT\ photon is required to have $\pt>150\gev$. 

Unlike the CRTs, CRW, and CRST one-lepton control regions the lepton is not treated as a jet and unlike the CRZs the leptons are not removed from the \met\ calculation. Instead, the photon is used to model the \met\ since the \met\ from $\ttbar+Z$ in the SR originates mostly from the neutrino decay of the \Zboson. 
The details of the selection for the one-lepton CR are summarized in Table~\ref{tb:ttG_1lepSel}.

\begin{table}[htpb]
  \caption{Selection criteria for the common \Wjets, single-top, and $\ttbar+\gamma$ control-region definitions.} 
  \begin{center}
    \def\arraystretch{1.4}
    \begin{tabular}{c||c|c|c}
      \hline \hline
      Selection                       & CRW                                             & CRST   & CRTTGamma            \\ \hline \hline
      Trigger                         & \multicolumn{2}{c|}{\met}                       & electron or muon                        \\ 
      \hline
      $N_{\ell}$                      & \multicolumn{3}{c}{1}                                                           \\ 
     \hline
     $\pT^{\ell}$                     & \multicolumn{2}{c|}{$>20 \gev$}                 & $>28\gev$                     \\ 
     \hline
      $N_{\gamma}$                    & \multicolumn{2}{c|}{-}                          & 1                             \\ 
     \hline
     $\pT^{\gamma}$                   & \multicolumn{2}{c|}{-}                          & $>150\gev$                    \\ 
     \hline
     $N_{\mathrm{jet}}$               & \multicolumn{2}{c|}{$\ge 4$ (including electron or muon)} & $\ge4$                        \\
      \hline     
      \ptzero, \ptone,\pttwo,\ptthree & \multicolumn{3}{c}{$80,80,40,40\gev$}                                           \\
      \hline
      \nBJet                          & $1$                                             & \multicolumn{2}{c}{$ \ge 2 $} \\
\hline   
      $\dphijettwomet$                & \multicolumn{2}{c|}{$>0.4$}                     & -                             \\ 
      \hline
      $\mT(\ell,\met)$                & \multicolumn{2}{c|}{$[30,100]\gev$}             & -                             \\ 
\hline
       $\drblmin$                     & \multicolumn{2}{c|}{$>2.0$}                     & -                             \\ 
\hline
 $\met$                               & \multicolumn{2}{c|}{$>250 \gev$}                & -                             \\ 
\hline
 $\drbjetbjet$                        & -                                               & $>1.5$ & -                    \\ 
\hline
 $\mantikttwelvezero$                 & $<60\gev$                                               & $>120\gev$ & -                    \\ 
\hline
 $\mtbmin$                 & -                                               & $>200\gev$ & -                    \\ 
       \hline\hline
    \end{tabular}
  \end{center}
  \label{tab:selectionCRWSTTTGamma}
\end{table}

%\begin{table}[htpb]
%  \caption{Selection for the $\ttbar+\gamma$ 1 lepton CR. The same triggers as described in Table~\ref{tb:lepTriggers} are used.}% and the same signal lepton requirements are mad as in Tables~\ref{tb:electronsSignal} and~\ref{tb:muonsSignal}}
%  \begin{center}
%    \begin{tabular}{c|c}
%      \hline \hline
%      Selection                 & Requirement     \\
%      \hline \hline
%      Event selection & Event cleaning \\
%      \hline
%      % Trigger (Data 2015) &\texttt{HLT\_g120\_loose}  \\ 
%      % \hline
%      % Trigger (Data 2016) & \texttt{HLT\_g140\_loose}  \\ 
%      %\hline
%      Leptons & exactly 1 \\
%      Lepton \pt & 28 GeV \\
%      \hline
%      Photons & exactly 1\\
%      \hline
%      jet multiplicity & $ \ge 4 $ \\
%      \hline
%      Jet \pT\ & (80,80,40,40) GeV \\
%      \hline
%      b-jet multiplicity & $\ge 2$ \\
%      \hline
%      $\gamma$ \pT\ & $> 150$ GeV \\
%      \hline\hline
%    \end{tabular}
%  \end{center}
%  \label{tb:ttG_1lepSel}
%\end{table}

%Figure \ref{ig:ttVFakeLepCheck} shows the fit to the \met\ modeled with the photon and the \mt\ of the photon and lepton.  \\

%\begin{figure}[htbp]
%\begin{center}
%\includegraphics[width=0.49\textwidth]{figures/Met_CRTTGamma_withRatio_log.eps}
%\includegraphics[width=0.49\textwidth]{figures/MtMetLep_CRTTGamma_withRatio_log.eps}
%\caption{\label{fig:ttVFakeLepCheck} Prefit distributions of the \met\ and \mtlepmet\ for fake lepton checks. Agreement at low \mtlepmet\ is reasonable indicating no significant contributions from fake leptons. The ratio between data and MC is given in the bottom panel. The hashed area in both the top and lower panel represents the uncertainty due to MC statistics.}
%\end{center}
%\end{figure}
%\clearpage

Truth studies are performed in order to check the differencies in
kinematic distributions. The truth \pT\ ratio for a 2-bjet
selection are shown in Fig.~\ref{fig:ttZ_vs_ttGamma_pt}.

%\begin{figure}[htpb]
%\centering
%%\includegraphics[scale=0.4]{figures/ttGamma/TruthStudies/Pt150}
%\includegraphics[scale=0.4]{figures/UnityNorm_pT150}
%\caption{Truth boson \pT\ ratio.}
%\label{fig:ttZ_vs_ttGamma_pt}
%\end{figure}
%\clearpage

\subsection{QCD multi-jet and all-hadronic \boldmath$t\bar{t}$ }
\label{sec:QCDbkgd}
The background from the production of multijet events and all-hadronic
\ttbar~ events is estimated with the jet smearing method. The main assumption of the \verb| JetSmearing| method is that the QCD background is dominated by the mis-measurement of multiple jets. The term {\it mis-measurement} refers to cases in which the hadronisation of partons is not fully reconstructed by ATLAS and cases in which the hadronisation (particularly for heavy-flavour quarks) produces real \met in the form of neutrinos.

The attributed sources of mis-measurement which are taken into account by the method are the following:
\begin{itemize}
\item Hadronic calorimeters are not perfect; there is some limit to granularity of calorimeters therefore they are not able to perfectly measure the energy of all particles.
\item Since jets are clusters of showering particles it is possible that not all of these particles can be contained within the jet radius. Some of the showering particles may be lost due to interacting with non-detector material. Additionally background particles from various different sources may enter into a jet cone, although this effect is reduced by cosmic background vetos and the overlap removal of other jets, photons, electrons and muons.
\item Not all jets are fully contained within the calorimeter systems, if a jet has large amounts of energy it can punch through to the muon system and potentially large amounts of the energy can be lost. This is one such source of the non-Gaussian part of the jet response; this effect always gives lost energy rather than an overestimation of the energy.
\item Jets that are close to areas of large amounts of dead material are vetoed, however there are still regions with small amounts of dead material in the calorimeters which can cause particles to deposit their energy. The sources of dead material include damaged or inactive parts of the detector, services for running electronics to the detector and various non-instrumented region from the support structure of ATLAS.
\item In decays of heavy flavour quarks, particularly those of b-quarks, real missing energy can be present from neutrinos. Typically: $\sim$76\% of b-quark decays will be hadronic (including hadronic tau decays); leaving 12\% of decays with muons and muon neutrinos; and 12\% of decays with electrons and electron neutrinos. The decays involving neutrinos will carry a fraction of the jet energy with them, this gives a larger non-Gaussian tail in the case of b-tagged jets.
\end{itemize}

The recommended procedure is followed and both control regions and validation regions were designed. \\

\begin{figure}[htpb]
  \begin{center}
    \subfloat[]{\includegraphics[width=0.44\textwidth]{figures/Znunu/MT2Chi2Lep_CRZAB_T0_log}}
    \subfloat[]{\includegraphics[width=0.44\textwidth]{figures/Znunu/MetLep_CRZE_log}}\\
    \subfloat[]{\includegraphics[width=0.44\textwidth]{figures/ttbar/CA_RISR_CRTopC}}
    \subfloat[]{\includegraphics[width=0.44\textwidth]{figures/Wjets/MtBMax_CRW_log}}\\
     \subfloat[]{\includegraphics[width=0.44\textwidth]{figures/singleTop/JetPt_1__CRST_log}}
   \subfloat[]{\includegraphics[width=0.44\textwidth]{figures/ttGamma/SigPhotonPt_0__CRTTGamma_log}}

    \caption[Distribution of kinematic variables in CRZ, CRTC, CRW, CRST, and CRTTGamma.]{Distributions of (a)~\mttwoprime\ in CRZAB-T0, (b)~\metprime\ in CRZE, (c)~\rISR\ in CRTC, 
      (d)~\mtbmax\ in CRW, (e)~the transverse momentum of the second-leading-\pT\ jet in CRST,  and
      (f)~the photon \pT\ in CRTTGamma. The stacked histograms show the SM prediction, normalized using scale factors derived from the simultaneous fit to all backgrounds. The ``Data/SM" plots show the ratio of data events to the total SM prediction. The hatched uncertainty band around the SM prediction and in the ratio plot illustrates the combination of MC statistical and detector-related systematic uncertainties. The rightmost bin includes overflow events.}
    \label{fig:CRs}
  \end{center}
\end{figure}


\section{Systematic Uncertainties}
\label{sec:Systematics}

Systematic uncertainties are associated with the predictions of all background components and the expected signal yields. The systematic uncertainties can be categorized into two sources: experimental and theoretical uncertainties. These systematic uncertainties can impact the expected event yields in the control and signal region as well as the transfer factors used when extrapolating the background expectation from the control to the signal region. \\

The main sources of detector-related systematic uncertainties in the SM background estimates originate from jet energy scale (JES) and jet energy resolution (JER), which reaches up to 16\%, $b-$tagging efficiency, up to 9\%, \met\ soft term, up to 6\%, and pileup, up to 14\%.  There is also an uncertainty in measured luminosity of 3.2\%.

Theory uncertainties affecting the background normalization and kinematic distribution shapes largely impact the background prediction in the signal regions, as they directly affect the background normalization and acceptance times efficiency. If a background normalization is determined by making use of dedicated control regions, then only systematics affecting the analysis acceptance are relevant. Statistical uncertainties in the evaluation of systematics are neglected in general; where necessary, selection cuts are loosened to make the systematic comparison statistically meaningful. \\ %The remainder of this section is dedicated to the discussion on how the theory systematic uncertainties have been derived for each of the background processes considered. \\

Theory uncertainties are evaluated as follows:
\begin{itemize}
	\item $W/Z+$jets: renormalization and factorization scales, and merging and resummation scales, are varied in SHERPA.  These are factors that come out of perturbation theory but shouldn't have any impact on physical observables.  The uncertainty for $W/Z+$jets are up to 5\% and 19\% respectively.
	\item $t\bar{t}$: uncertainty is due to hard scattering generation from the choice of model and emission of additional partons in the initial and final states, comparing generators POWHEG-Box+PYTHIA vs. HERWIG++ and SHERPA.  This impacts SRC the largest, by 11-71\%.
	\item $t\bar{t}+W/Z$: the uncertainty is estimated by varying the renormalization and factorization scales and choice of PDF and comparing generators.  This results in an uncertainty of up to 6\%.
	\item Single $t$: the background is dominated by $Wt$ subprocess and uncertainties are evaulated for the choice of parton-showering model and parton emission for initial- and final-state radiation.  A 30\% uncertainty is applied to the background estimate to account for the effect of interference between single-top quark and $t\bar{t}$ production.
\end{itemize}


\section{Fitting Procedure}
\label{sec:fitting}

The SM backgrounds in each SR are estimated with a profile likelihood fit using the observed number of events in the CRs.  The correlations in the systematic uncertainties that are common between SRs and CRs  are treated as nuisance parameters in the fit and are modeled by Gaussian probability density functions.  A normalization factor is then derived from the fit.  For backgrounds without a defined CR, contributions are estimated using the cross section.  \\


%The fitted scale factors for the backgrounds are summarized in Table \ref{table.scale.factors}.  
Tables \ref{table.bkgonly.SRA}, \ref{table.bkgonly.SRB}, \ref{table.bkgonly.SRC1to3}, \ref{table.bkgonly.SRC4to5}, and \ref{table.bkgonly.SRE} show the background and signal yields in simulation before and after the scale factors are applied for SRA, B, C, D, and E respectively. %, and Figures \ref{fig:bkgSRA}, \ref{fig:bkgSRB}, \ref{fig:bkgSRC1to3}, \ref{fig:bkgSRC4to5}, \ref{fig:bkgSRD}, and \ref{fig:bkgSRE} show a graphical breakdown of the backgrounds in these signal regions so that the proportion of each is easily seen. \\

%\begin{table}
%  \begin{center}
%    \begin{tabular*}{\textwidth}{@{\extracolsep{\fill}}lr}
      \noalign{\smallskip}\hline\noalign{\smallskip}
      {\bf MC sample}           & Fitted scale factor        \\[-0.05cm]
      \noalign{\smallskip}\hline\noalign{\smallskip}
      $t\bar{t}$ (SRA\_TT) &   $1.173 \pm 0.146$              \\
      $t\bar{t}$ (SRA\_TW) &   $1.138 \pm 0.112$              \\
      $t\bar{t}$ (SRA\_T0) &   $0.898 \pm 0.121$              \\
      \noalign{\smallskip}\hline\noalign{\smallskip}
      $t\bar{t}$ (SRB\_TT) &   $1.202 \pm 0.156$              \\
      $t\bar{t}$ (SRB\_TW) &   $0.969 \pm 0.0681$              \\
      $t\bar{t}$ (SRB\_T0) &   $0.924 \pm 0.0525$              \\
      \noalign{\smallskip}\hline\noalign{\smallskip}
      $t\bar{t}$ (SRC) &   $0.707 \pm 0.0498$              \\
      $t\bar{t}$ (SRD) &   $0.945 \pm 0.103$              \\
      $t\bar{t}$ (SRE) &   $1.012 \pm 0.180$              \\
      \noalign{\smallskip}\hline\noalign{\smallskip}
      $W+$jets &   $1.267 \pm 0.146$              \\
      \noalign{\smallskip}\hline\noalign{\smallskip}
      $Z+$jets (SRA,B TT and TW) &   $1.170 \pm 0.238$              \\
      $Z+$jets (SRA,B T0) &   $1.131 \pm 0.144$              \\
      $Z+$jets (SRD) &   $1.035 \pm 0.146$              \\
      $Z+$jets (SRE) &   $1.185 \pm 0.152$              \\
      \noalign{\smallskip}\hline\noalign{\smallskip}
      Single top &   $1.166 \pm 0.390$              \\
      $t\bar{t}$$\gamma$ &   $1.290 \pm 0.204$              \\
      \noalign{\smallskip}\hline\noalign{\smallskip}
    \end{tabular*}

%  \end{center}
%  \caption{Fitted scale factors for the MC background samples based on
%    36.07 \ifb of data.}
%  \label{table.scale.factors}
%\end{table}
%\clearpage


%\input{histFitterTables/YieldsTable.SRA.tex}
\begin{table}[htpb]
  \caption{Observed and expected yields, before and after the fit, for SRA and SRB.
The uncertainties include MC statistical uncertainties, detector-related systematic uncertainties, and theoretical uncertainties in the extrapolation from CR to SR.}
  \begin{center}
  \small
{\renewcommand{\arraystretch}{1.2}
\begin{tabular}{
S[table-alignment=left]
S[table-number-alignment=left]
S[table-number-alignment=left]
S[table-number-alignment=left]
S[table-number-alignment=left]
S[table-number-alignment=left]
S[table-number-alignment=left]
S[table-number-alignment=left]
S[table-number-alignment=left]
S[table-number-alignment=left]
S[table-number-alignment=left]
S[table-number-alignment=left]
S[table-number-alignment=left]
}
\hline\hline
 & \multicolumn{2}{c}{SRA-TT} & \multicolumn{2}{c}{SRA-TW} & \multicolumn{2}{c}{SRA-T0} & \multicolumn{2}{c}{SRB-TT} & \multicolumn{2}{c}{SRB-TW} & \multicolumn{2}{c}{SRB-T0}\\ \hline 
{Observed} & \multicolumn{1}{l}{$11$} &  & \multicolumn{1}{l}{$9$} &  & \multicolumn{1}{l}{$18$} &  & \multicolumn{1}{l}{$38$} &  & \multicolumn{1}{l}{$53$} &  & \multicolumn{1}{l}{$206$} & \\ \hline \hline 
 \multicolumn{3}{r}{Fitted background events} & \multicolumn{10}{r}{}\\ \hline 
{Total SM} & \multicolumn{2}{l}{$8.6\phantom{0} \pm 2.1\phantom{0}$} & \multicolumn{2}{l}{$9.3\phantom{0} \pm 2.2\phantom{0}$} & \multicolumn{2}{l}{$18.7\phantom{0} \pm 2.7\phantom{0}$} & \multicolumn{2}{l}{$39.3\phantom{0} \pm 7.6\phantom{0}$} & \multicolumn{2}{l}{$52.4\phantom{0} \pm 7.4\phantom{0}$} & \multicolumn{2}{l}{$179\phantom{000} \pm 26\phantom{000}$}\\ \hline 
{\ttbar} & \multicolumn{2}{l}{$0.71\;_{-\;0.71}^{+\;0.91}$} & \multicolumn{2}{l}{$0.51\;_{-\;0.51}^{+\;0.55}$} & \multicolumn{2}{l}{$\phantom{1}1.31 \pm 0.64$} & \multicolumn{2}{l}{$\phantom{3}7.3\phantom{0} \pm 4.3\phantom{0}$} & \multicolumn{2}{l}{$12.4\phantom{0} \pm 5.9\phantom{0}$} & \multicolumn{2}{l}{$\phantom{1}43\phantom{000} \pm 22\phantom{000}$}\\ 
{\Wjets} & \multicolumn{2}{l}{$0.82 \pm 0.15$} & \multicolumn{2}{l}{$0.89 \pm 0.56$} & \multicolumn{2}{l}{$\phantom{1}2.00 \pm 0.83$} & \multicolumn{2}{l}{$\phantom{3}7.8\phantom{0} \pm 2.8\phantom{0}$} & \multicolumn{2}{l}{$\phantom{5}4.8\phantom{0} \pm 1.2\phantom{0}$} & \multicolumn{2}{l}{$\phantom{1}25.8\phantom{0} \pm \phantom{2}8.8\phantom{0}$}\\ 
{\Zjets} & \multicolumn{2}{l}{$2.5\phantom{0} \pm 1.3\phantom{0}$} & \multicolumn{2}{l}{$4.9\phantom{0} \pm 1.9\phantom{0}$} & \multicolumn{2}{l}{$\phantom{1}9.8\phantom{0} \pm 1.6\phantom{0}$} & \multicolumn{2}{l}{$\phantom{3}9.0\phantom{0} \pm 2.8\phantom{0}$} & \multicolumn{2}{l}{$16.8\phantom{0} \pm 4.1\phantom{0}$} & \multicolumn{2}{l}{$\phantom{1}60.7\phantom{0} \pm \phantom{2}9.6\phantom{0}$}\\ 
{\ttV} & \multicolumn{2}{l}{$3.16 \pm 0.66$} & \multicolumn{2}{l}{$1.84 \pm 0.39$} & \multicolumn{2}{l}{$\phantom{1}2.60 \pm 0.53$} & \multicolumn{2}{l}{$\phantom{3}9.3\phantom{0} \pm 1.7\phantom{0}$} & \multicolumn{2}{l}{$10.8\phantom{0} \pm 1.6\phantom{0}$} & \multicolumn{2}{l}{$\phantom{1}20.5\phantom{0} \pm \phantom{2}3.2\phantom{0}$}\\ 
{Single top} & \multicolumn{2}{l}{$1.20 \pm 0.81$} & \multicolumn{2}{l}{$0.70 \pm 0.42$} & \multicolumn{2}{l}{$\phantom{1}2.9\phantom{0} \pm 1.5\phantom{0}$} & \multicolumn{2}{l}{$\phantom{3}4.2\phantom{0} \pm 2.2\phantom{0}$} & \multicolumn{2}{l}{$\phantom{5}5.9\phantom{0} \pm 2.8\phantom{0}$} & \multicolumn{2}{l}{$\phantom{1}26\phantom{000} \pm 13\phantom{000}$}\\ 
{Dibosons} & \multicolumn{2}{c}{${-} {-}$} & \multicolumn{2}{l}{$0.35 \pm 0.26$} & \multicolumn{2}{c}{${-} {-}$} & \multicolumn{2}{l}{$\phantom{3}0.13 \pm 0.07$} & \multicolumn{2}{l}{$\phantom{5}0.60 \pm 0.43$} & \multicolumn{2}{l}{$\phantom{17}1.04 \pm \phantom{2}0.73$}\\ 
{Multijets} & \multicolumn{2}{l}{$0.21 \pm 0.10$} & \multicolumn{2}{l}{$0.14 \pm 0.09$} & \multicolumn{2}{l}{$\phantom{1}0.12 \pm 0.07$} & \multicolumn{2}{l}{$\phantom{3}1.54 \pm 0.64$} & \multicolumn{2}{l}{$\phantom{5}1.01 \pm 0.88$} & \multicolumn{2}{l}{$\phantom{17}1.8\phantom{0} \pm \phantom{2}1.5\phantom{0}$}\\ \hline \hline 
 \multicolumn{3}{r}{Expected events before fit} & \multicolumn{10}{r}{}\\ \hline 
{Total SM} & \multicolumn{2}{l}{$7.1\phantom{3}\phantom{3}$}  & \multicolumn{2}{l}{$7.9\phantom{3}\phantom{3}$}  & \multicolumn{2}{l}{$16.3\phantom{3}\phantom{3}$}  & \multicolumn{2}{l}{$32.4\phantom{3}\phantom{3}$}  & \multicolumn{2}{l}{$46.1\phantom{3}\phantom{3}$}  & \multicolumn{2}{l}{$162\phantom{3}\phantom{2}$} \\ \hline 
{\ttbar} & \multicolumn{2}{l}{$0.60\phantom{3}\phantom{4}$}  & \multicolumn{2}{l}{$0.45\phantom{3}\phantom{4}$}  & \multicolumn{2}{l}{$\phantom{1}1.45\phantom{3}\phantom{4}$}  & \multicolumn{2}{l}{$\phantom{3}6.1\phantom{3}\phantom{3}$}  & \multicolumn{2}{l}{$12.8\phantom{3}\phantom{3}$}  & \multicolumn{2}{l}{$\phantom{1}47\phantom{3}\phantom{2}$} \\ 
{\Wjets} & \multicolumn{2}{l}{$0.65\phantom{3}\phantom{4}$}  & \multicolumn{2}{l}{$0.70\phantom{3}\phantom{4}$}  & \multicolumn{2}{l}{$\phantom{1}1.58\phantom{3}\phantom{4}$}  & \multicolumn{2}{l}{$\phantom{3}6.1\phantom{3}\phantom{3}$}  & \multicolumn{2}{l}{$\phantom{5}3.83\phantom{3}\phantom{3}$}  & \multicolumn{2}{l}{$\phantom{1}20.4\phantom{3}\phantom{3}$} \\ 
{\Zjets} & \multicolumn{2}{l}{$2.15\phantom{3}\phantom{3}$}  & \multicolumn{2}{l}{$4.2\phantom{3}\phantom{3}$}  & \multicolumn{2}{l}{$\phantom{1}8.63\phantom{3}\phantom{3}$}  & \multicolumn{2}{l}{$\phantom{3}7.7\phantom{3}\phantom{3}$}  & \multicolumn{2}{l}{$14.4\phantom{3}\phantom{3}$}  & \multicolumn{2}{l}{$\phantom{1}53.6\phantom{3}\phantom{3}$} \\ 
{\ttV} & \multicolumn{2}{l}{$2.46\phantom{3}\phantom{4}$}  & \multicolumn{2}{l}{$1.43\phantom{3}\phantom{4}$}  & \multicolumn{2}{l}{$\phantom{1}2.02\phantom{3}\phantom{4}$}  & \multicolumn{2}{l}{$\phantom{3}7.3\phantom{3}\phantom{3}$}  & \multicolumn{2}{l}{$\phantom{5}8.4\phantom{3}\phantom{3}$}  & \multicolumn{2}{l}{$\phantom{1}15.9\phantom{3}\phantom{3}$} \\ 
{Single top} & \multicolumn{2}{l}{$1.03\phantom{3}\phantom{4}$}  & \multicolumn{2}{l}{$0.60\phantom{3}\phantom{4}$}  & \multicolumn{2}{l}{$\phantom{1}2.5\phantom{3}\phantom{3}$}  & \multicolumn{2}{l}{$\phantom{3}3.6\phantom{3}\phantom{3}$}  & \multicolumn{2}{l}{$\phantom{5}5.1\phantom{3}\phantom{3}$}  & \multicolumn{2}{l}{$\phantom{1}22.4\phantom{3}\phantom{2}$} \\ 
{Dibosons} & \multicolumn{2}{l}{${-} {-}$} & \multicolumn{2}{l}{$0.35\phantom{3}\phantom{4}$}  & \multicolumn{2}{l}{${-} {-}$} & \multicolumn{2}{l}{$\phantom{3}0.13\phantom{3}\phantom{4}$}  & \multicolumn{2}{l}{$\phantom{5}0.60\phantom{3}\phantom{4}$}  & \multicolumn{2}{l}{$\phantom{17}1.03\phantom{3}\phantom{4}$} \\ 
{Multijets} & \multicolumn{2}{l}{$0.21\phantom{3}\phantom{4}$}  & \multicolumn{2}{l}{$0.14\phantom{3}\phantom{4}$}  & \multicolumn{2}{l}{$\phantom{1}0.12\phantom{3}\phantom{4}$}  & \multicolumn{2}{l}{$\phantom{3}1.54\phantom{3}\phantom{4}$}  & \multicolumn{2}{l}{$\phantom{5}1.01\phantom{3}\phantom{4}$}  & \multicolumn{2}{l}{$\phantom{17}1.8\phantom{3}\phantom{3}$} \\ 
\hline\hline
\end{tabular}

}
\end{center}
\label{tab:srABYields}
\end{table}

%\begin{figure}[H] 
%\begin{center}
% \includegraphics[width=0.15\textwidth]{figures/barCharts/bkgBarsSRA1.pdf} %\hspace{0.05\textwidth}
% \includegraphics[width=0.15\textwidth]{figures/barCharts/bkgBarsSRA2.pdf} %\hspace{0.05\textwidth}
% \includegraphics[width=0.15\textwidth]{figures/barCharts/bkgBarsSRA3.pdf} %\hspace{0.05\textwidth}
%  \includegraphics[width=0.15\textwidth]{figures/barCharts/bkgBarsSRB1.pdf} %\hspace{0.05\textwidth}
% \includegraphics[width=0.15\textwidth]{figures/barCharts/bkgBarsSRB2.pdf} %\hspace{0.05\textwidth}
% \includegraphics[width=0.15\textwidth]{figures/barCharts/bkgBarsSRB3.pdf} %\hspace{0.05\textwidth}
%    \caption[Backgrounds in SRA-TT, -TW, and -T0, and in SRB-TT, -TW, and -T0.]{Breakdown of the backgrounds in SRA-TT, -TW, and -T0, and in SRB-TT, -TW, and -T0 from left to right.}
%    \label{fig:bkgSRA}
%\end{center}
%\end{figure}

\clearpage


%\input{histFitterTables/YieldsTable.SRB.tex}
%
%\begin{figure}[H] 
%\begin{center}
% \includegraphics[width=0.2\textwidth]{figures/barCharts/bkgBarsSRB1.pdf} %\hspace{0.05\textwidth}
% \includegraphics[width=0.2\textwidth]{figures/barCharts/bkgBarsSRB2.pdf} %\hspace{0.05\textwidth}
% \includegraphics[width=0.2\textwidth]{figures/barCharts/bkgBarsSRB3.pdf} %\hspace{0.05\textwidth}
%    \caption{Breakdown of the backgrounds in SRB-TT, -TW, and -T0 from left to right.}
%    \label{fig:bkgSRB}
%\end{center}
%\end{figure}
%
%\clearpage
%
%\input{histFitterTables/YieldsTable.SRC1to3.tex}

\begin{table}[htpb]
  \caption{Observed and expected yields, before and after the fit.
The uncertainties include MC statistical uncertainties, detector-related systematic uncertainties, and theoretical uncertainties in the extrapolation from CR to SR.}
  \begin{center}
{\renewcommand{\arraystretch}{1.2}
\begin{tabular}{
S[table-alignment=left]
S[table-number-alignment=left]
S[table-number-alignment=left]
S[table-number-alignment=left]
S[table-number-alignment=left]
S[table-number-alignment=left]
S[table-number-alignment=left]
S[table-number-alignment=left]
S[table-number-alignment=left]
S[table-number-alignment=left]
S[table-number-alignment=left]
}
\hline\hline
 & \multicolumn{2}{c}{SRC1} & \multicolumn{2}{c}{SRC2} & \multicolumn{2}{c}{SRC3} & \multicolumn{2}{c}{SRC4} & \multicolumn{2}{c}{SRC5}\\ \hline 
{Observed} & \multicolumn{1}{l}{$20$} &  & \multicolumn{1}{l}{$22$} &  & \multicolumn{1}{l}{$22$} &  & \multicolumn{1}{l}{$1$} &  & \multicolumn{1}{l}{$0$} & \\ \hline \hline 
 \multicolumn{3}{r}{Fitted background events} & \multicolumn{8}{r}{}\\ \hline 
{Total SM} & \multicolumn{2}{l}{$20.6\phantom{0} \pm 6.5\phantom{0}$} & \multicolumn{2}{l}{$27.6\phantom{0} \pm 4.9\phantom{0}$} & \multicolumn{2}{l}{$18.9\phantom{0} \pm 3.4\phantom{0}$} & \multicolumn{2}{l}{$7.7\phantom{0} \pm 1.2\phantom{0}$} & \multicolumn{2}{l}{$0.91 \pm 0.73$}\\ \hline 
{\ttbar} & \multicolumn{2}{l}{$12.9\phantom{0} \pm 5.9\phantom{0}$} & \multicolumn{2}{l}{$22.1\phantom{0} \pm 4.3\phantom{0}$} & \multicolumn{2}{l}{$14.6\phantom{0} \pm 3.2\phantom{0}$} & \multicolumn{2}{l}{$4.91 \pm 0.97$} & \multicolumn{2}{l}{$0.63\;_{-\;0.63}^{+\;0.70}$}\\ 
{\Wjets} & \multicolumn{2}{l}{$\phantom{2}0.80 \pm 0.37$} & \multicolumn{2}{l}{$\phantom{2}1.93 \pm 0.49$} & \multicolumn{2}{l}{$\phantom{1}1.91 \pm 0.62$} & \multicolumn{2}{l}{$1.93 \pm 0.46$} & \multicolumn{2}{l}{$0.21 \pm 0.12$}\\ 
{\Zjets} & \multicolumn{2}{c}{${-} {-}$} & \multicolumn{2}{c}{${-} {-}$} & \multicolumn{2}{c}{${-} {-}$} & \multicolumn{2}{c}{${-} {-}$} & \multicolumn{2}{c}{${-} {-}$}\\ 
{\ttV} & \multicolumn{2}{l}{$\phantom{2}0.29 \pm 0.16$} & \multicolumn{2}{l}{$\phantom{2}0.59 \pm 0.38$} & \multicolumn{2}{l}{$\phantom{1}0.56 \pm 0.31$} & \multicolumn{2}{l}{$0.08 \pm 0.08$} & \multicolumn{2}{l}{$0.06 \pm 0.02$}\\ 
{Single top} & \multicolumn{2}{l}{$\phantom{2}1.7\phantom{0} \pm 1.3\phantom{0}$} & \multicolumn{2}{l}{$\phantom{2}1.2\phantom{0}\;_{-\;1.2}^{+\;1.4\phantom{0}}$} & \multicolumn{2}{l}{$\phantom{1}1.22 \pm 0.69$} & \multicolumn{2}{l}{$0.72 \pm 0.37$} & \multicolumn{2}{c}{${-} {-}$}\\ 
{Dibosons} & \multicolumn{2}{l}{$\phantom{2}0.39 \pm 0.33$} & \multicolumn{2}{l}{$\phantom{2}0.21\;_{-\;0.21}^{+\;0.23}$} & \multicolumn{2}{l}{$\phantom{1}0.28 \pm 0.18$} & \multicolumn{2}{c}{${-} {-}$} & \multicolumn{2}{c}{${-} {-}$}\\ 
{Multijets} & \multicolumn{2}{l}{$\phantom{2}4.6\phantom{0} \pm 2.4\phantom{0}$} & \multicolumn{2}{l}{$\phantom{2}1.58 \pm 0.77$} & \multicolumn{2}{l}{$\phantom{1}0.32 \pm 0.17$} & \multicolumn{2}{l}{$0.04 \pm 0.02$} & \multicolumn{2}{c}{${-} {-}$}\\ \hline \hline 
 \multicolumn{3}{r}{Expected events before fit} & \multicolumn{8}{r}{}\\ \hline 
{Total SM} & \multicolumn{2}{l}{$25.4\phantom{3}\phantom{3}$}  & \multicolumn{2}{l}{$36.0\phantom{3}\phantom{3}$}  & \multicolumn{2}{l}{$24.2\phantom{3}\phantom{3}$}  & \multicolumn{2}{l}{$9.2\phantom{3}\phantom{3}$}  & \multicolumn{2}{l}{$1.1\phantom{3}\phantom{4}$} \\ \hline 
{\ttbar} & \multicolumn{2}{l}{$18.2\phantom{3}\phantom{3}$}  & \multicolumn{2}{l}{$31.2\phantom{3}\phantom{3}$}  & \multicolumn{2}{l}{$20.6\phantom{3}\phantom{3}$}  & \multicolumn{2}{l}{$7.0\phantom{3}\phantom{4}$}  & \multicolumn{2}{l}{$0.89\phantom{3}\phantom{4}$} \\ 
{\Wjets} & \multicolumn{2}{l}{$\phantom{2}0.64\phantom{3}\phantom{4}$}  & \multicolumn{2}{l}{$\phantom{2}1.53\phantom{3}\phantom{4}$}  & \multicolumn{2}{l}{$\phantom{1}1.51\phantom{3}\phantom{4}$}  & \multicolumn{2}{l}{$1.53\phantom{3}\phantom{4}$}  & \multicolumn{2}{l}{$0.17\phantom{3}\phantom{4}$} \\ 
{\Zjets} & \multicolumn{2}{l}{${-} {-}$} & \multicolumn{2}{l}{${-} {-}$} & \multicolumn{2}{l}{${-} {-}$} & \multicolumn{2}{l}{${-} {-}$} & \multicolumn{2}{l}{${-} {-}$}\\ 
{\ttV} & \multicolumn{2}{l}{$\phantom{2}0.22\phantom{3}\phantom{4}$}  & \multicolumn{2}{l}{$\phantom{2}0.46\phantom{3}\phantom{4}$}  & \multicolumn{2}{l}{$\phantom{1}0.44\phantom{3}\phantom{4}$}  & \multicolumn{2}{l}{$0.07\phantom{3}\phantom{4}$}  & \multicolumn{2}{l}{$0.05\phantom{3}\phantom{4}$} \\ 
{Single top} & \multicolumn{2}{l}{$\phantom{2}1.44\phantom{3}\phantom{3}$}  & \multicolumn{2}{l}{$\phantom{2}1.0\phantom{3}\phantom{3}$}  & \multicolumn{2}{l}{$\phantom{1}1.04\phantom{3}\phantom{4}$}  & \multicolumn{2}{l}{$0.62\phantom{3}\phantom{4}$}  & \multicolumn{2}{l}{${-} {-}$}\\ 
{Dibosons} & \multicolumn{2}{l}{$\phantom{2}0.39\phantom{3}\phantom{4}$}  & \multicolumn{2}{l}{$\phantom{2}0.21\phantom{3}\phantom{4}$}  & \multicolumn{2}{l}{$\phantom{1}0.28\phantom{3}\phantom{4}$}  & \multicolumn{2}{l}{${-} {-}$} & \multicolumn{2}{l}{${-} {-}$}\\ 
{Multijets} & \multicolumn{2}{l}{$\phantom{2}4.6\phantom{3}\phantom{3}$}  & \multicolumn{2}{l}{$\phantom{2}1.58\phantom{3}\phantom{4}$}  & \multicolumn{2}{l}{$\phantom{1}0.32\phantom{3}\phantom{4}$}  & \multicolumn{2}{l}{$0.04\phantom{3}\phantom{4}$}  & \multicolumn{2}{l}{${-} {-}$}\\ 
\hline\hline
\end{tabular}

}
\end{center}
\label{tab:srCYields}
\end{table}

%\begin{figure}[H] 
%\begin{center}
% \includegraphics[width=0.2\textwidth]{figures/barCharts/bkgBarsSRC1to31.pdf} %\hspace{0.05\textwidth}
% \includegraphics[width=0.2\textwidth]{figures/barCharts/bkgBarsSRC1to32.pdf} %\hspace{0.05\textwidth}
% \includegraphics[width=0.2\textwidth]{figures/barCharts/bkgBarsSRC1to33.pdf} %\hspace{0.05\textwidth}
%  \includegraphics[width=0.2\textwidth]{figures/barCharts/bkgBarsSRC4to51.pdf} %\hspace{0.05\textwidth}
% \includegraphics[width=0.2\textwidth]{figures/barCharts/bkgBarsSRC4to52.pdf} %\hspace{0.05\textwidth}
%    \caption[Breakdown of the backgrounds in SRC1-5.]{Breakdown of the backgrounds in SRC1-5 from left to right.}
%    \label{fig:bkgSRC1to3}
%\end{center}
%\end{figure}

\clearpage

%\input{histFitterTables/YieldsTable.SRC4to5.tex}
%
%\begin{figure}[H] 
%\begin{center}
% \includegraphics[width=0.2\textwidth]{figures/barCharts/bkgBarsSRC4to51.pdf} %\hspace{0.05\textwidth}
% \includegraphics[width=0.2\textwidth]{figures/barCharts/bkgBarsSRC4to52.pdf} %\hspace{0.05\textwidth}
%    \caption{Breakdown of the backgrounds in SRC4-5 from left to right.}
%    \label{fig:bkgSRC4to5}
%\end{center}
%\end{figure}
%
%\clearpage

%\input{histFitterTables/YieldsTable.SRD.tex}
\begin{table}[htpb]
  \caption{Observed and expected yields, before and after the fit, for SRD and SRE.
The uncertainties include MC statistical uncertainties, detector-related systematic uncetainties, and theoretical uncertainties in the extrapolation from CR to SR.}
  \begin{center}
{\renewcommand{\arraystretch}{1.2}
\begin{tabular}{
S[table-alignment=left]
S[table-number-alignment=left]
S[table-number-alignment=left]
S[table-number-alignment=left]
S[table-number-alignment=left]
S[table-number-alignment=left]
S[table-number-alignment=left]
}
\hline\hline
 & \multicolumn{2}{c}{SRD-low} & \multicolumn{2}{c}{SRD-high} & \multicolumn{2}{c}{SRE}\\ \hline 
{Observed} & \multicolumn{1}{l}{$27$} &  & \multicolumn{1}{l}{$11$} &  & \multicolumn{1}{l}{$3$} & \\ \hline \hline 
 \multicolumn{3}{r}{Fitted background events} & \multicolumn{4}{r}{}\\ \hline 
{Total SM} & \multicolumn{2}{l}{$25.1\phantom{0} \pm 6.2\phantom{0}$} & \multicolumn{2}{l}{$8.5\phantom{0} \pm 1.5\phantom{0}$} & \multicolumn{2}{l}{$3.64 \pm 0.79$}\\ \hline 
{\ttbar} & \multicolumn{2}{l}{$\phantom{2}3.3\phantom{0} \pm 3.3\phantom{0}$} & \multicolumn{2}{l}{$0.98 \pm 0.88$} & \multicolumn{2}{l}{$0.21\;_{-\;0.21}^{+\;0.39}$}\\ 
{\Wjets} & \multicolumn{2}{l}{$\phantom{2}6.1\phantom{0} \pm 2.9\phantom{0}$} & \multicolumn{2}{l}{$1.06 \pm 0.34$} & \multicolumn{2}{l}{$0.52 \pm 0.27$}\\ 
{\Zjets} & \multicolumn{2}{l}{$\phantom{2}6.9\phantom{0} \pm 1.5\phantom{0}$} & \multicolumn{2}{l}{$3.21 \pm 0.62$} & \multicolumn{2}{l}{$1.36 \pm 0.25$}\\ 
{\ttV} & \multicolumn{2}{l}{$\phantom{2}3.94 \pm 0.85$} & \multicolumn{2}{l}{$1.37 \pm 0.32$} & \multicolumn{2}{l}{$0.89 \pm 0.19$}\\ 
{Single top} & \multicolumn{2}{l}{$\phantom{2}3.8\phantom{0} \pm 2.1\phantom{0}$} & \multicolumn{2}{l}{$1.51 \pm 0.74$} & \multicolumn{2}{l}{$0.66 \pm 0.49$}\\ 
{Dibosons} & \multicolumn{2}{c}{${-} {-}$} & \multicolumn{2}{c}{${-} {-}$} & \multicolumn{2}{c}{${-} {-}$}\\ 
{Multijets} & \multicolumn{2}{l}{$\phantom{2}1.12 \pm 0.37$} & \multicolumn{2}{l}{$0.40 \pm 0.15$} & \multicolumn{2}{c}{${-} {-}$}\\ \hline \hline 
 \multicolumn{3}{r}{Expected events before fit} & \multicolumn{4}{r}{}\\ \hline 
{Total SM} & \multicolumn{2}{l}{$22.4\phantom{3}\phantom{3}$}  & \multicolumn{2}{l}{$7.7\phantom{3}\phantom{3}$}  & \multicolumn{2}{l}{$3.02\phantom{3}\phantom{4}$} \\ \hline 
{\ttbar} & \multicolumn{2}{l}{$\phantom{2}3.4\phantom{3}\phantom{3}$}  & \multicolumn{2}{l}{$1.04\phantom{3}\phantom{4}$}  & \multicolumn{2}{l}{$0.21\phantom{3}\phantom{4}$} \\ 
{\Wjets} & \multicolumn{2}{l}{$\phantom{2}4.8\phantom{3}\phantom{3}$}  & \multicolumn{2}{l}{$0.84\phantom{3}\phantom{4}$}  & \multicolumn{2}{l}{$0.42\phantom{3}\phantom{4}$} \\ 
{\Zjets} & \multicolumn{2}{l}{$\phantom{2}6.7\phantom{3}\phantom{3}$}  & \multicolumn{2}{l}{$3.10\phantom{3}\phantom{4}$}  & \multicolumn{2}{l}{$1.15\phantom{3}\phantom{4}$} \\ 
{\ttV} & \multicolumn{2}{l}{$\phantom{2}3.06\phantom{3}\phantom{4}$}  & \multicolumn{2}{l}{$1.07\phantom{3}\phantom{4}$}  & \multicolumn{2}{l}{$0.69\phantom{3}\phantom{4}$} \\ 
{Single top} & \multicolumn{2}{l}{$\phantom{2}3.3\phantom{3}\phantom{3}$}  & \multicolumn{2}{l}{$1.30\phantom{3}\phantom{4}$}  & \multicolumn{2}{l}{$0.56\phantom{3}\phantom{4}$} \\ 
{Dibosons} & \multicolumn{2}{l}{${-} {-}$} & \multicolumn{2}{l}{${-} {-}$} & \multicolumn{2}{l}{${-} {-}$}\\ 
{Multijets} & \multicolumn{2}{l}{$\phantom{2}1.12\phantom{3}\phantom{4}$}  & \multicolumn{2}{l}{$0.40\phantom{3}\phantom{4}$}  & \multicolumn{2}{l}{${-} {-}$}\\ 
\hline\hline
\end{tabular}

}
\end{center}
\label{tab:srDEYields}
\end{table}

%\begin{figure}[H] 
%\begin{center}
% \includegraphics[width=0.2\textwidth]{figures/barCharts/bkgBarsSRD1.pdf} %\hspace{0.05\textwidth}
% \includegraphics[width=0.2\textwidth]{figures/barCharts/bkgBarsSRD2.pdf} %\hspace{0.05\textwidth}
%  \includegraphics[width=0.2\textwidth]{figures/barCharts/bkgBarsSRE1.pdf} %\hspace{0.05\textwidth}
%    \caption[Breakdown of the backgrounds in SRD-low and -high.]{Breakdown of the backgrounds in SRD-low and -high from left to right.}
%    \label{fig:bkgSRD}
%\end{center}
%\end{figure}

\clearpage

%\input{histFitterTables/YieldsTable.SRE.tex}
%
%\begin{figure}[H] 
%\begin{center}
% \includegraphics[width=0.2\textwidth]{figures/barCharts/bkgBarsSRE1.pdf} %\hspace{0.05\textwidth}
%    \caption{Breakdown of the backgrounds in SRE.}
%    \label{fig:bkgSRE}
%\end{center}
%\end{figure}
%
%\clearpage

For discovery a p-value is calculated in each SR and subregion independently.  For the exclusion fits, the orthogonal subregions of SRA, SRB, and SRC are statistically combined.  Then SRA, B, C, D, and E are combined by taking the result with the best expected confidence level. In the case of overlapping signal regions the smallest 95\% confidence level is chosen for each model.
%The complete results of the background fits are shown in Appendix \ref{appendixHistfitter}.  

\section{Results}
\label{sec:results}



The observed yields compared to the background estimates (after applying the scale factors) for all SRs are shown in Tables \ref{table.bkgonly.SRA}, \ref{table.bkgonly.SRB}, \ref{table.bkgonly.SRC1to3}, \ref{table.bkgonly.SRD}, and \ref{table.bkgonly.SRE}.  No significant excess above the SM expectation is observed in any of the signal regions. \\
%, , table.bkgonly.SRC1to3, table.bkgonly.SRC4to5, table.bkgonly.SRD, table.bkgonly.SRE
% histFitterTables/YieldsTable.SRB, histFitterTables/YieldsTable.SRC1to3, histFitterTables/YieldsTable.SRC4to5, histFitterTables/YieldsTable.SRD, histFitterTables/YieldsTable.SRE

As shown in Figure \ref{fig:srSum} the VRs match the background estimation well.  It can be seen that the SRs do as well.


\begin{figure}[!hp] 
\begin{center}
 \includegraphics[width=0.8\textwidth]{figures/regionSummaryVR.pdf}\\ %\hspace{0.05\textwidth}
 \includegraphics[width=0.8\textwidth]{figures/regionSummaryLog.pdf}%\hspace{0.05\textwidth}
    \caption[Final yields for all the validation and signal regions]{Final yields for all the validation and signal regions. The stacked histograms show the SM expectation and the hatched uncertainty band around the SM expectation shows the MC statistical and detector-related systematic uncertainties.}
    \label{fig:srSum}
\end{center}
\end{figure}
\clearpage

%% Unblinded Results

Figures~\ref{fig:SRAunblinded},~\ref{fig:SRBunblinded},~\ref{fig:SRCunblinded},~\ref{fig:SRDunblinded},~\ref{fig:SREunblinded} show the postfit, unblinded distribution of some of the most discriminating variables of SRA, SRB, SRC, SRD, and SRE at 36.07 \ifb. For SRA and SRB the distributions for individual categories are shown. Additionally, the error bands include both MC statistical and all detector systematical uncertainties. \\

%\begin{figure}[!hp] 
%\begin{center}
%\includegraphics[width=0.45\textwidth]{figures/Met_SRA_TT.eps}
%\includegraphics[width=0.45\textwidth]{figures/Met_SRA_TW.eps}
%\includegraphics[width=0.45\textwidth]{figures/Met_SRA_T0.eps}
%\includegraphics[width=0.45\textwidth]{figures/MT2Chi2_SRA_TT.eps}
%\includegraphics[width=0.45\textwidth]{figures/MT2Chi2_SRA_TW.eps}
%\includegraphics[width=0.45\textwidth]{figures/MT2Chi2_SRA_T0.eps}
%\caption{Unblinded \met\ and \mttwo\ distributions for three SRA categories for 36.07 \ifb.}
%\label{fig:SRAunblinded}
%\end{center}
%\end{figure}
%\clearpage
%
%\begin{figure}[!hp] 
%\begin{center}
%\includegraphics[width=0.45\textwidth]{figures/MtBMax_SRB_TT.eps}
%\includegraphics[width=0.45\textwidth]{figures/MtBMax_SRB_TW.eps}
%\includegraphics[width=0.45\textwidth]{figures/MtBMax_SRB_T0.eps}
%\includegraphics[width=0.45\textwidth]{figures/MtBMin_SRB_TT.eps}
%\includegraphics[width=0.45\textwidth]{figures/MtBMin_SRB_TW.eps}
%\includegraphics[width=0.45\textwidth]{figures/MtBMin_SRB_T0.eps}
%\caption{Unblinded \mtbmax\ and \mtbmin\ distributions for SRB for 36.07 \ifb.}
%\label{fig:SRBunblinded}
%\end{center}
%\end{figure}
%\clearpage
%
%
%\begin{figure}[!hp] 
%\begin{center}
%  \includegraphics[width=0.45\textwidth]{figures/CA_RISR_SRC1_5.eps}
%  \includegraphics[width=0.45\textwidth]{figures/CA_PTISR_SRC1_5.eps}  
%\caption{Unblinded \rISR\ and \pTISR\ distributions for SRC1-5 for 36.07 \ifb.}
%\label{fig:SRCunblinded}
%\end{center}
%\end{figure}
%
%
%\begin{figure}[!hp] 
%\begin{center}
%\includegraphics[width=0.45\textwidth]{figures/MtBMax_SRD_low.eps}
%\includegraphics[width=0.45\textwidth]{figures/MtBMax_SRD_high.eps}
%\includegraphics[width=0.45\textwidth]{figures/JetPt_JetLeadTagIndex_JetPt_JetSubleadTagIndex__SRD_low.eps}
%\includegraphics[width=0.45\textwidth]{figures/JetPt_JetLeadTagIndex_JetPt_JetSubleadTagIndex__SRD_high.eps}
%\caption{Unblinded \mtbmax\ and \ptone\ distributions for SRD-low and SRD-high for 36.07 \ifb.}
%\label{fig:SRDunblinded}
%\end{center}
%\end{figure}
%\clearpage
%
%
%\begin{figure}[!hp] 
%\begin{center}
%\includegraphics[width=0.45\textwidth]{figures/HtSig_SRE.eps}
%\includegraphics[width=0.45\textwidth]{figures/Ht_SRE.eps}
%\includegraphics[width=0.45\textwidth]{figures/Met_SRE.eps}
%\caption{Unblinded \htsig, \HT, and \met\ distributions for SRE for 36.07 \ifb.}
%\label{fig:SREunblinded}
%\end{center}
%\end{figure}
%\clearpage

When no statistical excess is observed the cause may be the absence of signal, but can also be due to a downward fluctuation in the background.  In the case of a downward fluctuation the limit may be much better than the actual experimental sensitivity.  To account for this the CL\textsubscript{S} method\cite{CLs1, CLs2} along with the asymptotic formulae\cite{likelihoodFit} can be employed, where the probability is normalized to background-only probability.  In this case the limits are more conservative. \\ %Orthogonal signal subregions  \\

%When no significant excess if observed the 95\% confidence level of the signal events is obtained the simultaneous fit to the SRs and CRs using the CLS method as described in \cite{CLs1, CLs2} and the asymptotic formulae as described in \cite{likelihoodFit}. The CLS method scales the probability of excluding a signal by the probability that the data is consistent with the background, so if the background has a downward fluctuation, i.e. the number of events is less than expected, then the limit is increased.  \\

%The model-independent limits on the visible signal cross sections, $\sigma_{\textrm{vis}} \equiv \sigma \times A \times \epsilon$, where $\sigma$ is the production cross section, $A$ is the detector acceptance, and $\epsilon$ is the selection efficiency for a signal.  \\ %Maybe add this back in?

There are two types of limits that are evaluated: 

\begin{itemize}
\item Expected limits: obtained by setting the nominal event yield in each SR to the background expectation The $\pm 1 \sigma$ contours are evaluated using the  $\pm 1 \sigma$ uncertainties of the background estimates.
\item Observed limits: obtained by using the actual event yield and the $\pm 1 \sigma$ contours are evaluated by varying the signal cross section by the $\pm 1 \sigma$ of the theory uncertainties.  If the actual event yield is larger than the expected yield then the limit is weaker.
\end{itemize}



The results of the discovery fit for 36.07 \ifb\ are summarized by the
model-independent upper limits, as evaluated with asymptotics
and shown in Table \ref{table.results.exclxsec.pval.upperlimit}. In the asymptotic case, the calculator does not return a p0 value when the number of observed events is less than expected.  The table shows the 95\% confidence level upper limits on the visible cross section ($\langle \epsilon \sigma \rangle$, the detector acceptance multiplied by the efficiency), the number of signal events, the confidence level for the background-only hypothesis, and the discover p-values.  The smaller the p-value the more likely for the background-only hypothesis to be incorrect.  When the number of observed events are smaller than predicted a p-value of 0.50 is assigned.  The smallest p-values are 0.19, 0.24, 0.27, and 0.29 for SRB-T0, SRD-high, SRA-TT, and SRC3.  These values are too large to reject the background-only theory\footnote{For excluding a signal hypothesis a p-value greater than 0.05, which corresponds to a 95\% confidence level, is used.  The standard of a 5$\sigma$ excess for a discovery has a p-value of $2.87 \times 10^{-7}$.}. \\

\begin{table}[htpb]
  \caption[95\% CL upper limits]{Left to right: 95\% CL upper limits on the average visible cross section
($\langle\sigma A \epsilon\rangle_{\rm obs}^{95}$) where the average comes from possibly multiple production channels and on the number of
signal events ($S_{\rm obs}^{95}$ ).  The third column
($S_{\rm exp}^{95}$) shows the 95\% CL upper limit on the number of
signal events, given the expected number (and $\pm 1\sigma$
excursions of the expected number) of background events.
The two last columns indicate the CL$_\mathrm{B}$ value, i.e. the confidence level observed for the background-only hypothesis, and the discovery $p$-value ($p$) and the corresponding significance ($z$).
}
\label{tab:upLimits}

\begin{center}
    
% \begin{table}
% \begin{center}
% \setlength{\tabcolsep}{0.0pc}
{\renewcommand{\arraystretch}{1.3}
\begin{tabular*}{\textwidth}{@{\extracolsep{\fill}}lccccc}
\noalign{\smallskip}\hline\noalign{\smallskip}
{\bf Signal channel}                        & $\langle{\rm \sigma} A \epsilon\rangle_{\rm obs}^{95}$ [fb]  &  $S_{\rm obs}^{95}$  & $S_{\rm exp}^{95}$ & CL$_\mathrm{B}$ & $p$ ($z$)  \\
\noalign{\smallskip}\hline\noalign{\smallskip}
%%
% SRA-TT    & $0.31$ &  $11.2$ & $ { 9.1 }^{ +4.3 }_{ -2.5 }$ & $0.69$  & $ 0.29$~$(0.57)$ \\%
% SRA-TW    & $0.25$ &  $9.0$ & $ { 9.2 }^{ +3.6 }_{ -2.8 }$ & $0.47$
%  & $ 0.50$~$(0.00)$ \\%
% SRA-T0    & $0.40$ &  $14.6$ & $ { 14.9 }^{ +5.2 }_{ -4.3 }$ &
%  $0.46$ & $ 0.50$~$(0.00)$ \\%
% SRB-TT    & $0.65$ &  $23.4$ & $ { 24.0 }^{ +7.8 }_{ -6.8 }$ &
%  $0.46$ & $ 0.50$~$(0.00)$ \\%
% SRB-TW    & $0.73$ &  $26.2$ & $ { 26.0 }^{ +8.8 }_{ -6.6 }$ &
%  $0.52$ & $ 0.48$~$(0.05)$ \\%  
% SRB-T0    & $2.93$ &  $105.9$ & $ { 90.8 }^{ +24.3 }_{ -21.7 }$ &
% $0.72$ & $ 0.27$~$(0.61)$ \\%
% SRC1    & $0.44$ &  $16.0$ & $ { 16.3 }^{ +5.8 }_{ -4.2 }$ & $0.47$ &
%  $ 0.50$~$(0.00)$ \\%
% SRC2    & $0.35$ &  $12.6$ & $ { 15.5 }^{ +5.9 }_{ -4.2 }$ & $0.26$ &
% $ 0.50$~$(0.00)$ \\%
% SRC3    & $0.44$ &  $15.8$ & $ { 12.8 }^{ +4.7 }_{ -2.7 }$ & $0.69$ &
% $ 0.30$~$(0.54)$ \\%
% SRC4    & $0.09$ &  $3.1$ & $ { 6.5 }^{ +3.3 }_{ -2.1 }$ & $0.02$ & $
% 0.50$~$(0.00)$ \\%
% SRC5    & $0.06$ &  $3.0$ & $ { 3.9 }^{ +1.0 }_{ -0.3 }$ & $0.32$ & $
% 0.49$~$(0.02)$ \\%
% SRD-low    & $0.57$ &  $20.7$ & $ {
%   19.8 }^{ +6.8 }_{ -4.9 }$ & $0.57$ & $ 0.43$~$(0.19)$ \\%
% SRD-high    & $0.32$ &  $11.6$ & $ {
%   9.6 }^{ +3.9 }_{ -2.4 }$ & $0.71$ & $ 0.27$~$(0.60)$ \\%
% SRE    & $0.14$ &  $5.0$ & $ { 5.6 }^{ +2.7 }_{ -1.7 }$ & $0.40$ & $
% 0.50$~$(0.00)$ \\%


SRA-TT    & $0.30$ &  $11.0$ & $ { 8.7 }^{ +3.0 }_{ -1.4 }$ & $0.78$ & $ 0.23$~$(0.74)$ \\%
SRA-TW    & $0.27$ &  $9.6$ & $ { 9.6 }^{ +2.8 }_{ -2.1 }$ & $0.50$ & $ 0.50$~$(0.00)$ \\%
SRA-T0    & $0.31$ &  $11.2$ & $ { 11.5 }^{ +3.8 }_{ -2.0 }$ & $0.46$ & $ 0.50$~$(0.00)$ \\%
SRB-TT    & $0.54$ &  $19.6$ & $ { 20.0 }^{ +6.5 }_{ -4.9 }$ & $0.46$ & $ 0.50$~$(0.00)$ \\%
SRB-TW    & $0.60$ &  $21.7$ & $ { 21.0 }^{ +7.3 }_{ -4.3 }$ & $0.54$ & $ 0.37$~$(0.33)$ \\%
SRB-T0    & $2.19$ &  $80$ & $ { 58 }^{ +23 }_{ -17 }$ & $0.83$ & $ 0.13$~$(1.15)$ \\%
SRC1    & $0.42$ &  $15.1$ & $ { 15.8 }^{ +4.8 }_{ -3.5 }$ & $0.48$ & $ 0.50$~$(0.00)$ \\%
SRC2    & $0.31$ &  $11.2$ & $ { 13.9 }^{ +5.9 }_{ -3.6 }$ & $0.24$ & $ 0.50$~$(0.00)$ \\%
SRC3    & $0.42$ &  $15.3$ & $ { 12.3 }^{ +4.7 }_{ -3.4 }$ & $0.73$ & $ 0.27$~$(0.62)$ \\%
SRC4    & $0.10$ &  $3.5$ & $ { 6.7 }^{ +2.8 }_{ -1.8 }$ & $0.00$ & $ 0.50$~$(0.00)$ \\%
SRC5    & $0.09$ &  $3.2$ & $ { 3.0 }^{ +1.1 }_{ -0.1 }$ & $0.23$ & $ 0.23$~$(0.74)$ \\%
SRD-low    & $0.50$ &  $17.9$ & $ { 16.4 }^{ +6.3 }_{ -4.0 }$ & $0.62$ & $ 0.36$~$(0.35)$ \\%
SRD-high    & $0.30$ &  $10.9$ & $ { 8.0 }^{ +3.4 }_{ -1.3 }$ & $0.79$ & $ 0.21$~$(0.79)$ \\%
SRE    & $0.17$ &  $6.1$ & $ { 6.4 }^{ +1.4 }_{ -2.4 }$ & $0.42$ & $ 0.50$~$(0.00)$ \\%

\noalign{\smallskip}\hline\noalign{\smallskip}
\end{tabular*}
}
%\end{table}
%

\end{center}
\end{table}
%\input{histFitterTables/UpperLimitTable.tex}
\clearpage

%The results of the exclusion fits for SRA, B, C, D, and E are shown in Figure 
%The exclusion of orthogonal signal subregions, e.g. SRA-TT, -TW, and -T0, are statistically combined.   

\section{Interpretations}

The results have been interpreted in terms of a simplified model and for several pMSSM interpretations.

\subsection{Simplified Model}
The exclusion curves for the 100\% \stop $\ra t\ninoone$ \gls{br} is shown in Figure \ref{fig:SRABC_exclusion}.  Included in blue are the results from the 8 TeV analysis.  The observed and expected limits are shown in red and blue respectively.  For a Natural-SUSY inspired mixed grid scenario, where the \stop\ decays to a $t\ninoone$ or $b\chinoonepm$ with different branching ratios is shown in Figure \ref{fig:tbMet_exclusion}.  Finally the SRE results are interpreted for indirect top-squark production through gluino decays in terms of the \stop\ vs.\ $\gluino$ mass plane with $\Delta m(\stop,\ninoone)=5\GeV$. Gluino masses up to $m_{\gluino}=1800\GeV$ with $\mstop<800\GeV$ are excluded as shown in Fig. \ref{fig:SRE_exclusion}. 

%The yields for the validation and signal regions are shown in Figure \ref{fig:srSum}.  
%Figures~\ref{fig:SRAunblinded},~\ref{fig:SRBunblinded},~\ref{fig:SRCunblinded},~\ref{fig:SRDunblinded},~\ref{fig:SREunblinded} show the postfit, unblinded distribution of some of the most discriminating variables of SRA, SRB, SRC, SRD, and SRE at 36.07 \ifb. For SRA and SRB the distributions for individual categories are shown. Additionally, the error bands include both MC statistical and all detector systematical uncertainties. \\
%
%\begin{figure}[!hp] 
%\begin{center}
%\includegraphics[width=0.45\textwidth]{figures/Met_SRA_TT.eps}
%\includegraphics[width=0.45\textwidth]{figures/Met_SRA_TW.eps}
%\includegraphics[width=0.45\textwidth]{figures/Met_SRA_T0.eps}
%\includegraphics[width=0.45\textwidth]{figures/MT2Chi2_SRA_TT.eps}
%\includegraphics[width=0.45\textwidth]{figures/MT2Chi2_SRA_TW.eps}
%\includegraphics[width=0.45\textwidth]{figures/MT2Chi2_SRA_T0.eps}
%\caption{Unblinded \met\ and \mttwo\ distributions for three SRA categories for 36.07 \ifb.}
%\label{fig:SRAunblinded}
%\end{center}
%\end{figure}
%
%\begin{figure}[!hp] 
%\begin{center}
%\includegraphics[width=0.45\textwidth]{figures/MtBMax_SRB_TT.eps}
%\includegraphics[width=0.45\textwidth]{figures/MtBMax_SRB_TW.eps}
%\includegraphics[width=0.45\textwidth]{figures/MtBMax_SRB_T0.eps}
%\includegraphics[width=0.45\textwidth]{figures/MtBMin_SRB_TT.eps}
%\includegraphics[width=0.45\textwidth]{figures/MtBMin_SRB_TW.eps}
%\includegraphics[width=0.45\textwidth]{figures/MtBMin_SRB_T0.eps}
%\caption{Unblinded \mtbmax\ and \mtbmin\ distributions for SRB for 36.07 \ifb.}
%\label{fig:SRBunblinded}
%\end{center}
%\end{figure}
%
%
%
%\begin{figure}[!hp] 
%\begin{center}
%  \includegraphics[width=0.45\textwidth]{figures/CA_RISR_SRC1.eps}
%    \includegraphics[width=0.45\textwidth]{figures/CA_RISR_SRC2.eps}
%  \includegraphics[width=0.45\textwidth]{figures/CA_RISR_SRC3.eps}
%  \includegraphics[width=0.45\textwidth]{figures/CA_RISR_SRC4.eps}
%  \includegraphics[width=0.45\textwidth]{figures/CA_RISR_SRC5.eps}
%\caption{Unblinded \rISR\ and \pTISR\ distributions for SRC1-5 for 36.07 \ifb.}
%\label{fig:SRCunblinded}
%\end{center}
%\end{figure}
%
%
%\begin{figure}[!hxp] 
%\begin{center}
%\includegraphics[width=0.45\textwidth]{figures/MtBMax_SRD_low.eps}
%\includegraphics[width=0.45\textwidth]{figures/MtBMax_SRD_high.eps}
%\includegraphics[width=0.45\textwidth]{figures/JetPt_JetLeadTagIndex_JetPt_JetSubleadTagIndex__SRD_low.eps}
%\includegraphics[width=0.45\textwidth]{figures/JetPt_JetLeadTagIndex_JetPt_JetSubleadTagIndex__SRD_high.eps}
%\caption{Unblinded \mtbmax\ and \ptone\ distributions for SRD-low and SRD-high for 36.07 \ifb.}
%\label{fig:SRDunblinded}
%\end{center}
%\end{figure}
%
%
%\begin{figure}[!hp] 
%\begin{center}
%\includegraphics[width=0.45\textwidth]{figures/HtSig_SRE.eps}
%\includegraphics[width=0.45\textwidth]{figures/Ht_SRE.eps}
%\includegraphics[width=0.45\textwidth]{figures/Met_SRE.eps}
%\caption{Unblinded \htsig, \HT, and \met\ distributions for SRE for 36.07 \ifb.}
%\label{fig:SREunblinded}
%\end{center}
%\end{figure}
%

%There is no statistical excess in any of the signal regions.  





\begin{figure}[htpb]
  \begin{center} \includegraphics[width=0.7\textwidth]{figures/atlascls_m0m12_wband1_showcms0_StopZL2016_SRABCDE_Tt_directTTplusbWN_all_Output_fixSigXSecNominal_hypotest__1_harvest_list.pdf}%
    \caption[Exclusion contours as a function of $\stop$ and
      $\ninoone$ masses in the scenario where both top squarks decay
      via $\stop\to t^{(*)} \ninoone$.]{Observed (red solid line) and expected (blue solid line)
      exclusion contours at 95\% CL as a function of $\stop$ and
      $\ninoone$ masses in the scenario where both top squarks decay
      via $\stop\to t^{(*)} \ninoone$. Masses that are lower than the masses along the lines are excluded. Uncertainty bands corresponding to the $\pm 1
      \sigma$ variation on the expected limit (yellow band) and the
      sensitivity of the observed limit to $\pm 1\sigma$ variations of
      the signal theoretical uncertainties (red dotted lines) are also
      indicated. Observed limits from all third-generation Run-1 searches~\cite{Atlas8TeVSummary} at $\sqrt{s}=8$ TeV centre-of-mass energy are overlaid for comparison in blue.}
    \label{fig:SRABC_exclusion}%\label{fig:SRD4_exclusion}
  \end{center}
\end{figure}
\clearpage

\begin{figure}[htpb]
  \begin{center}
    \includegraphics[width=0.7\textwidth]{figures/SRABCD_mixed_dm1.pdf}
    \caption[Exclusion contours for the Natural SUSY-inspired mixed grid scenario]{Observed (solid line) and expected (dashed line) exclusion contours at 95\% CL as a function of $\stop$ and $\ninoone$ masses and branching ratio to $\stop\to t\LSP$ in the Natural SUSY-inspired mixed grid scenario where $m_{\chinoonepm}=\mLSP+1\gev$. %Uncertainty bands corresponding to the $\pm 1 \sigma$ variation on the expected limit (yellow band) and the sensitivity of the observed limit to $\pm 1\sigma$ variations of the signal theoretical uncertainties (red dotted lines) are also indicated. Observed limits from the Run~1 search~\cite{stop1L8TeV,stop2L8TeV} are overlaid for comparison. 
}
    \label{fig:tbMet_exclusion}
  \end{center}
\end{figure}
\clearpage

\begin{figure}[htpb]
  \begin{center}
    \includegraphics[width=0.7\textwidth]{figures/SRE_exclusion}
    \caption[Exclusion controus for the scenario where both
      gluinos decay via $\gluino\to t\stop\to t\ninoone$]{Observed (red solid line) and expected (blue solid line)
      exclusion contours at 95\% CL as a function
      of $\gluino$ and $\stop$ masses in the scenario where both
      gluinos decay via $\gluino\to t\stop\to t\ninoone+$soft
      and $\Delta m(\stop,\ninoone)=5\GeV$. Uncertainty bands corresponding to the $\pm 1
      \sigma$ variation on the expected limit (yellow band) and the
      sensitivity of the observed limit to $\pm 1\sigma$ variations of
      the signal theoretical uncertainties (red dotted lines) are also
      indicated. Observed limits from previous searches with the ATLAS detector at $\sqrt{s}=8$ and $\sqrt{s}=13$ TeV are overlaid in grey and blue~\cite{GtcStop1L,Gtc1L,GtcMonojet}.}
    \label{fig:SRE_exclusion}
  \end{center}
\end{figure}
\clearpage

%Three different likelihood fits are performed to extract the results:
%
%\begin{itemize}
%	\item Background-only fit: Only the control regions are used to constrain the fit parameters.  Potential signal contamination is neglected and the number of observed events in the signal region is not taken into account.
%	\item Exclusion fit: Both the CR and SR are used to contrain the fit parameters.  The signal contribution as predicted by the tested model is taken into account in both regions and the 
%\end{itemize}



%...are performed using HistFitter\cite{histfitter}.  HistFitter is a software framework for statistical data analysis used by the ATLAS Collaboration to analyze large datasets and is the standard statistical tool for SUSY searches.  


\subsection{pMSSM}
The results have also been interpreted in the context of pMSSM, which is described in \ref{sec:pmssm}.  There are three specific models within pMSSM for which the results have been interpreted:

\begin{itemize}
	\item Non-asymptotic higgsino: A simplified model motivated by naturalness with a higgsino LSP, ${m_{\chinoonepm}=\mLSP+5\gev}$, and ${m_{\ninotwo}=\mLSP+10\gev}$, assumes three sets of branching ratios for the considered decays of $\stop\to t\ninotwo$, $\stop\to t\LSP$, $\stop\to b\chinoonepm$~\cite{naturalSUSY}. A set of branching ratios with BR($\stop\to t\ninotwo$, $\stop\to t\LSP$, $\stop\to b\chinoonepm$) = 33\%, 33\%, 33\% is considered which is equivalent to a pMSSM model with a mostly left-handed top squark and $\tanb=60$ (ratio of vacuum expectation values of the two Higgs doublets). Additionally, BR($\stop\to t\ninotwo$, $\stop\to t\LSP$, $\stop\to b\chinoonepm$) = 45\%, 10\%, 45\% and BR($\stop\to t\ninotwo$, $\stop\to t\LSP$, $\stop\to b\chinoonepm$) = 25\%, 50\%, 25\% are assumed.% which correspond to scenarios with $\mqlthree < \mtr$ (regardless of the choice of \tanb) and $\mtr<\mqlthree$ with $\tanb=20$, respectively. Here \mqlthree\ represents the left-handed third-generation mass parameter and \mtr\ is the right-handed top-squark mass parameter. Limits in the \mstop\ and \mLSP\ plane are shown in Fig.~\ref{fig:nonAsymhiggsino_exclusion}.  

	 \item Wino NLSP pMSSM: This model is motivated by models with gauge unification at the GUT scale. The LSP is bino-like and has mass \mone\ and where the NLSP is wino-like with mass \mtwo, while $\mtwo=2\mone$ and $\mstop>\mone$~\cite{naturalSUSY}. Limits are set for both positive and negative $\mu$ (the higgsino mass parameter) as a function of the \stop\ and \ninoone\ masses which can be translated to different \mone\ and \mqlthree, and are shown in Fig.~\ref{fig:winoNLSP_exclusion}. Only bottom and top-squark production are considered in this interpretation. Allowed decays in the top-squark production scenario are $\stop\to t \ninotwo\to h/Z \LSP$, at a maximum branching ratio of 33\%, and $\stop \to b \chinoonepm$. Whether the $\ninotwo$ dominantly decays to a $h$ or $Z$ is determined by the sign of $\mu$. Along the diagonal region, the $\stop\to t\LSP$ decay with 100\% BR is also considered. The equivalent decays in bottom-squark production are $\sbottom\to t\chinoonepm$ and $\sbottom\to b\ninotwo$. %The remaining pMSSM parameters have the following values: $\mthree=2.2$ TeV (gluino mass parameter), $\ms=\sqrt{\stopone\stoptwo}=1.2$ TeV (geometric mean of top-squark masses), $\xtms=\sqrt{6}$ (mixing parameter between the left- and right-handed states, where $X_{t}=\at-\mu/\tanb$ and $\at$ is the trilinear coupling parameter in the top quark sector), and $\tanb=20$. All other pMSSM parameters are set to $>$3 TeV. 

	\item Well-tempered neutralino pMSSM: A model that provides a viable dark matter candidate in which three light neutralinos and a light chargino, which are composed as a mixture of bino and higgsino states, are considered with masses within $50$~\GeV\ of the lightest state~\cite{atlasDM,wellTemp}. The model is designed to satisfy the SM Higgs-boson mass and the dark matter relic density ($0.10<\Omega h^{2}<0.12$, where $\Omega$ is density parameter and $h$ is the Planck constant~\cite{relic_density}).% with pMSSM parameters: $\mone=-(\mu+\delta)$ where $\delta=20-50\gev$, $\mtwo=2.0$ TeV, $\mthree=1.8$ TeV, $\ms=0.8-1.2$~\TeV, $\xtms\sim\sqrt{6}$, and $\tanb=20$. For this model, limits are shown in Fig.~\ref{fig:wellTemp_exclusion}. Only bottom- and top-squark production are considered in this interpretation. The signal grid points were produced in two planes, $\mu$ vs \mtr\ and $\mu$ vs \mqlthree, and then projected to the corresponding \stop\ and \ninoone\ masses. All other pMSSM parameters are set to $>$3 TeV. 
\end{itemize}

\begin{figure}[htpb]
  \begin{center}
   \includegraphics[width=0.7\textwidth]{figures/SRABCD_tN1tN2bC1.pdf}
    \caption[Exclusion contours for the pMSSM-inspired non-asymptotic higgsino simplified model.]{Observed (solid line) and expected (dashed line) exclusion contours at 95\% CL as a function of \mstop\ and \mLSP\ for the pMSSM-inspired non-asymptotic higgsino simplified model for a small tan$\beta$ with BR($\stop\to t\ninotwo$, $\stop\to t\LSP$, $\stop\to b\chinoonepm$) = 45\%, 10\%, 45\% (blue), a large tan$\beta$ with BR($\stop\to t\ninotwo$, $\stop\to t\LSP$, $\stop\to b\chinoonepm$) = 33\%, 33\%, 33\% (red), and a small right-handed top-squark mass parameter with BR($\stop\to t\ninotwo$, $\stop\to t\LSP$, $\stop\to b\chinoonepm$) = 25\%, 50\%, 25\% (green) assumption. Uncertainty bands correspond to the $\pm 1 \sigma$ variation on the expected limit.}
    \label{fig:nonAsymhiggsino_exclusion}
  \end{center}
\end{figure}
\clearpage

\begin{figure}[htpb]
  \begin{center}
    \includegraphics[width=0.7\textwidth]{figures/SRABCD_winoNLSP.pdf}
    \caption[Exclusion contours for the Wino NLSP pMSSM model.]{Observed (solid line) and expected (dashed line) exclusion contours at 95\% CL as a function of $\stop$ and $\ninoone$ masses for the Wino NLSP pMSSM model for both positive (blue) and negative (red) values of $\mu$. Uncertainty bands correspond to the $\pm 1 \sigma$ variation on the expected limit. % Uncertainty bands corresponding to the $\pm 1 \sigma$ variation on the expected limit (yellow band) and the sensitivity of the observed limit to $\pm 1\sigma$ variations of the signal theoretical uncertainties (red dotted lines) are also indicated.
    }
    \label{fig:winoNLSP_exclusion}
  \end{center}
\end{figure}
\clearpage

\begin{figure}[htpb]
  \begin{center}
    \includegraphics[width=0.7\textwidth]{figures/SRABCD_wellTempered.pdf}
    \caption[Exclusion contour for the well-tempered pMSSM model.]{Observed (solid line) and expected (dashed line) exclusion contours at 95\% CL as a function of $\stop$ and $\ninoone$ masses for the left-handed top-squark mass parameter scan (red) as well as in the right-handed top-squark mass parameter scan (blue) in the well-tempered pMSSM model. Uncertainty bands correspond to the $\pm 1 \sigma$ variation on the expected limit.} 
    
    \label{fig:wellTemp_exclusion}
  \end{center}
\end{figure}
\clearpage


\section{Outlook}


The amount of data that the LHC has produced in Run 2 has exceeded expectations.  However, as can be seen in Section \ref{fig:1tevstopreach}, the rate at which increasing luminosity increases the discovery potential slows with more data.  This is due to the increase in background, and as the mass of the stop increases the cross section decreases. There is a need for much more data and the High-Luminosity LHC will be invaluable for searches in the future and improvements to the trigger system, such as triggering on large-radius jets as the gFEX is designed to do, will also improve chances of discovery.  \\


\begin{figure}[!h]
	\centering
	\includegraphics[width=\textwidth]{figures/significancesSRA.pdf}
%	\begin{subfigure}[b]{0.455\textwidth}
%		\includegraphics[width=\textwidth]{figures/significance25_50_100SRATT1000}
%		\caption{SRA-TT}
%	\end{subfigure}
%	\\
%	\begin{subfigure}[b]{0.455\textwidth}
%		\includegraphics[width=\textwidth]{figures/significance25_50_100SRATW1000}
%		\caption{SRA-TW}
%	\end{subfigure}
%	~
%	\begin{subfigure}[b]{0.455\textwidth}
%		\includegraphics[width=\textwidth]{figures/significance25_50_100SRAT01000}
%		\caption{SRA-T0}
%	\end{subfigure}
	\caption[Significance as a function of integrated luminosity for 1000~\gev\ stop]{Signal significant as a function of integrated luminosity for simplified model with $(\mstop,\mLSP) = (1000,1)~\gev$ and 100\% \stop $\ra t\ninoone$ \gls{br} for SRA subregions. The color lines represent different levels of background estimate uncertainty.}
	\label{fig:1tevstopreach}
\end{figure}
\clearpage


%Additionally, there are orthogonal stop searches in the 1-lepton and 2-lepton channels.  The 2-lepton exclusion limits puts the stop quark at 940 GeV with a massless neutrino.  The 1-lepton search puts the limit at 

The CMS experiment also has conducted a search for the all-hadronic decay of the stop, though with some different approaches.  For instance the method of identifying top quarks; instead of using mass categories the approach is to identify based on \pt.  For high \pt\  top, jets are reclustered with the \antikt\ algorithm with a distance parameter of R=0.8.  Substructure is also required and the tops, and kinematic variables are used for two separate BDTs, one for top quarks and one for $W$ bosons, which discriminate between signal and background.  Intermediate \pt\ candidates use the two highest $b-$tagged jets and then find a $W$ boson candidate to form resolved top candidates.  Various kinematic varables from these top-tagged jets are then input to a BDT to discriminate signal from background.  Stop masses up to 1040 GeV and neutralino masses up to 500 GeV are excluded with this search.  The multiple uses of BDTs improved the exclusion, but adds complications to the analysis.\\

There is also a need to find ways to remove background and improve the analysis, and there is ongoing R\&D for the analysis.  For example, at the time of this writing work to improve $b-$jet efficiency as well as calibrating smaller-radius jets, which can help resolve structure in larger radius jets, is being carried out.  In addition to this, it may be worth an effort to design some machine learning studies.  While a BDT study was used in this analysis and helped to design the top categories, much more sophisticated networks have been created that can be tested.  The downside of using machine learning is the risk of creating a black box where some complex nonlinear discrimination is being carried out, the details of which may be nearly impossible to figure out.  However it can at least provide a metric of what is possible to achieve with cuts.


%There is an additional requirement for the signal regions with a large mass difference between the stop and neutralino for the angle between the \met\ vector and highest \pt\ jet to be greater than 0.5, since the \met\ tends to be aligned with that jet.  This requirement is loosened for smaller mass differences.  Stop masses up to 1040 GeV and neutralino masses up to 500 GeV are excluded with this search. \\




%%limitations in the analysis, ideas for improvements, recommendations, evaluate and compare to CMS and their strengths and weaknesses

